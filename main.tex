\documentclass[12pt]{article}
\usepackage[ngerman]{babel}
\usepackage[utf8]{inputenc}
\usepackage{graphicx}
\usepackage[hidelinks]{hyperref}
\usepackage[a4paper, lmargin={4cm}, rmargin={2cm}, tmargin={2.5cm}, bmargin={2.5cm}]{geometry}
\usepackage{amssymb}
\usepackage{amsthm}
\usepackage{amsmath}
\usepackage{dsfont}
\graphicspath{ {images/} }

\usepackage{csquotes}
\usepackage{biblatex}
\addbibresource{chapters/bachelor_sources.bib}

\newtheorem{theorem}{Satz}[section]
\newtheorem{corollary}{Korollar}[theorem]
\newtheorem{lemma}[theorem]{Lemma}
\newtheorem{remark}[theorem]{Bemerkung}
\theoremstyle{definition}
\newtheorem{definition}[theorem]{Definition}


\DeclareMathOperator{\ip}{ip}
\DeclareMathOperator{\cpl}{Cpl}
\newcommand{\cplc}{\cpl_{\operatorname{c}}}
\newcommand{\cplbc}{\cpl_{\operatorname{bc}}}
\DeclareMathOperator{\uf}{uf}
\DeclareMathOperator{\id}{id}
\DeclareMathOperator{\CFP}{CFP}

\begin{document}

\title{
    {Adaptierte Wasserstein Metrik} \\
    {Westfälische Wilhelms-Universität}
}

\author{Kilian Mandon}
\date{\today}

\maketitle

\section*{Abstract}
Abstract goes here

\section{Einleitung}
Diese Arbeit bespricht die \emph{adaptierte Wasserstein-Metrik} wie sie von Daniel Bartl, Mathias Beiglböck und Gudmund Pammer in \cite{main_paper} vorgestellt wurde. Die adaptierte Wasserstein-Metrik $\mathcal{AW}_p$ ist eine Metrik auf dem Raum adaptierter stochastischer Prozesse $\FP_p$, die nicht nur die Verteilung, sondern auch die Filtration der Prozesse berücksichtigt. Für viele wichtige Operationen der Wahrscheinlichkeitstheorie wie die Doob-Zerlegung, Optimal-Stopping-Probleme oder Martingale ist die Berücksichtigung der gegebenen Filtration essentiell. Die adaptierte Wasserstein-Metrik kann diesen Anspruch in vielen Punkten erfüllen, so zeigen wir etwa in Abschnitt \ref{ch:topological_properties} die Stetigkeit der Doob-Zerlegung und die Stetigkeit des Ertrags eines Optimal-Stopping-Problems. 

Die Arbeit beginnt mit einer ausführlichen Vorstellung der Grundlagen für die adaptierte Wasserstein-Metrik. Kapitel \ref{ch:basics} bespricht die klassische Wasserstein-Metrik, bedingte Erwartungen und Verteilungen sowie die Eigenschaft der Bikausalität von Kopplungen. Es werden viele kleinere Aussagen gemacht, die für die übrige Arbeit wichtig sind.

Die Kapitel \ref{ch:wasserstein_space}, \ref{ch:adapted_functions} und \ref{ch:topological_properties} erarbeiten die wichtigsten Aussagen des zugrundeliegenden Papers \cite{main_paper} und stellen die Beweise in ausführlicherer Form dar. In Kapitel \ref{ch:wasserstein_space} wird ein polnischer Raum $(\mathcal{Z}_1, d)$ konstruiert, sodass $(\FP_p, \mathcal{AW}_p)$ isomorph zu $\mathcal{P}_p(\mathcal{Z}_1)$ mit der klassischen Wasserstein-Metrik ist. In Kapitel \ref{ch:adapted_functions} werden die \emph{adaptierten Funktionen} vorgestellt, die für adaptierte stochastische Prozesse separierend wirken: Zwei Prozesse $\mathbb{X,Y}$ sind genau dann gleich bezüglich der adaptierten Wasserstein-Metrik, wenn die Erwartungswerte $\mathbb{E}(f(\mathbb{X}))$ und $\mathbb{E}(f(\mathbb{Y}))$ für alle adaptierten Funktionen $f$ übereinstimmen. In Kapitel \ref{ch:topological_properties} werden wichtige geometrische und topologische Eigenschaften der adaptierten Wasserstein-Metrik hergeleitet. Es wird gezeigt, dass eine Familie $(\mathbb{X}_i)_{i\in I}$ genau dann relativ kompakt ist, wenn die Verteilungen $(\mathcal{L}(X_i))_{i\in I}$ relativ kompakt bezüglich der Wasserstein-Metrik sind, das heißt wenn $(\mathcal{L}(X_i))_{i\in I}$ straff und uniform $p$-integrierbar sind. Außerdem wird die Stetigkeit von wichtigen Operationen wie der Doob-Zerlegung und Optimal-Stopping gezeigt, sowie die Eigenschaft, dass $\FP_p$ ein geodätischer Raum ist.

Zusätzlich zur Vorstellung dieser Ergebnisse des Papers werden als eigene Resultate in Proposition \ref{thm:awp_convergence_char} die Konvergenz bezüglich $\mathcal{AW}_p$ charakterisiert und in Kapitel \ref{ch:implementation} der Fall von endlichen Wahrscheinlichkeitsräumen betrachtet, in welchem die adaptierte Wasserstein-Metrik mit Methoden der linearen Programmierung berechnet werden kann.

\section{Grundlagen}
\subsection{Notation}
Für den Verlauf dieser Arbeit fixieren wir einen Zeit-Horizont $N \in \mathbb{N}$ und $p \in [1,\infty)$. Für $1\leq t \leq N$ sei $\mathcal{X}_t$ ein polnischer Raum mit einer kompatiblen Metrik $d_{\mathcal{X}_t}$. Für Mengen $\left(A_{i}\right)_{s\leq i \leq r}$ kürzen wir das kartesische Produkt ab durch
$$A_{s:r} := A_s \times ... \times A_r$$
Mit dieser Notation schreiben wir $\mathcal{X}:=\mathcal{X}_{1:N}$.

Für einen polnischen Raum $A$ mit kompatibler Metrik $d_A$ betrachten wir zwei Räume von Maßen: $\mathcal{P}(A)$, den Raum der Borel-Wahrscheinlichkeitsmaße auf $A$, und die Teilmenge $\mathcal{P}_p(A) \subset \mathcal{P}(A)$ der Wahrscheinlichkeitsmaße mit endlichem $p$-ten Moment, das heißt
$$\int d_A^p(a_0, x) \mu(dx) < \infty$$
für ein fixes (und damit für alle) $a_0 \in A$. Für zwei Wahrscheinlichkeitsmaße $\mu \in \mathcal{P}(A), \nu \in \mathcal{P}(B)$ schreiben wir $\cpl(\mu, \nu)$ für die Menge aller \emph{Kopplungen} von $\mu$ und $\nu$, das heißt Maße $\pi \in \mathcal{P}(A \times B)$ mit $\pi(\cdot, B)\equiv \mu$ und $\pi(A, \cdot)=\nu$.

\subsection{Wasserstein-Distanz}
\begin{definition}[Wasserstein-Distanz]\label{def:wasserstein_distance}
    Sei $A$ polnisch mit einer fixierten kompatiblen Metrik $d_A$. Die $p$-te Wasserstein"=Distanz zwischen zwei Wahrscheinlichkeitsmaßen $\mu, \nu \in \mathcal{P}_p(A)$ ist definiert als
    $$\mathcal{W}_p(\mu, \nu) := \left( \inf_{\pi \in \cpl(\mu, \nu)} \int d^p_A(x, y) \pi(dx, dy) \right)^\frac{1}{p}$$
\end{definition}
Wir versehen $\mathcal{P}(A)$ mit der Topologie von schwacher Konvergenz, das heißt eine Basis der Topologie sind Mengen $\{U_{f, x, r} \vert f \in C_b(A), x\in \mathbb{R}, r>0\}$ der Form 
$$U_{f, x, r} = \left\{ \mu \in \mathcal{P}(A) \vert \left| \int f d\mu - x\right| < r\right\}$$
Bezüglich dieser Topologie betrachten wir auf $\mathcal{P}(A)$ die Borel $\sigma$"=Algebra. Auf $\mathcal{P}_p(A)$ ist die Topologie induziert durch $\mathcal{W}_p$, und auch hier betrachten wir die assoziierte Borel $\sigma$-Algebra.
\begin{lemma}\label{thm:closed_couplings}
Seien $\mathcal{A,B}$ polnische Räume. Dann gilt: 
\begin{enumerate}
    \item Für $(\mu_n)_{n\in \mathbb{N}} \subset\mathcal{P}(A), \mu_n \rightarrow \mu \in \mathcal{P}(A)$ ist der Grenzwert eindeutig.
    \item Für $(\pi_n) \subset \mathcal{P}(\mathcal{A} \times \mathcal{B}), \pi_n \rightarrow \pi$ schwach konvergieren auch die Marginalien von $\pi_n$ schwach gegen die von $\pi$. 
    \item Für $\mu \in \mathcal{P}(\mathcal{A}), \nu \in \mathcal{P}(\mathcal{B})$ ist $\cpl(\mu, \nu)$ abgeschlossen.
\end{enumerate}
\end{lemma}
\begin{proof}
    Seien $\mu, \tilde{\mu}$ zwei Grenzwerte von $(\mu_n)$ in schwacher Konvergenz. Sei $E \subseteq \mathcal{A}$ abgeschlossen. Für $m \in \mathbb{N}$ betrachte die stetige Funktion 
    $$f_m: x \mapsto (1-m\cdot d(x,E)) \vee 0$$
    $f_m$ ist stetig und beschränkt, also gilt 
    $$\lim_{n \rightarrow \infty} \mu_n(f_m) = \mu(f_m) = \tilde{\mu}(f_m)$$
    Es konvergiert $f_m \rightarrow \mathds{1}_E$ fast sicher dominiert durch $1$, somit gilt also auch $\mu(E) =\tilde{\mu}(E)$. Da abgeschlossene Mengen ein $\cap$-stabiler Erzeuger sind, folgt $\mu=\tilde{\mu}$ und damit die erste Behauptung.

    Sei $(\pi_n)_{n\in\mathbb{N}} \subset \mathcal{P}(\mathcal{A}\times\mathcal{B})$ eine Folge und konvergent gegen $\pi$. Wir schreiben $\mu_n, \nu_n$ bzw. $\mu, \nu$ für die Marginalien von $\pi_n$ bzw. $\pi$. Sei $f\in C_b(\mathcal{A})$. Wir können $f \in C_b(\mathcal{A}\times\mathcal{B})$ auffassen als $x,y\mapsto f(x)$. Da $f$ stetig und beschränkt ist, gilt 
    $$\mu(f) = \pi(f) = \lim_{n\rightarrow \infty} \pi_n(f) = \lim_{n\rightarrow\infty} \mu_n(f)$$ 
    Daraus folgt die zweite Behauptung.

    Für eine Folge von Kopplungen $(\pi_n)_{n\in\mathbb{N}} \subset \cpl(\mu,\nu)$ konvergent gegen $\pi$ müssen also auch die Marginalien von $pi_n$ gegen die von $\pi$ konvergieren. Die Marginalien sind aber konstant $\mu$ und $\nu$, und wegen der Eindeutigkeit des Grenzwertes nach dem ersten Punkt gilt $\pi \in \cpl(\mu,\nu)$.
\end{proof}
\begin{lemma}\label{thm:optimal_coupling}
In \ref{def:wasserstein_distance} wird dass Infimum tatsächlich angenommen, das heißt es existiert ein $\pi \in \cpl(\mu, \nu)$, sodass
$$\mathcal{W}_p(\mu, \nu) = \left( \int d^p_A(x, y) \pi(dx, dy) \right)^{\frac{1}{p}}$$
Tatsächlich kann eine Borel-messbare Abbildung $k: \mathcal{P}_p(A) \times \mathcal{P}_p(A) \rightarrow \mathcal{P}_p(A\times A)$ gewählt werden, sodass für $\mu,\nu \in \mathcal{P}_p(A)$ $k(\mu, \nu)$ eine optimale Kopplung ist.
\end{lemma}
Für den Beweis benötigen wir den Satz von Prokhorov. Auf einem topologischen Raum $X$ mit Borel Algebra nennen wir eine Menge $M \subset \mathcal{P}(X)$ straff, falls für jedes $\varepsilon >0$ eine kompakte Menge $K\subset X$ existiert, sodass für jedes $\mu \in M$ gilt $\mu(X\setminus K) < \varepsilon$. Auf polnischen Räumen gibt der Satz von Prokhorov eine nützliche Charakterisierung von Straffheit:
\begin{theorem}[Satz von Prokhorov]\label{thm:prokhorov}
Sei $(A, d)$ ein polnischer Raum. Dann ist die Topologie schwacher Konvergenz auf $\mathcal{P}(A)$ vollständig metrisierbar und es gilt folgende Äquivalenz:

Eine Menge $M \subset \mathcal{P}(A)$ ist straff, genau dann wenn sie relativ kompakt ist.
\end{theorem}
\begin{proof}\ref{thm:optimal_coupling}
    \begin{enumerate}
        \item
    Zuerst zeigen wir die Existenz einer optimalen Kopplung. Seien $\mu, \nu \in \mathcal{P}_p(A)$. Die Mengen $\{\mu\}$ und $\{\nu\}$ sind straff nach dem Satz von Prokhorov, da sie bezüglich schwacher Konvergenz abgeschlossen und folgenkompakt sind. Wir können also für ein $\varepsilon > 0$ kompakte Mengen $K_1$ und $K_2$ wählen, sodass $\mu(A\setminus K_1) < \varepsilon$ und $\nu(A\setminus K_2) < \varepsilon$. Die Menge $K_1 \times K_2$ ist immer noch kompakt, und für $\pi \in \cpl(\mu, \nu)$ gilt 
    $$\pi(A\times A \setminus K_1 \times K_2) \leq \pi(A\times A \setminus K_1 \times A) + \pi(A\times A \setminus A \times K_2) \leq 2\varepsilon$$
    Also ist $\cpl(\mu, \nu)$ straff und mit dem Satz von Prokhorov relativ kompakt. Nach Lemma \ref{thm:closed_couplings} ist sind die Kopplungen auch abgeschlossen, also sogar kompakt. Wähle eine Folge $\pi_n\in \cpl(\mu, \nu)$ mit 
    $$\left(\mathbb{E}_{\pi_n}(d^p(x,y))\right)^{\frac{1}{p}} \rightarrow \mathcal{W}_p(\mu, \nu)$$
    Nach Übergang zu einer Teilfolge können wir annehmen, dass $\pi_n \rightarrow \pi \in \cpl(\mu, \nu)$. Für jedes $m \in \mathbb{N}$ ist $x,y\mapsto d^p(x,y)\wedge m$ stetig und beschränkt, also 
    $$\mathbb{E}_\pi(d^p(x,y)\wedge m) = \lim_{n\rightarrow \infty} \mathbb{E}_{\pi_n}(d^p(x,y) \wedge m) \leq \mathcal{W}_p^p(\mu, \nu)$$
    Für $m \rightarrow \infty$ folgt wegen dominierter Konvergenz $\mathbb{E}_\pi(d^p(x,y))\leq \mathcal{W}_p^p(\mu, \nu)$ und somit ist $\pi$ eine optimale Kopplung.
\item 
    Wir benutzen folgende Aussage über messbare Auswahl:

    \emph{Eine surjektive, Borel-messbare Abbildung $f$ zwischen polnischen Räumen, für die alle Fasern $f^{-1}(y)$ kompakt sind, hat eine Borel-messbare Rechtsinverse.}

    \end{enumerate}
    % TODO Proof optimal coupling
\end{proof}
\begin{definition}
    Sei $A$ ein polnischer Raum mit fixierter Metrik $d_A$. Wir nennen eine Familie $\mathcal{K} \subset \mathcal{P}_p(A)$ \emph{$p$-uniform integrierbar}, falls für alle $\varepsilon>0$ und $a_0 \in A$ ein $R_\varepsilon>0$ existiert, sodass
    $$\sup\limits_{\mu \in \mathcal{K}} \int\limits_{A\setminus B_{R_\varepsilon}(a_0)} d_A^p(a_0, x)\mu(dx) \leq \varepsilon $$
    Für eine Funktion $f: A \rightarrow \mathbb{R}$ sagen wir \emph{f hat $p$-Wachstum}, falls 
    $$|f(x)|\leq a(d_A^p(a_0, x) + 1)$$
    für ein $a\in \mathbb{R}$ und $a_0 \in A$.
\end{definition}
Mit der Dreiecksungleichung gelten beide Eigenschaften wenn sie für ein $a_0 \in A$ gelten schon für alle $a_0 \in A$.
Für Konvergenz in $\mathcal{P}_p(A)$ gilt folgender Zusammenhang:
\begin{lemma}\label{thm:conv_char}
Sei $A$ ein polnischer Raum mit fixierter Metrik $d_A$.Sei $(\mu_n) \subset \mathcal{P}_p(A), \mu \in \mathcal{P}_p(A)$. Dann sind die folgenden äquivalent:
\begin{enumerate}
    \item $\mu_n \rightarrow \mu$ bezüglich $\mathcal{W}_p$
    \item $\mu_n \rightarrow \mu$ schwach und $(\mu_n)$ sind $p$-uniform integrierbar.
    \item $\mu_n(f) \rightarrow \mu(f)$ für jedes stetige $f$ mit $p$-Wachstum.
    \item $\mu_n \rightarrow \mu$ schwach und $\mu_n(d_A^p(\cdot, a_0)) \rightarrow \mu(d_A^p(\cdot, a_0))$ für ein $a_0 \in A$
\end{enumerate}
\end{lemma}
\begin{proof}
% TODO: Referenz Paper Ambrosio Giggli
Wir beweisen die Äquivalenz zwischen den Punkten (2) bis (4). Die Anbindung an Punkt (1) ist umfangreicher und kann zum Beispiel bei (TODO Quelle Villani, OT old and new) gefunden werden. \\ 

(2)$\Rightarrow$(3): \\
Wir können $f$ in $f_+ = \max(0, f)$ und $f_-=\max(0, -f)$ zerlegen mit $f=f_+-f_-$. Diese sind nichtnegativ und weiterhin stetig mit $p$-Wachstum. Wir können also annehmen, dass $f\geq 0$. 

Für $R>0$ seien $\chi_R:A\rightarrow[0,1]$ stetig mit kompaktem Träger und $\chi_{R_{\vert_{B_R(a_0)}}}=1$, monoton wachsend in $R$. Wir können $f$ von unten approximieren durch stetige beschränkte Funktionen $f_m = f\chi_m$. Es gilt $\mu(f) = \lim\limits_{m\rightarrow \infty} \mu(f_m)$ nach monotoner Konvergenz und für jedes $m \in \mathbb{N}$ gilt 
$$\mu(f_m) = \lim\limits_{n\rightarrow\infty}\mu_n(f_m) \leq \liminf\limits_{n\rightarrow\infty}\mu_n(f)$$
und somit auch 
$$\mu(f) = \lim_{m\rightarrow\infty} \mu(f_m) \leq \liminf\limits_{n\rightarrow\infty}\mu_n(f)$$
Für die andere Ungleichung, fixiere $\varepsilon>0, a_0 \in A$ und wähle $R>1$ sodass
$$\sup\limits_{n \in \mathbb{N}} \int\limits_{A\setminus B_R(a_0)}d_A^p(a_0, x)\mu_n(dx) < \varepsilon$$
Setze $\chi:=\chi_R$.
Nun ist, da $f(x)\leq C(1+d_A^p(a_0, x)) \leq 2Cd_A^p(a_0, x), x\notin B_r(a_0)$ 
$$\int fd\mu_n = \int f\chi d\mu_n + \int f(1-\chi)d\mu_n \leq \int f\chi d\mu_n + 2C\varepsilon \rightarrow \mu(\chi f) + 2C\varepsilon$$
und 
$$\limsup_{n\rightarrow\infty}\mu_n(f) \leq \mu(\chi f) + 2C\varepsilon \leq \mu(f) + 2C\varepsilon$$
Da $\varepsilon$ beliebig folgt $\mu_n(f) \rightarrow \mu(f)$

(3)$\Rightarrow$(4) ist klar, da $d^p_A(a_0, \cdot)$ stetig mit $p$-Wachstum ist.

(4)$\Rightarrow$(2): \\
Angenommen es existiert ein $\varepsilon > 0$ und $a_0 \in A$ sodass für jedes $R>0$ 
\begin{equation} \label{eq:1}
\sup\limits_{n\in\mathbb{N}}\int_{A\setminus B_R(a_0)}d_A^p(a_0, x)\mu_n(dx)>\varepsilon
\end{equation}
Beachte, dass es für jedes $R>0$ unendlich viele $n \in \mathbb{N}$ geben muss, die \ref{eq:1} erfüllen: Ansonsten könnte man R so vergrößern, dass für die endlich vielen Terme das Integral unter $\varepsilon$ gezwungen wird, und hätte damit einen Widerspruch. Es gilt also
$$\limsup\limits_{n\in\mathbb{N}}\int_{A\setminus B_R(a_0)}d^p_A(a_0, x)\mu_n(dx)>\varepsilon$$
Wähle für jedes $R>0$ $\chi_R$ stetig mit kompaktem Träger in $B_R(a_0)$ und identisch $1$ auf $B_{\frac{R}{2}}(a_0)$.
\begin{align*}
    \int d^p_A(a_0, x)\chi_R\mu(dx) &= \lim\limits_{n\rightarrow\infty}\int d^p_A(a_0, x)\chi_R \mu_n(dx) \\
    &= \lim\limits_{n\rightarrow\infty}\left(\int d^p_A(a_0, x)\mu_n(dx) - \int d^p_A(a_0, x)(1-\chi_R)\mu_n(dx) \right) \\
    &= \int d^p_A(a_0, x)\mu(dx) - \lim\limits_{n\rightarrow\infty}\int d^p_A(a_0, x)(1-\chi_R)\mu_n(dx) \\
    &\leq \int d^p_A(a_0, x)\mu(dx) - \limsup\limits_{n\rightarrow\infty} \int_{A\setminus B_R(a_0)}d^p_A(a_0, x)\mu_n(dx) \\
    &\leq \int d^p_A(a_0, x)\mu(dx) - \varepsilon
\end{align*}
und nun folgt
$$\int d^p_A(a_0, x) = \lim\limits_{n\rightarrow\infty} \int d^p_A(a_0, x)\chi_n\mu(dx) \leq \int d^p_A(a_0, x)\mu(dx) - \varepsilon$$
was ein Widerspruch ist.
\end{proof}
\begin{lemma} \label{thm:weak_topology_metric}
Sei $A$ ein polnischer Raum mit fixierter Metrik $d_A$. Betrachte weiterhin die beschränkte Metrik $\widehat{d}:=d_A \wedge 1$. Für Verteilungen $\mu_n \subset \mathcal{P}(A)$, $\mu \in \mathcal{P}(A)$ konvergieren $\mu_n \rightarrow \mu$ schwach genau dann, wenn $\mu_n \rightarrow \mu$ bezüglich $\mathcal{W}_p$ auf $(A, \widehat{d})$. Die beschränkte Wassersteinmetrik metrisiert also die schwache Topologie.
\end{lemma}
\begin{proof}
    Gelte zunächst $\mu_n \rightarrow \mu$ schwach und fixiere ein $a_0 \in A$. Die Funktion $\widehat{d}^p(\cdot, a_0)$ ist stetig und beschränkt, also gilt $\mu_n(\widehat{d}^p(\cdot, a_0)) \rightarrow \mu(\widehat{d}^p(\cdot, a_0))$ und mit Lemma \ref{thm:conv_char} folgt $\mu_n \rightarrow \mu$ bezüglich $\mathcal{W}_p$ auf $(A, \widehat{d})$.

    Umgekehrt, gelte $\mu_n \rightarrow \mu$ bezüglich $\mathcal{W}_p$ auf $(A, \widehat{d})$. Die Frage von schwacher Konvergenz ist nur eine Frage der Topologie und nicht der konkreten Metrik, und die Topologien auf $(A, d_A)$ und $(A, \widehat{d})$ sind identisch. Somit folgt auch wieder aus Lemma \ref{thm:conv_char}, dass $\mu_n \rightarrow \mu$ schwach.
\end{proof}
\begin{remark}
    Die Wahl von $\widehat{d}$ als $d_A \wedge 1$ ist nicht zentral für den Beweis, wichtig ist lediglich dass die Metrik beschränkt ist und bei kleinen Abständen identisch, damit die gleiche Topologie erzeugt wird. Wir werden dieses Lemma später auch auf Produkträumen $X \times Y$ benutzen, in denen wir in den einzelnen Räumen die Metrik beschränken und dann die Produktmetrik dieser neuen Metriken verwenden. Das ist kein Problem, da die Metrik trotzdem beschränkt und bei kleinen Abständen identisch ist.
\end{remark}
\subsection{Bedingte Erwartungen und Bedingte Verteilungen}
\begin{definition}
Seien $(\Omega, \mathcal{F}), (S, \mathcal{S})$ messbare Räume. Mit einem stochastischen Kern $\mu:\Omega \rightarrow S$ bezeichnen eine Abbildung $\Omega \rightarrow \mathcal{P}(S), x\mapsto \mu_x$ sodass für jede Menge $A \in \mathcal{S}$ die Evaluations-Abbildung $x \mapsto \mu_x(A)$ messbar bezüglich der Borel $\sigma$-Algebra auf $[0,1]$ ist.
\end{definition}
Die bedingte Verteilung wird ein Kernbaustein der Konstruktion vom Informationsprozess. Die Existenz beruht auf dem Folgenden Satz:
%TODO: Herausfinden ob S wirklich polnisch sein muss, Referenz Kallenberg
\begin{theorem}\label{thm:disintegration}
Seien $(S, \mathcal{S}), (T,\mathcal{T})$ polnische Räume mit ihren Borel $\sigma$-Algebren. Sei $\rho$ eine Verteilung auf $S\times T$ und $\nu$ die Marginalie von $\rho$ über $S$, d.h. $\nu(A) = \rho(A\times T)$. Dann gibt es einen $\nu$-fast sicher eindeutigen stochastischen Kern $\mu:S\rightarrow T$, sodass $\rho = \nu \otimes \mu$.
\end{theorem}
\begin{definition}
Für zwei Zufallsvariablen $X, Y$ auf $(\Omega, \mathcal{A}, \mathbb{P})$ mit Werten in polnischen Räumen $(S, \mathcal{S})$ und $(T, \mathcal{T})$ existiert nach \ref{thm:disintegration} ein fast sicher eindeutiger Kern $\mu: S\rightarrow T$ mit $\mathbb{P}^{X, Y}=\mathbb{P}^X\otimes \mu$. Diesen bezeichnen wir als \emph{bedingte Verteilung von $Y$ bezüglich $X$} und schreiben $\mathcal{L}(Y \vert X)$.

Für eine $\sigma$-Algebra $\mathcal{F} \subset \mathcal{A}$ definieren wir (falls $\Omega$ polnisch ist) die bedingte Verteilung $\mathcal{L}(Y\vert \mathcal{F})$ wie zuvor bezüglich der Identitätsabbildung $\operatorname{id}:(\Omega, \mathcal{A}) \rightarrow (\Omega, \mathcal{F})$, das heißt als den fast-sicher eindeutigen $\mathcal{F}$-messbaren Kern $\mu:\Omega\rightarrow S$, für den gilt $\left(\mathbb{P}_{\vert \mathcal{F}} \otimes \mu\right)(A\times B)=\mathbb{P}(A, Y\in B)$ für alle $A \in \mathcal{F}, B\in \mathcal{S}$.
\end{definition}
Eng verwandt mit der bedingten Verteilung ist die bedingte Erwartung:
\begin{definition}
Für eine reellwertige, integrierbare Zufallsvariable $X: (\Omega, \mathcal{F}) \rightarrow (\mathbb{R}, \mathcal{B}(\mathbb{R}))$ und eine Unter-$\sigma$-Algebra $\mathcal{G}\subset \mathcal{F}$ definieren wir die bedingte Erwartung $\mathbb{E}(X\vert \mathcal{G})$ als die fast-sicher eindeutige $\mathcal{G}$-messbare Zufallsvariable $\tilde{X}$, sodass für alle beschränkten $\mathcal{G}$-messbaren $Y$
$$\mathbb{E}(\tilde{X}Y) = \mathbb{E}(XY)$$
Für Zufallsvariablen $X$ und $Y$ schreiben wir $\mathbb{E}(Y\vert X)$ für $\mathbb{E}(Y \vert \sigma(X))$.
\end{definition}
Die bedingte Verteilung und die bedingte Erwartung haben folgenden Zusammenhang:
\begin{lemma}\label{thm:law_expectancy_connection}
Seien $X: (\Omega, \mathcal{F})\rightarrow (S, \mathcal{S}), Y: (\Omega, \mathcal{F})\rightarrow(T, \mathcal{T})$ Zufallsvariablen mit Werten in polnischen Räumen $(S, \mathcal{S})$ und $(T, \mathcal{T})$ und sei $\mu: S\rightarrow T$ ein stochastischer Kern. Dann ist $\mu = \mathcal{L}(Y\vert X)$ genau dann wenn für alle messbaren, beschränkten $f: T\rightarrow \mathbb{R}$
$$\mathbb{E}(f(Y) \vert X) = \int f(y)\mu_X(dy) \text{ fast sicher}$$
\end{lemma}
\begin{proof}
    "$\Rightarrow$": \\
    Sei $f:T\rightarrow \mathbb{R}$ messbar und beschränkt. Mit einem Funktionserweiterungsargument ist $\int f(y)\mu_X(dy)$ messbar bezüglich $\sigma(X)$. Für Mengen $A' \in \sigma(X), A'=X^{-1}(A)$ ist 
    \begin{align*}
    \mathbb{E}(f(Y(\omega))\mathds{1}_{A'}(\omega)) &= \mathbb{E}(f(Y)\mathds{1}_A(X))\\
    &=\int \int f(y) \mathds{1}_A(x)\mu_x(dy)\mathbb{P}^X(dx) \\
    &= \int \mathds{1}_{A'}(\omega) \int f(y)\mu_{X(\omega)}(dy) \mathbb{P}(d\omega)
    \end{align*}
    Mit einem Funktionserweiterungsargument folgt für alle beschränkten, $\sigma(X)$ messbaren $g: \Omega\rightarrow \mathbb{R}$
    $$\mathbb{E}(f(Y(\omega))g(\omega)) = \int g(\omega)\int f(y)\mu_{X(\omega)}(dy)\mathbb{P}(d\omega)$$
    und damit die Behauptung. \\
    "$\Leftarrow$": \\
    Seien $A \in \mathcal{S}, B\in \mathcal{T}$. Dann gilt
    \begin{align*}
        \mathbb{P}(X\in A, Y\in B) &= \mathbb{E}(\mathds{1}_A(X) \mathds{1}_B(Y)) \\
        &= \mathbb{E}(\mathds{1}_A(X)\mathbb{E}(\mathds{1}_B(Y)\vert X)) \\
        &= \int \int \mathds{1}_B(Y) \mu_x(dy) \mathds{1}_A(x)\mathbb{P}^X(dx) \\
        &= (\mathbb{P}^X \otimes \mu)(A\times B)
    \end{align*}
    und somit ist $\mu=\mathcal{L}(Y\vert X)$
\end{proof}
\begin{lemma}\label{thm:kernel_characterization}
    Seien $(S, \mathcal{S}), (T, \mathcal{T})$ polnische Räume und $\mu: S\rightarrow \mathcal{P}_p(T)$ eine Abbildung. $\mu$ ist ein stochastischer Kern, genau dann wenn es eine messbare Abbildung bezüglich der Borel-Algebra auf $\mathcal{P}_p$ ist.
\end{lemma}
\begin{proof}
"$\Rightarrow$": \\
Die $\sigma$-Algebra auf $\mathcal{P}_p$ wird nach Lemma \ref{thm:conv_char} von der gröbsten Topologie erzeugt, bezüglich der die Abbildungen $\pi_f: \mu \mapsto \mu(f)$ für $f$ stetig mit $p$-Wachstum stetig sind. Daher wird auch die Borel-Algebra von diesen Abbildungen erzeugt. Damit die Abbildung $\mu: x\mapsto \mu_x$ müssen also genau alle Abbildungen $\pi_f \circ \mu: x\mapsto \mu_x(f)$ messbar sein. Das sind sie aber über ein Funktionserweiterungsargument sogar für alle integrierbaren, messbaren $f$: Für Indikatorfunktionen $\chi_A$ sind das genau die Abbildungen $x\mapsto \mu_x(A)$ die nach Voraussetzung messbar sind, und Messbarkeit bezüglich der Borel-Algebra auf $\mathbb{R}$ ist abgeschlossen unter Linearkombinationen und Limiten. Somit ist die Abbildung $x \mapsto \mu_x$ messbar bezüglich der Borel-Algebra auf $\mathcal{P}_p$. \\
"$\Leftarrow$": \\
Sei nun die Abbildung $x \mapsto \mu_x$ messbar. Um zu zeigen, dass Abbildungen $x\mapsto \mu_x(A)$ messbar sind, reicht es zu zeigen, dass die Evaluationsabbildungen $\mu \mapsto \mu(A)$ messbar sind bezüglich der Borel-Algebra auf $\mathcal{P}_p(T)$. Wir setzen
$$\mathcal{C}:=\left\{A \in \mathcal{T} \vert \mu \mapsto \mu(A) \text{ ist messbar}\right\}$$
Betrachte zunächst offene Mengen $U\in \mathcal{T}$. Für eine Folge $\mu_n \rightarrow \mu$ bezüglich $\mathcal{W}_p$ gilt insbesondere $\mu_n\rightarrow \mu$ schwach. Somit gilt mit dem Portmonteau Theorem
$$\liminf_{n\rightarrow\infty}\mu_n(U) \geq \mu(U)$$
Die Abbildung $\mu \mapsto \mu(U)$ ist also halbstetig von unten und somit messbar. Die offenen Mengen sind ein $\cap$-stabiler Erzeuger von $\mathcal{T}$, es reicht also zu zeigen dass $\mathcal{C}$ ein Dynkin-System ist. Für $A\in\mathcal{C}$ ist die Abbildung $\mu \mapsto \mu(A^c)=1-\mu(A)$ wieder messbar, und für $(A_i)_{i\in\mathbb{N}}\subset \mathcal{C}$ disjunkt ist 
$\mu\mapsto \mu(\bigcup_{i\in\mathbb{N}}A_i)=\sum_{i\in\mathbb{N}}\mu(A_i)$ auch messbar. Somit ist $\mathcal{C}=\mathcal{T}$.
\end{proof}
\begin{remark}\label{rem:kernel_char_no_p}
Mit dem gleichen Argument gilt die Behauptung auch für Abbildungen $\mu: S \mapsto \mathcal{P}(T)$, wenn wir auf $\mathcal{P}(T)$ die Borel-Algebra betrachten, die durch die Topologie schwacher konvergenz erzeugt wird.
\end{remark}
\begin{corollary}\label{thm:pmoments}
    Sei $X:(\Omega, \mathcal{F})\rightarrow(S, \mathcal{S})$ eine Zufallsvariable mit Werten in einem polnischen Raum $S$ mit fixierter Metrik $d$. Weiterhin sei $\mathcal{G}\subset \mathcal{F}$ und $X$ habe $p$-te Momente, das heißt $\mathbb{E}(d^p(a_0, X))<\infty$. Dann gelten folgende Aussagen:
    \begin{enumerate}
        \item Wir können $\mathcal{L}(X\vert \mathcal{G})$ als eine messbare Abbildung $\Omega\rightarrow\mathcal{P}_p(S)$ wählen.
        \item $\mathcal{L}(X\vert \mathcal{G})$ hat selbst wieder $p$-te Momente bezüglich der Wasserstein-Metrik.
    \end{enumerate}
\end{corollary}
\begin{proof}
\begin{enumerate}
    \item Sei $\mu:\Omega\rightarrow\mathcal{P}(S)$ eine beliebige Version der bedingten Verteilung. Betrachte die Menge $A:=\{\omega\in\Omega\vert \mu_\omega(d^p(a_0, \cdot)) = \infty\}$. Da 
$$\int_{\Omega}\int_{S} d^p(a_0, x)\mu_{\omega}(dx)\mathbb{P}(d\omega) = \mathbb{E}(d^p(a_0, X))<\infty$$
ist $A$ eine Nullmenge. Setze nun 
$$\tilde{\mu}_\omega=\mathds{1}_{A^c}(\omega)\mu_\omega + \mathds{1}_{A}\delta_{a_0}$$
Für $B\in\mathcal{S}$ ist $\omega\mapsto\tilde{\mu}_\omega(B) = \omega \mapsto \mathds{1}_{A^c}\mu_\omega(B)+\mathds{1}_A \delta_{a_0}$ weiterhin messbar als Kombination messbarer Abbildungen, $\tilde{\mu}$ ist also ein Kern. Da $A$ eine Nullmenge ist, ist $\mathbb{P}_{\vert \mathcal{G}}\otimes \tilde{\mu} = \mathbb{P}_{\vert\mathcal{G}} \otimes \mu$, somit ist $\tilde{\mu}$ eine Version von $\mathcal{L}(X\vert \mathcal{G})$ mit Werten in $\mathcal{P}_p$. Nach dem vorherigen Lemma ist sie messbar bezüglich der Borel-Algebra auf $\mathcal{P}_p$.
\item Wie immer können wir die fixierte Stelle gegenüber der wir $p$-te Momente berechnen frei wählen. Eine einfache Wahl ist $\delta_{a_0}$ für ein $a_0\in S$, da es nur eine Möglichkeit der Kopplung von Dirac-Maßen in der Definition der Wasserstein-Metrik gibt: Das Produktmaß. Es gilt nun
$$\mathbb{E}\left(\mathcal{W}_p^p\left(\mathcal{L}(X\vert\mathcal{G}), \delta_{a_0}\right)\right)=\mathbb{E}\left(\int d^p(a_0, x)\mathcal{L}(X\vert\mathcal{G})(dx)\right)=\mathbb{E}\left(d^p(a_0, X)\right) < \infty$$
\end{enumerate}
\end{proof}
\begin{lemma}\label{thm:kernel_prod}
Seien $\kappa_1: \Omega\rightarrow T,\kappa_2: \Omega\rightarrow S$ Kerne. Dann ist auch $\omega \mapsto \kappa_1(\omega) \otimes \kappa_2(\omega)$ ein Kern.
\end{lemma}
\begin{proof}
Schreibe $\mu:=\kappa_1\otimes\kappa_2$ und setze 
$$\mathcal{C}:=\left\{C\subset\sigma(A\times B, A\in\mathcal{T}, B\in\mathcal{S})\vert \mu(C) \text{ ist messbar}\right\}$$
$\mathcal{C}$ ist ein Dynkinsystem: Für $C\in\mathcal{C}$ ist $\mu(C^c)=1-\mu(C)$ wieder messbar, und für $(C_i)_{i\in\mathbb{N}}\subset \mathcal{C}$ ist 
$$\mu(\bigcup_{i\in\mathbb{N}}(C_i))=\sum_{i\in\mathbb{N}}\mu(C_i)$$
auch wieder messbar. Ein $\cap$-stabiler Erzeuger sind Mengen der Form $A\times B, A\in \mathcal{T}, B\in\mathcal{S}$. Für diese Mengen gilt
$$\mu(A\times B) = \kappa_1(A)\times\kappa_2(B)$$
und das ist wieder messbar als Produkt messbarer Funktionen. Insgesamt ist $\kappa_1\otimes\kappa_2$ wieder ein Kern.
\end{proof}
\begin{lemma}\label{thm:determined_kernel}
Seien $X, Y$ Zufallsvariablen auf $(\Omega, \mathcal{F})$ mit Werten in polnischen Räumen $S$ und $T$. Sei $\mathcal{G}\subset \mathcal{F}$ mit $\sigma(X)\subset \mathcal{G}$. Dann ist 
$$\mathcal{L}\left((X,Y)\vert \mathcal{G}\right) = \delta_X \otimes \mathcal{L}(Y\vert \mathcal{G})$$
\end{lemma}
\begin{proof}
    $\delta_X$ ist ein Kern und messbar bezüglich $\sigma(X)\subset \mathcal{G}$, und auch $\mathcal{L}(Y\vert \mathcal{G})$ ist ein $\mathcal{G}$-messbarer Kern. Somit ist mit dem vorherigen Lemma auch $\delta_X \otimes \mathcal{L}(Y\vert\mathcal{G})$ ein $\mathcal{G}$ messbarer Kern. Ferner gilt für Mengen $A\in \mathcal{G}, B\in\mathcal{S}, C \in \mathcal{S}$:
    \begin{align*}
        \mathbb{P}_{\vert\mathcal{G}} \otimes &\left(\delta_X\otimes \mathcal{L}(Y\vert \mathcal{G})\right)(A\times(B\times X)) \\
        &= \int \int \mathds{1}_A(\omega)\mathds{1}_B(x)\mathds{1}_C(y) (\delta_X\otimes \mathcal{L}(Y\vert\mathcal{G}))(dx, dy)\mathbb{P}_{\vert\mathcal{G}}(d\omega) \\
        &= \int \mathds{1}_A(\omega) \mathds{1}_B(X)\int\mathds{1}_C(y)\mathcal{L}(Y\vert\mathcal{G})(dy)\mathbb{P}_{\vert\mathcal{G}}(d\omega) \\
        &= \mathbb{E}\left(\mathds{1}_A(\omega)\mathds{1}_B(X(\omega))\mathbb{E}(\mathds{1}_C(Y)\vert \mathcal{G})\right) \\
        &= \mathbb{E}(\mathds{1}_A(\omega)\mathds{1}_B(X)\mathds{1}_C(Y)) \\
        &= \mathbb{P}(A, X\in B, Y\in C)
    \end{align*}
    Somit gilt 
    $$\mathbb{P}_{\vert\mathcal{G}} \otimes \left(\delta_X\otimes \mathcal{L}(Y\vert \mathcal{G})\right)(A, (X, Y) \in E) = \mathbb{P}(A, (X, Y)\in E)$$
    für Mengen $E$ der Form $E=A\times B$. Das ist ein $\cap$-stabiler Erzeuger von $\mathcal{S}\otimes\mathcal{T}$, und somit gilt die Gleichheit auch allgemein.
\end{proof}
\begin{lemma}\label{thm:pushforward_measurable}
Seien $S, T$ polnische Räume mit Borel-Algebra. Sei $f:S\rightarrow T$ beschränkt und messbar. Dann ist die \emph{Push-Forward Abbildung} 
$$f_*:\mathcal{P}_p(S)\rightarrow\mathcal{P}_p(T), \, \mu \mapsto f_*(\mu)\quad  \text{ mit } f_*(\mu)(A) = \mu(f^{-1}(A))$$
messbar. Falls $f$ stetig ist, so ist es auch $f_*$.
\end{lemma}
\begin{proof}
Für $\mu \in \mathcal{P}_p(S)$ und $t_0 \in T$ fix gilt, da $f$ beschränkt ist
$$\int d(t_0, x) f_*(\mu)(dx) = \int d(t_0, f(s))\mu(ds) < \infty$$
Die Abbildung ist also wohldefiniert.
Mit Lemma \ref{thm:kernel_characterization} ist $f_*$ messbar genau dann, wenn für jedes $A \in \mathcal{T}$ die Abbildung $\mu \mapsto (f_*(\mu))(A)=\mu(f^{-1}(A))$ messbar ist. Das ist aber klar: Auf $\mathcal{P}_p(S)$ ist die Identität messbar und somit gilt, auch mit Lemma \ref{thm:kernel_characterization}, dass $\mu \mapsto \mu(B)$ messbar ist für alle $B\in \mathcal{S}$, also insbesondere für $f^{-1}(A) \in \mathcal{S}$. 

Sei nun $f$ stetig und $(\mu_n)_{n\in\mathbb{N}}, \mu \in \mathcal{P}_p(S)$ mit $\mu_n\rightarrow \mu$ bezüglich $\mathcal{W}_p$. Sei $g:T\rightarrow \mathbb{R}$ stetig mit $p$-Wachstum, das heißt $|g(x)| \leq c(d(x, s_0)+1)$ für fixierte $c\in\mathbb{R}, s_0\in S$. Dann gilt
    $$(f_*(\mu_n))(g) = \int g(f(s)) \mu(ds)$$
Für die Funktion $g(f(s))$ gilt $|g(f(s))|\leq c(d(f(s), s_0)+1)$ und das ist beschränkt, da $f$ beschränkt ist. Somit ist $g\circ f$ beschränkt und $\mu_n(g\circ f) \rightarrow \mu(g\circ f) = (f_*(\mu))(g)$. Insgesamt ist $f_*$ stetig.
\end{proof}
\begin{remark}
Mit Bemerkung \ref{rem:kernel_char_no_p} ist für ein (möglicherweise nicht beschränkts) messbares $f: S\rightarrow T$ mit dem gleichen Argument $f_*$ auch als Abbildung $f_*: \mathcal{P}(S) \rightarrow \mathcal{P}(T)$ messbar.
\end{remark}
\begin{corollary}\label{thm:pushforward_law}
Seien $S, T$ polnische Räume, $f:S\rightarrow T$ messbar. Weiterhin sei $X:(\Omega, \mathcal{F}) \rightarrow (S, \mathcal{S})$ und $\mathcal{G}\subset\mathcal{F}$. Dann ist 
$$\mathcal{L}(f(X)\vert \mathcal{G}) = f_*\mathcal{L}(X\vert\mathcal{G})$$
\end{corollary}
\begin{proof}
Mit dem vorherigen Lemma und der anschließenden Bemerkung ist $f_*\mathcal{L}(X\vert\mathcal{G})$ eine $\mathcal{G}$-messbare Abbildung. Weiter gilt für $A\in\mathcal{G}, B\in\mathcal{T}$
\begin{align*}
    (\mathbb{P}\otimes f_*\mathcal{L}(X\vert \mathcal{G}))(A\times B) &= \int \mathds{1}_A(\omega) \int \mathds{1}_B(t) f_*(\mathcal{L}(X\vert\mathcal{G}))(dt) \mathbb{P}(d\omega) \\
    &= \int \mathds{1}_A(\omega) \int \mathds{1}_{f^{-1}(B)}(s)\mathcal{L}(X\vert\mathcal{G})(ds) \mathbb{P}(d\omega) \\
    &= \mathbb{P}(A, X\in f^{-1}(B)) \\
    &= \mathbb{P}(A, f(X) \in B)
\end{align*}
\end{proof}

\begin{lemma}\label{thm:pushforward_expectancy}
Sei $(\Omega, \mathcal{F}, \mathbb{P})$ ein Wahrscheinlichkeitsraum, $(S, \mathcal{S})$ ein polnischer Raum. Weiterhin seien $X: \Omega \rightarrow S$ messbar und $U: S\rightarrow \mathbb{R}$ beschränkt und messbar. Betrachte zwei $\sigma$-Algebren $\mathcal{G}\subset \mathcal{F}$ und $\mathcal{A} \subset \mathcal{S}$, sodass $\mathcal{G}$ die kleinste $\sigma$-Algebra bezüglich der $X: (\Omega, \mathcal{G}) \rightarrow (S, \mathcal{A})$ messbar ist. Dann gilt
$$\mathbb{E}_{\mathbb{P}}(U(X) \vert \mathcal{G}) = \mathbb{E}_{\mathbb{P}^X}(U \vert \mathcal{A})(X) \quad \text{ fast sicher}$$
\end{lemma}
\begin{proof}
Schreibe $\mu := \mathbb{P}^X$. Da $X: (\Omega, \mathcal{G}) \rightarrow (S, \mathcal{A})$ messbar ist, ist auch $\mathbb{E}_\mu(U\vert \mathcal{A})(X)$ $\mathcal{G}$-messbar. Für eine $\mathcal{G}$-messbares, beschränktes $V:\Omega \rightarrow \mathbb{R}$ faktorisiert $V$ als $V(\omega) = \tilde{V}(X(\omega))$ für ein $\mathcal{A}$-messbares $\tilde{V}$. Somit gilt
\begin{align*}
\int V(\omega)U(X(\omega))\mathbb{P}(d\omega) &= \int \tilde{V}(X)U(X) \mu(dX)  \\
&= \int \tilde{V}(X)\mathbb{E}_\mu(U \vert \mathcal{A})(X) \mu(dX) \\
&= \int \tilde{V}(X(\omega)) \mathbb{E}_\mu(U \vert \mathcal{A})(X(\omega)) \mathbb{P}(d\omega) \\
&= \int V(\omega) \mathbb{E}_\mu(U \vert \mathcal{A})(X(\omega)) \mathbb{P}(d\omega) \\
\end{align*}
und damit ist $\mathbb{E}_{\mathbb{P}^{X}}(U \vert \mathcal{A})(X)$ eine Version der bedingten Erwartung $\mathbb{E}_{\mathbb{P}}(U(X) \vert \mathcal{G})$.
\end{proof}

\subsection{Kopplungen}
\begin{definition}[Bedingte Unabhängigkeit]
    Für einen Wahrscheinlichkeitsraum $(\Omega, \mathcal{F}, \mathbb{P})$, eine Unter $\sigma$-Algebra $\mathcal{G} \subset \mathcal{F}$ und Mengen $A, B \in \mathcal{F}$ schreiben wir $\mathbb{P}(A \vert \mathcal{G}) := \mathbb{E}(\mathds{1}_A\vert \mathcal{G})$ und nennen $A$ und $B$ \emph{unabhängig bedingt auf $\mathcal{G}$}, falls
    \begin{equation}\label{eq:def_cond_ind}
    \mathbb{P}(A\cap B\vert \mathcal{G}) = \mathbb{P}(A\vert \mathcal{G}) \mathbb{P}(B \vert \mathcal{G}) \quad \text{ fast sicher }
    \end{equation}
    Wir nennen Mengensysteme $(\mathcal{E}_i)_{i\in I}$ unabhängig bedingt auf $\mathcal{G}$, falls jede endliche Auswahl von Mengen $A_t \in \mathcal{E}_{i_t}, t=1,...,n$ die Gleichung
    $$\mathbb{P}\left(\bigcap_{t=1}^n A_t\vert \mathcal{G}\right) = \prod_{t=1}^n \mathbb{P}(A_t \vert \mathcal{G}) \quad \text{ fast sicher }$$
    erfüllt.
\end{definition}
\begin{definition}[Filtrierter Prozess]
Betrachte ein Fünf-Tupel
$$\mathbb{X}=\left(\Omega, \mathcal{F}, \mathbb{P}, \left(\mathcal{F}_t\right)_{t=1}^N, \left(X_t\right)_{t=1}^N\right)$$
wobei $(\Omega, \mathcal{F}, \mathbb{P})$ ein Wahrscheinlichkeitsraum, $\left(\mathcal{F}_t\right)_{t=1}^N\subset \mathcal{F}$ eine Filtration und \\
$(X_t)_{t=1}^N$ ein stocahstischer Prozess mit Werten in $\mathcal{X}_{1:N}$ adaptiert an $(\mathcal{F})_{t=1}^N$. Ein solches Tupel $\mathbb{X}$ nennen wir einen \emph{filtrierten (stochastischen) Prozess}. Wir schreiben $\mathcal{FP}$ für den Raum aller stochastischen Prozess und $\mathcal{FP}_p$ für den Raum aller stocahstischen Prozess mit $p$-ten Momenten, das heißt $\mathbb{E}(d_{\mathcal{X}_t}(a_0, X_t))<\infty, a_0\in\mathcal{X}_t, t=1,...,N$.
\end{definition}
\begin{definition}[Kausale Kopplungen]
Seien $\mathbb{X}, \mathbb{Y}$ filtrierte Prozesse. Wir nennen $\pi$ eine \emph{Kopplung} von $\mathbb{X}$ und $\mathbb{Y}$, falls die Marginalien $\mathbb{P}^\mathbb{X}$ und $\mathbb{P}^\mathbb{Y}$ sind (das heißt $\pi \in \cpl(\mathbb{P}^\mathbb{X}, \mathbb{P}^\mathbb{Y})$). Weiterhin nennen wir $\pi$
\begin{enumerate}
\item \emph{kausal} (oder kausal von $\mathbb{X}$ zu $\mathbb{Y}$) falls für jedes $1\leq t\leq N$, bedingt auf $\mathcal{F}_{t, 0}^{\mathbb{X}, \mathbb{Y}}$, die Mengensysteme $\mathcal{F}_{N, 0}^{\mathbb{X}, \mathbb{Y}}$ und $\mathcal{F}_{0, t}^{\mathbb{X},\mathbb{Y}}$ unabhängig sind.
\item \emph{antikausal} (oder kausal von $\mathbb{Y}$ zu $\mathbb{X}$), falls für jedes $1 \leq t \leq N$, bedingt auf $\mathcal{F}_{0, t}^{\mathbb{X}, \mathbb{Y}}$, die Mengensysteme $\mathcal{F}_{0, N}^{\mathbb{X}, \mathbb{Y}}$ und $\mathcal{F}_{t, 0}^{\mathbb{X}, \mathbb{Y}}$ unabhängig sind.
\item \emph{bikausal}, falls es kausal und antikausal ist.
\end{enumerate}
Wir schreiben $\cpl(\mathbb{X}, \mathbb{Y})$, $\cplc(\mathbb{X}, \mathbb{Y})$, $\cplbc(\mathbb{X}, \mathbb{Y})$ für die Mengen der Kopplungen, kausalen Kopplungen und bikausalen Kopplungen von $\mathbb{X}$ und $\mathbb{Y}$.
\end{definition}
% TODO: Beweis von Kallenberg übertragen oder zitieren
\begin{lemma}\label{thm:causality_characterization}
Sei $\pi$ eine Kopplung von filtrierten Prozessen $\mathbb{X}$ und $\mathbb{Y}$. Dann sind äquivalent:
\begin{enumerate}
\item $\pi$ ist kausal (von $\mathbb{X}$ zu $\mathbb{Y}$).
\item $\mathbb{E}_\pi(U \vert \mathcal{F}_{t, t}^{\mathbb{X}, \mathbb{Y}}) = \mathbb{E}_\pi(U \vert \mathcal{F}_{t, 0}^{\mathbb{X}, \mathbb{Y}})$ für alle $1\leq t\leq N$ und beschränkte, $\mathcal{F}_N^\mathbb{X}$-messbare $U$.
\item $\mathbb{E}_\pi(V\vert \mathcal{F}_{N, 0}^{\mathbb{X}, \mathbb{Y}}) = \mathbb{E}_\pi(V \vert \mathcal{F}_{t, 0}^{\mathbb{X}, \mathbb{Y}})$ für alle $1\leq t\leq N$ und beschränkte, $\mathcal{F}_t^\mathbb{Y}$-messbare $V$.
\end{enumerate}
\end{lemma}

\section{Der Wasserstein-Raum stochastischer Prozesse}
\begin{definition}[Kanonischer Raum]
    Sei $p \in [1, \infty)$. Wir definieren iterativ eine Folge von geschachtelten Räumen. Schreibe $(\mathcal{Z}_N, d_{\mathcal{Z}_N}) := (\mathcal{X}_N, d_{\mathcal{X}_N})$ und für $t=N-1, ..., 1$ setze
    \begin{equation}
        \mathcal{Z}_t := \mathcal{Z}_t^- \times \mathcal{Z}_t^+ := \mathcal{X}_t \times \mathcal{P}_p(\mathcal{Z}_{t+1})
    \end{equation}
    mit der Metrik $d^p_{\mathcal{Z}_t} := d^p_{\mathcal{X}_t} + \mathcal{W}^p_{p, \mathcal{Z}_{t+1}}$. Elemente von $\mathcal{Z}_t$ schreiben wir als $z_t=(z_t^-, z_t^+) \in \mathcal{Z}_t^- \times \mathcal{Z}_t^+$. Der \emph{kanonische filtrierte Raum} ist nun
    \begin{equation}
        (\mathcal{Z}, \mathcal{F}^\mathcal{Z}, (\mathcal{F}_t^\mathcal{Z})_{t=1}^N)
    \end{equation}
    wobei $\mathcal{Z} = \mathcal{Z}_{1:N}$, $\mathcal{F}^\mathcal{Z}$ die Borelsche $\sigma$-Algebra auf $\mathcal{Z}$ und $\mathcal{F}^\mathcal{Z}_t=\sigma(z \mapsto z_{1:t})$ die kleinste $\sigma$-Algebra, die die Projketionen bis zur $t$-ten Koordinate messbar macht.
\end{definition}
Die auf diese Weise konstruierten Räume sind polnisch, da alle verwendeten Operationen abgeschlossen unter dieser Eigenschaft sind.
\begin{definition}[Der Informationsprozess]
Für $\mathbb{X}\in \mathcal{FP}_p$ definieren wir rekursiv den assoziierten \emph{Informationsprozess} wie folgt: Für $t=N$ sei $\ip_N(\mathbb{X}):=X_N$ und für $t=N-1,...,1$ sei
\begin{align*}
   \ip_t(\mathbb{X})&:=(\ip_t^-(\mathbb{X}), \ip_t^+(\mathbb{X})) \\
   &:= (X_t, \mathcal{L}(\ip_{t+1}(\mathbb{X}) \vert \mathcal{F}_t^{\mathbb{X}})) \in \mathcal{Z}_t
\end{align*}
$\ip_t(\mathbb{X})$ und $\ip_{1:t}(\mathbb{X})$ sind messbar bezüglich $\mathcal{F}_t^\mathbb{X}$, der Informationsprozess ist also adaptiert an $\left(\mathcal{F}_t^\mathbb{X}\right)_{t=1}^N$.
\end{definition}
\begin{remark}
Mit unserer Vorarbeit im letzten Kapitel ist diese Konstruktion wohldefiniert: Mit der zweiten Aussage aus Korollar \ref{thm:pmoments} hat induktiv jedes Paar $(X_t, \mathcal{L}(\ip_{t+1}(\mathbb{X})\vert \mathcal{F}_t^\mathbb{X}))$ wieder $p$-te Momente, während die erste Version liefert, dass wir die bedingte Verteilung als messbare Abbildung mit Werten in $\mathcal{P}_p$ wählen können.
\end{remark}
Der eben konstruierte kanonische filtrierte Raum und der Informationsprozess entstehen durch eine iterative Schachtelung von Räumen und Räumen von Verteilungen darauf. Für die Analyse dieser Prozesse ist es nützlich einen Operator einzuführen, der diese Schachtelung wieder ''entfaltet'':
\begin{definition}[Unfold-Operator]
Für $1\leq t\leq N$ ist der Unfold-Operator eine Abbildung $\uf_t:\mathcal{P}_p(\mathcal{Z}_t) \rightarrow \mathcal{P}_p(\mathcal{Z}_{t:N})$. Wir definieren ihn durch $\uf_N=\operatorname{id}$ und für $1\leq t\leq N-1$ induktiv durch
$$\uf_{t}(\mu)(dz_{t:N})=\mu(dz_t)\uf_{t+1}(z_t^+)(dz_{t+1:N})$$
Für eine messbare Abbildung $f:\mathcal{Z}_{t:N}\rightarrow \mathbb{R}$ und ein Maß $\mu\in\mathcal{P}_p(\mathcal{Z}_t)$ bedeutet das
$$\int f(z_{t:N}) \uf_t(\mu)(dz_{t:N}) = \int\int ...\int f(z_t,...,z_N)z^+_{N-1}(dz_N)...z_{t}^+(dz_{t+1})\mu(dz_t)$$
\end{definition}
\begin{lemma}\label{thm:bounded_unfold}
    Es gibt Punkte $a_n \in \mathcal{Z}_n, 1\leq t\leq N$ und Zahlen $n_t \in \mathbb{N}$, sodass 
    $$\int d^p(a_{t+1:n}, z_{t+1:N}) \uf_{t+1}(z_t^+)(dz_{t+1:N})\leq n_td^p(a_t, z_t)$$
    Weiterhin gibt es für eine Verteilung $\mu \in \mathcal{P}_p(\mathcal{Z}_t)$ eine Zahl $n\in \mathbb{N}$ mit
    $$\int d^p(z_{t:N}, a_{t:N})\uf_t(\mu)(dz_{t:N})\leq n\int d^p(z_t, a_t)\mu(dz_t)< \infty$$ 
    Es hat also $\uf_t(\mu)$ $p$-te Momente und der Unfold-Operator ist somit wohldefiniert.
\end{lemma}
\begin{proof}
Wähle die Punkte als $a_N \in \mathcal{X}_N$ und für $1\leq t \leq N-1$ $a_0^-\in\mathcal{X}_t$ und $a_0^+ = \delta_{a_{t+1}}$. 
Wir zeigen die Behauptung induktiv. Für $t=N-1$ gilt sie, da
$$\int d^p(a_N, z_N)z_{N-1}^+(dz_N)=\mathcal{W}_p^p(z_{N-1}^+, \delta_{a_N}) \leq d^p(z_{N-1}, a_{N-1})$$
Beachte hierbei wieder, dass es in der Definition der Wassersteinmetrik bezüglich einem Dirac-Maß nur eine mögliche Kopplung gibt. Nehmen wir nun an, die Aussage gilt für $t+1$. Dann folgt die Aussage für $t$:
\begin{align*}
\int &d^p(a_{t+1:N}, z_{t+1}:N) \uf_{t+1}(z_t^+)(dz_{t+1:N}) \\
&= \int\int d^p(a_{t+2:N}, z_{t+2:N}) + d^p(a_{t+1}, z_{t+1}) \uf_{t+2}(z_{t+1}^+)(dz_{t+2:N})z_t^+(dz_{t+1}) \\
&\leq \int n_{t+1}d^p(a_{t+1}, z_{t+1}) + d^p(a_{t+1}, z_{t+1}) z_t^+(dz_{t+1}) \\
&= \int (n_{t+1}+1)d^p(a_{t+1}, z_{t+1}) z_t^-(dz_{t+1}) \\
&= (n_{t+1}+1)\mathcal{W}_p^p(\delta_{a_{t+1}}, z_t^-) \\
&\leq (n_{t+1}+1)d_p^p(a_t, z_t)
\end{align*}
Mit einem ähnlichen Argument gilt nun auch die zweite Behauptung:
\begin{align*}
\int &d^p(z_{t:N}, a_{t:N})\uf_{t}(\mu)(dz_{t:N}) \\
&= \int\int d^p(z_{t+1:N}, a_{t+1:N})+d^p(z_t, a_t)\uf_{t+1}(z_t^+)(dz_{t+1:N})\mu(dz_t) \\
&\leq \int (n_{t+1}+1)d^p(z_t, a_t)\mu(dz_t) < \infty
\end{align*}
wobei letzte Geichung gilt, da $\mu \in\mathcal{P}_p(\mathcal{Z}_t)$.
\end{proof}
\begin{lemma}\label{thm:properties_unfold}
Für $1\leq t\leq N-1$ gelten die folgenden Eigenschaften:
\begin{enumerate}
\item $\uf_t$ ist stetig.
\item Für $\mathbb{X}\in\mathcal{FP}_p$ gilt
\begin{align}
    \mathcal{L}(\ip(\mathbb{X})\vert \mathcal{F}_t^\mathbb{X})=\delta_{\ip_{1:t}(\mathbb{X})}\otimes \uf_{t+1}(\ip_t^+(\mathbb{X}))
\end{align}
und 
$$\mathcal{L}(\ip(\mathbb{X}))=\uf_1(\mathcal{L}(\ip_1(\mathbb{X})))$$
Für $f:\mathcal{Z}\rightarrow \mathbb{R}$ beschränkt und Borel messbar gilt
\begin{align}\label{eq:uf_expected_value}
    \mathbb{E}(f(\ip(\mathbb{X}))\vert \mathcal{F}_t^\mathbb{X}) = \int f(ip_{1:t}(\mathbb{X}), z_{t+1:N}) \uf_{t+1}(ip_{t}^+(\mathbb{X}))(dz_{t+1:N})
\end{align}
\end{enumerate}
\end{lemma}
\begin{proof}
\begin{enumerate}
    \item Wir zeigen die Stetigkeit induktiv. Für $t=N$ ist $\uf_N=\operatorname{id}$ und somit stetig. Nehmen wir nun an, $\uf_{t+1}$ ist stetig. Seien $\mu_n, \mu\in\mathcal{P}_p(\mathcal{Z}_t)$ mit $\mu_n\rightarrow \mu$ und sei $f:\mathcal{Z}_{t:N}\rightarrow\mathbb{R}$ beschränkt und gleichmäßig stetig. Betrachte die Funktion 
    $$g(z_t)=\int f(z_t, z_{t+1:N}) \uf_{t+1}(z_t^+)(dz_{t+1:N})$$
    Die Projektion auf $z_t^+$ ist kontrahierend, somit gilt für $z_t^n\rightarrow z_t$ auch $z_t^{n,+}\rightarrow z_t^+$, und da $\uf_{t+1}$ stetig ist folgt $\nu_n:=\uf_{t+1}(z_t^{n,+})\rightarrow \uf_{t+1}(z_t^+)=:\nu$. 
    \begin{align}\label{eq:g_continuous}
        \begin{split}
        |g(z_t^n)-g(z_t)| \leq &\left|\int f(z_t^n, z_{t+1:N})-f(z_t,z_{t+1:N})\nu_n(dz_{t+1:N})\right| \\
        &+ \left|\int f(z_{t:N})\nu_n(dz_{t+1:N}) - \int f(z_{t:N})\nu(dz_{t+1:N}))\right|
        \end{split}
    \end{align}
    Der erste Term auf der rechten Seite ist beschränkt durch 
    $$\max\limits_{z_{t+1:N}}\left|f(z_t^n, z_{t+1:N})-f(z_t, z_{t+1:N})\right|\rightarrow 0$$
    da $f$ gleichmäßig stetig ist. Der zweite Summand auf der rechten Seite konvergiert gegen $0$, da $\nu_n \rightarrow \nu$ in $\mathcal{W}_p$, also insbesondere schwach. $g$ ist also stetig und auch beschränkt, da $f$ beschränkt ist, und somit
    $$\uf_t(\mu_n)(f)=\int g(z_t)\mu_n(dz_t)\rightarrow \int g(z_t)\mu(dz_t)=\uf_t(\mu)(f)$$
    Wir haben also $\uf_t(\mu_n)\rightarrow\uf_t(\mu)$ schwach und es bleibt zu zeigen, dass die $p$-ten Momente konvergieren. Wähle dazu die Punkte $a_t$ wie in Lemma \ref{thm:bounded_unfold}. Dann gilt
    \begin{align*}
        \uf_{t}&(\mu_n)(d^p(a_{t:N}, \cdot)) \\
        &= \int\int d^p(a_t, z_t) + d^p(a_{t+1:N}, z_{t+1:N})\uf_{t+1}(z_t^+)(dz_{t+1:N})\mu_n(dz_t) \\
        &=\int d^p(a_t, z_t)\mu_n(dz_t) + \int\int d^p(a_{t+1:N}, z_{t+1:N})\uf_{t+1}(z_t^+)(dz_{t+1:N})\mu_n(dz_t)
    \end{align*}
    Es gilt $\mu_n(d^p(a_t, \cdot))\rightarrow \mu(d^p(a_t, \cdot))$ da $\mu_n\rightarrow\mu$ in $\mathcal{W}_p$.
    Für den zweiten Term betrachten wir wieder das innere Integral
    $$g(z_t) = \int d^p(a_{t+1:N}, z_{t+1:N})\uf_{t+1}(z_t^+)(dz_{t+1:N})$$
    Der Integrand ist Lipschitz stetig, also auch gleichmäßig stetig, und hat $p$-Wachstum. Mit der gleichen Aufteilung der Terme wie in Gleichung \ref{eq:g_continuous} können wir also wieder folgern, dass $g$ stetig ist. Mit Lemma \ref{thm:bounded_unfold} ist $g(z_t)\leq cd^p(a_t, z_t)$, $g$ hat also $p$-Wachstum und somit $\uf_{t}(\mu_n)(g)\rightarrow \uf_{t}(\mu)(g)$. Insgesamt gilt $\uf_{t}(\mu_n)\rightarrow\uf_t(\mu)$ bezüglich $\mathcal{W}_p$.
    \item $\ip_{1:t}$ ist messbar bezüglich $\mathcal{F}_t^X$ und $\ip_t^+(\mathbb{X})=\mathcal{L}(\ip_{t+1}(\mathbb{X})\vert \mathcal{F}_t^\mathbb{X})$ genauso. Da $\uf_t$ mit der ersten Aussage des Beweises stetig und somit auch messbar ist, ist $\uf_{t+1}(\ip_t^+(\mathbb{X}))$ ein $\mathcal{F}_t^\mathbb{X}$ messbarer Kern. Mit Lemma \ref{thm:kernel_prod} ist somit auch $\delta_{\ip_{1:t}(\mathbb{X})}\otimes \uf_{t+1}(\ip_t^+(\mathbb{X}))$ ein $\mathcal{F}_t^\mathbb{X}$ messbarer Kern. Mit Lemma \ref{thm:law_expectancy_connection} reicht es also für jedes $t$ Gleichung \ref{eq:uf_expected_value} nachzurechnen. Wir zeigen die Behauptung nun durch Induktion über $t$. Für $t=N-1$ ist mit Lemma \ref{thm:determined_kernel} 
    $$\mathcal{L}(\ip(\mathbb{X})\vert \mathcal{F}_{N-1}^\mathbb{X})=\delta_{\ip_{1:N-1}(\mathbb{X})} \otimes \mathcal{L}(\ip_N(\mathbb{X})\vert \mathcal{F}_{N-1}^\mathbb{X})$$
    was genau $\delta_{\ip_{1:N-1}(\mathbb{X})}\otimes \uf_{N}(\ip_{N-1}^+(\mathbb{X}))$ entspricht. Nehmen wir nun an, die Gleichung gilt für ein $2\leq t\leq N-1$. Für $f: \mathcal{Z}\rightarrow\mathbb{R}$ beschränkt und messbar, betrachte
    $$g(z_{1:t})=\int f(z_{1:t}, z_{t+1:N})\uf_{t+1}(z_t^+)(dz_{t+1:N})$$
    Nach Induktionsvoraussetzung gilt $\mathbb{E}(f(\ip(\mathbb{X}))\vert\mathcal{F}_t^\mathbb{X})=g(\ip_{1:t}(\mathbb{X}))$.
    Dann ist 
    \begin{align*}
    \mathbb{E}(f(\ip(\mathbb{X}))\vert\mathcal{F}_{t-1}^\mathbb{X}) &= \mathbb{E}\left( \mathbb{E}(f(\ip(\mathbb{X}))\vert \mathcal{F}_{t}^\mathbb{X}) \vert \mathcal{F}_{t-1}^\mathbb{X}\right) \\
    &= \mathbb{E}(g(\ip_{1:t})\vert \mathcal{F}_{t-1}^\mathbb{X})
    \end{align*}
    Da wieder mit Lemma \ref{thm:determined_kernel} 
    $$\mathcal{L}(\ip_{1:t}\vert \mathcal{F}_{t-1}^\mathbb{X})=\delta_{\ip_{1:t-1}(\mathbb{X})}\otimes \mathcal{L}(\ip_t\vert \mathcal{F}_{t-1}^\mathbb{X})=\delta_{\ip_{1:t-1}(\mathbb{X})}\otimes \ip_{t-1}^+(\mathbb{X})$$
    folgt also
    \begin{align*}
    \mathbb{E}(f(\ip(\mathbb{X}))\vert \mathcal{F}_{t-1}^\mathbb{X}) &= \int g(\ip_{1:t-1}(\mathbb{X}), z_t) \ip_{t-1}^+(\mathbb{X})(dz_t) \\
    &= \int \int f(\ip_{1:t-1}, z_{t:N}) \uf_{t+1}(z_t^+)(dz_{t+1:N})\ip_{t-1}^+(\mathbb{X})(dz_t) \\
    &= \int f(\ip_{1:t-1}, z_{t:N}) \uf_{t}(\ip_{t-1}^+(\mathbb{X}))(dz_{t:N})
    \end{align*}
    und damit die Behauptung erste Behauptung.
    Der letzte Teil der Behauptung
    $$\mathcal{L}(\ip(\mathbb{X}))=\uf_1(\mathcal{L}(\ip_1(\mathbb{X})))$$
    folgt mit dem gleichen Argument wie oben, wenn wir die $\sigma$-Algebren fortsetzen durch die triviale $\sigma$-Algebra $\mathcal{F}_0^\mathbb{X}=\{\emptyset, \Omega\}$ und den Informationsprozess durch $\ip_0^+(\mathbb{X})=\mathcal{L}(\ip_1(\mathbb{X})\vert \mathcal{F}_0^\mathbb{X}) \equiv \mathcal{L}(\ip_1(\mathbb{X}))$.
\end{enumerate}
\end{proof}
Mit dieser Darstellung der bedingten Verteilung vom Informationsprozess können wir auch bedingte Verteilungen und Erwartungen von Funktionen in $\ip(\mathbb{X})$ konstruieren:
% TODO: Diskussion vom Fall P statt P_p
\begin{lemma}[Der Informationsprozess ist ''self-aware'']\label{thm:self_awareness}
Für jede beschränkte, Borel - messbare (stetige) Funktion $f:\mathcal{Z}\rightarrow \mathbb{R}$ und $1\leq t\leq N$ gibt es eine beschränkte, Borel-messbare (stetige) Funktion $g:\mathcal{Z}_{1:t}\rightarrow \mathbb{R}$ mit 
$$\mathbb{E}(f(\ip(\mathbb{X}))\vert \mathcal{F}_t^\mathbb{X}) = g(\ip_{1:t}(\mathbb{X}))) \quad \text{ für alle }\mathbb{X}\in\mathcal{FP}_p$$
Allgemeiner: Für einen polnischen Raum $\mathcal{A}$ und eine Borel messbare (stetige) Funktion $f:\mathcal{Z}\rightarrow \mathcal{A}$ und $1\leq t\leq N$ gibt es eine messbare (stetige) Funktion $g:\mathcal{Z}_{1:t}\rightarrow \mathcal{P}(\mathcal{A})$ mit estf
$$\mathcal{L}(f(\ip(\mathbb{X}))\vert \mathcal{F}_t^\mathbb{X})=g(\ip_{1:t}(\mathbb{X}))\quad\text{ für alle } \mathbb{X}\in\mathcal{FP}_p$$
\end{lemma}
\begin{proof}
Sei $f:\mathcal{Z}\rightarrow \mathcal{A}$ beschränkt und messbar (bzw. stetig). Mit dem gleichen Argument wie im ersten Teil vom Beweis von \ref{thm:properties_unfold} ist die Abbildung 
$$F:\mathcal{Z}_{1:t}\rightarrow\mathcal{P}_p(\mathcal{Z}), \quad z_{1:t}\mapsto \delta_{z_{1:t}} \otimes \uf_{t+1}(z_t^+)$$ 
stetig: Der Kern $\uf_{t+1}(z_t^+)$ ist stetig nach Lemma \ref{thm:properties_unfold} (da auch die Projektion auf $z_t^+$ kontrahierend, also stetig ist). Wenn wir nun eine gleichmäßig stetige beschränkte Testfunktion $g:\mathcal{Z}\rightarrow\mathbb{R}$ nehmen konvergiert für eine Folge $z_{1:t}^n\rightarrow z_{1:t}\in\mathcal{Z}_{1:t}$ auch
$$\left(\delta_{z_{1:t}^n}\otimes \uf_{t+1}(z_t^{n, +})\right)(g) = \int g(z_{1:t}^n, z_{t+1:N})\uf_{t+1}(z_t^{n, +})(dz_{t+1:N})$$
gegen 
$$\int g(z_{1:t}, z_{t+1:N})\uf_{t+1}(z_t^+)(dz_{t+1:N})=(\delta_{z_{1:t}} \otimes \uf_{t+1}(z_t^+))(g)$$
(das sind genau die Terme von \ref{thm:properties_unfold}) und auch die $p$-ten Momente konvergieren, wieder mit der gleichen Argumentation wie in \ref{thm:properties_unfold}. $F$ ist also in der Tat stetig. Betrachte nun die Funktion $G(z_{1:t}) = f_*F(z_{1:t})$. Mit Lemma \ref{thm:pushforward_measurable} ist $G$ messbar (und stetig, falls $f$ stetig ist). Mit Korollar \ref{thm:pushforward_law} und dem zweiten Teil von Lemma \ref{thm:properties_unfold} gilt nun die Gleichungskette
$$\mathcal{L}(f(\ip(\mathbb{X}))\vert \mathcal{F}_t^\mathbb{X}) = f_*\mathcal{L}(\ip(\mathbb{X}) \vert \mathcal{F}_t^\mathbb{X}) = f_*F(\ip_{1:t}(\mathbb{X}))=G(\ip_{1:t}(\mathbb{X}))$$

Für die erste Gleichung betrachten wir nun den Spezialfall $\mathcal{A}=\mathbb{R}$. Konstruiere $G$ genau wie zuvor. Nun gilt mit Lemma \ref{thm:law_expectancy_connection}
$$\mathbb{E}(f(\ip(\mathbb{X}))\vert \mathcal{F}_t^\mathbb{X}) = \int a \mathcal{L}(f(\ip(\mathbb{X}))\vert \mathcal{F}_t^\mathbb{X})(da) = \int a G(\ip_{1:t}(\mathbb{X})(da)$$
Die Abbildung $\mathcal{P}_p(\mathbb{R}) \rightarrow \mathbb{R}, \mu \mapsto \int a \mu(da)$ ist stetig, da $\operatorname{id}$ stetig mit $p$-Wachstum (beachte $p\geq 1$). Somit ist $g(z_{1:t}) = \int a G(z_{1:t})(da)$ messbar und, falls $f$ (und somit auch $G$) stetig ist, auch stetig. Insgesamt folgt die Behauptung.
\end{proof}

%TODO: Insbesondere hier sollte geprüft werden, inwiefern der Wechsel zwischen FP und FP_p vonstatten geht (kann man jedes CFP_p durch mu in P_p(Z_1) darstellen?)
\begin{definition}[Kanonischer filtrierter Prozess]
    Wir nennen $\mathbb{X} \in \mathcal{FP}_p$ einen \emph{kanonischen filtrierten Prozess}, kurz $\mathbb{X} \in \CFP_p$ falls $\mathbb{X}$ der Form
    \begin{equation}\label{eq:def_canonical_filtered_process}
    \mathbb{X} = \left( \mathcal{Z}, \mathcal{F}^\mathcal{Z}, \uf_1(\bar{\mu}), \left(\mathcal{F}_t^\mathcal{Z}\right)_{t=1}^N, Z^-\right)
    \end{equation}
    ist, wobei $\bar{\mu} \in \mathcal{P}_p(\mathcal{Z}_1)$.
\end{definition}
\begin{definition}[Assoziierter kanonischer filtrierter Prozess]
    Für $\mathbb{X} \in \mathcal{FP}_p$ sei $\overline{\mathbb{X}} \in \CFP_p$ gegeben durch Gleichung \ref{eq:def_canonical_filtered_process} bezüglich $\bar\mu := \mathcal{L}(\ip_1(\mathbb{X}))$, mit Lemma \ref{thm:properties_unfold} also
    \begin{equation}\label{eq:associated_process}
        \overline{\mathbb{X}} = \left( \mathcal{Z}, \mathcal{F}^\mathcal{Z}, \mathcal{L}(\ip(\mathbb{X})), \left(\mathcal{F}_t^\mathcal{Z}\right)_{t=1}^N, Z^-\right)
    \end{equation}
    Wir nennen $\overline{\mathbb{X}}$ den \emph{zu $\mathbb{X}$ assoziierten kanonischen filtrierten Prozess}.
\end{definition}

\begin{lemma}
Seien $\mathbb{X}, \mathbb{Y} \in \mathcal{FP}_p$ und $\overline{\mathbb{X}}, \overline{\mathbb{Y}}$ die assoziierten kanonischen Prozesse. Dann gelten die folgenden Aussagen:
\begin{enumerate}
\item $(\id, \ip(\mathbb{X}))_*\mathbb{P}^\mathbb{X} \in \cplbc(\mathbb{X}, \overline{\mathbb{X}})$.
\item Für $\pi \in \cplc(\mathbb{X}, \mathbb{Y})$ ist $(\ip(\mathbb{X}), \ip(\mathbb{Y}))_* \pi \in \cplc(\overline{\mathbb{X}}, \overline{\mathbb{Y}})$.
\item Für $\pi \in \cplbc(\mathbb{X}, \mathbb{Y})$ ist $(\ip(\mathbb{X}), \ip(\mathbb{Y}))_* \pi \in \cplbc(\overline{\mathbb{X}}, \overline{\mathbb{Y}})$.
\end{enumerate}
\end{lemma}
\begin{proof}
\begin{enumerate}
\item Schreibe $\gamma:=(\id, \ip(\mathbb{X}))_* \mathbb{P}^\mathbb{X}$. Wir prüfen zunächst Kausalität über den dritten Punkt der Charakterisierung von Kausalität in Lemma \ref{thm:causality_characterization}. Sei dazu $V: \mathcal{Z}\rightarrow\mathbb{R}$ beschränkt und $\mathcal{F}_t^\mathbb{Z}$ messbar. Wir bemerken, dass bezüglich $\gamma$ auf $\Omega^\mathbb{X} \times \mathcal{Z}$ gilt $z=\ip(\mathbb{X})$ fast sicher (denn $\gamma\left(z \neq \ip(\mathbb{X})\right) = \mathbb{P}^\mathbb{X}\left(\ip(\mathbb{X}) \neq \ip(\mathbb{X})\right)$). Wir haben also fast sicher $V(z) = V(\ip(\mathbb{X}))$. Da $V$ $\mathcal{F}_t^\mathcal{Z}$ ist, faktorisiert es über die Abbildung $z \mapsto z_{1:t}$. Es gilt also
$$V(\ip(\mathbb{X})) = \tilde{V}(\ip_{1:t}(\mathbb{X}))$$
$\ip(\mathbb{X})$ ist adaptiert an $\mathcal{F}_t^\mathbb{X}$, der Term auf der rechten Seite (und somit auch $V(\ip(\mathbb{X}))$) ist also $\mathcal{F}_t^\mathbb{X}$ messbar. Insgesamt erhalten Wir
\begin{align*}
    \mathbb{E}_\gamma(V \vert \mathcal{F}_{N, 0}^{\mathbb{X}, \mathcal{Z}}) &= \mathbb{E}_\gamma(V(\ip(\mathbb{X}))\vert \mathcal{F}_{N, 0}^{\mathbb{X}, \mathcal{Z}}) = V(\ip(\mathbb{X})) \\
    &= \mathbb{E}_\gamma(V(\ip(\mathbb{X})) \vert \mathcal{F}_{t, 0}^{\mathbb{X}, \mathcal{Z}}) = \mathbb{E}_\gamma(V \vert \mathcal{F}_{t, 0}^{\mathbb{X}, \mathcal{Z}})
\end{align*}
Für Kausalität von $\overline{\mathbb{X}}$ zu $\mathbb{X}$ überprüfen wir den zweiten Punkt von Lemma \ref{thm:causality_characterization}: Sei $U: \mathcal{Z}\rightarrow \mathbb{R}$ beschränkt und $\mathcal{F}_N^\mathcal{Z}$ messbar. Wieder gilt $U(z) = U(\ip(\mathbb{X}))$ $\gamma$ - fast sicher. Des weiteren gilt
\begin{equation}\label{eq:39i_1}
\mathbb{E}_\gamma(U \vert \mathcal{F}_{t,t}^{\mathbb{X}, \mathcal{Z}}) = \mathbb{E}_\gamma(U \vert \mathcal{F}_{t, 0}^{\mathbb{X}, \mathcal{Z}})
\end{equation}
denn jede $\mathcal{F}_{t,t}^{\mathbb{X}, \mathcal{Z}}$ messbare Funktion $G$ ist fast sicher $\mathcal{F}_{t, 0}^{\mathbb{X}, \mathcal{Z}}$ messbar: $G$ ist messbar bezüglich der kleinsten $\sigma$-Algebra, bezüglich der 
$$(\id, z_{1:t}): (\Omega^\mathbb{X}, \mathcal{F}^\mathbb{X})\otimes (\mathcal{Z}, \mathcal{F}^{\mathcal{Z}}) \rightarrow (\Omega^\mathbb{X}, \mathcal{F}_t^\mathbb{X}) \otimes (\mathcal{Z}_{1:t}, \mathcal{F}^{Z_{1:t}})$$
messbar ist. Somit faktorisiert $G$ darüber: $G(\omega, z) = \tilde{G}(\omega, z_{1:t}) = \tilde{G}(\omega, \ip_{1:t}(\mathbb{X}))$ fast sicher. Die letzte Abbildung ist aber $\mathcal{F}_{t,0}^{\mathbb{X}, \mathcal{Z}}$ messbar. Mit gleicher Argumentation ist 
\begin{equation}\label{eq:39i_2}
\mathbb{E}_\gamma(U \vert \mathcal{F}_{0, t}^{\mathbb{X}, \mathcal{Z}}) = \mathbb{E}(U \vert \ip_{1:t}(\mathbb{X}))
\end{equation}
Die linke Seite ist $\mathcal{F}_{0, t}^{\mathbb{X}, \mathcal{Z}}$ messbar und faktorisiert also als $G(z_{1:t})=G(\ip_{1:t}(\mathbb{X}))$ fast sicher, und ist somit messbar bezüglich $\sigma(\ip_{1:t}(\mathbb{X}))$.
Zuletzt gilt 
\begin{equation}\label{eq:39i_3}
\mathbb{E}_\gamma(U(\ip(\mathbb{X}))\vert \mathcal{F}_{t, 0}^{\mathbb{X}, \mathcal{Z}}) = \mathbb{E}_\gamma(U(\ip(\mathbb{X})) \vert \ip_{1:t}(\mathbb{X}))
\end{equation}
Nach Lemma \ref{thm:self_awareness} faktorisiert die linke Seite über $\ip_{1:t}(\mathbb{X})$ und ist also messbar diesbezüglich. Wir setzen nun die Gleichungen \ref{eq:39i_1}, \ref{eq:39i_2}, \ref{eq:39i_3} zusammen:
\begin{align*}
    \mathbb{E}_\gamma(U \vert \mathcal{F}_{t,t}^{\mathbb{X}, \mathcal{Z}}) &= \mathbb{E}_\gamma(U(\ip(\mathbb{X})) \vert \mathcal{F}_{t, 0}^{\mathbb{X}, \mathcal{Z}}) 
    = \mathbb{E}_\gamma(U(\ip(\mathbb{X}))\vert \ip_{1:t}(\mathbb{X})) \\
    &= \mathbb{E}_\gamma(U(\ip(\mathbb{X}))\vert \mathcal{F}_{0, t}^{\mathbb{X}, \mathcal{Z}}) = \mathbb{E}_\gamma(U \vert \mathcal{F}_{0, t}^{\mathbb{X}, \mathcal{Z}})
\end{align*}
\item Schreibe $\overline\pi:=(\ip(\mathbb{X}), \ip(\mathbb{Y}))_* \pi$ und sei $U:\mathcal{Z}\rightarrow \mathbb{R}$ $\mathcal{F}_{N}^{\mathcal{Z}}$-messbar und beschränkt. $U(\ip(\mathbb{X}))$ ist beschränkt und $\mathcal{F}_N^\mathbb{X}$ messbar. Mit Lemma \ref{thm:causality_characterization} gilt also
\begin{equation}\label{eq:310_0}
    \mathbb{E}_\pi(U(\ip(X))\vert \mathcal{F}_{t,t}^{\mathbb{X}, \mathbb{Y}}) = \mathbb{E}_\pi(U(\ip(\mathbb{X}))\vert \mathcal{F}_{t, 0}^{\mathbb{X}, \mathbb{Y}}) = \mathbb{E}_\pi(U(\ip(\mathbb{X})) \vert \ip_{1:t}(\mathbb{X}))
\end{equation}
wobei wir in der letzten Gleichheit wieder benutzt haben, dass auf $\mathcal{F}_t^\mathbb{X}$ bedingte Erwartungen nach Lemma \ref{thm:self_awareness} über $\ip_{1:t}(\mathbb{X})$ faktorisieren. Die kleinste $\sigma$-Algebra auf $\Omega^\mathbb{X}$, bezüglich der $\ip(\mathbb{X}): \Omega^{\mathbb{X}} \rightarrow (\mathcal{Z}, \mathcal{F}_t^\mathcal{Z})$ messbar ist, ist genau die kleinste $\sigma$-Algebra bezüglich der $\ip_{1:t}(\mathbb{X})$ messbar ist. Mit Lemma \ref{thm:pushforward_expectancy} gilt also
\begin{equation}\label{eq:310_1}
    \mathbb{E}_{\overline\pi}(U \vert \mathcal{F}_{t, 0}^{\mathcal{Z}, \mathcal{Z}})(\ip(\mathbb{X})) = \mathbb{E}_\pi(U(\ip(\mathbb{X})) \vert \ip_{1:t}(\mathbb{X})) = \mathbb{E}_\pi(U(\ip(\mathbb{X})) \vert \mathcal{F}_{t,t}^{\mathbb{X}, \mathbb{Y}})
\end{equation}
wobei die letzte Gleichheit nach Gleichung \ref{eq:310_0} gilt.
An dem zweiten Term sieht man, dass der dritte Term bereits $\sigma(\ip_{1:t}(\mathbb{X}), \ip_{1:t}(\mathbb{Y}))$-messbar ist. Gleichzeitig ist $\sigma(\ip_{1:t}(\mathbb{X}), \ip_{1:t}(\mathbb{Y})) \subset \mathcal{F}_{t,t}^{\mathbb{X}, \mathbb{Y}}$, da der Informationsprozess adaptiert ist. Mit der Turmeigenschaft erhalten wir
\begin{align}\label{eq:310_2}
    \begin{split}
\mathbb{E}_\pi(U(\ip(\mathbb{X})) \vert \mathcal{F}_{t,t}^{\mathbb{X}, \mathbb{Y}}) &= \mathbb{E}_\pi\left[\mathbb{E}_\pi(U(\ip(\mathbb{X}))\vert \mathcal{F}_{t,t}^{\mathbb{X}, \mathbb{Y}}) \vert \ip_{1:t}(\mathbb{X}), \ip_{1:t}(\mathbb{Y})\right] \\ 
&= \mathbb{E}_\pi\left(U(\ip(\mathbb{X})) \vert \ip_{1:t}(\mathbb{X}), \ip_{1:t}(\mathbb{Y})\right)
    \end{split}
\end{align}
Nun ist $\sigma(\ip_{1:t}(\mathbb{X}), \ip_{1:t}(\mathbb{Y}))$ die kleinste $\sigma$-Algebra, bezüglich derer 
$$\ip(\mathbb{X}), \ip(\mathbb{Y}): \Omega^\mathbb{X} \times \Omega^\mathbb{Y} \rightarrow (\mathcal{Z}\times \mathcal{Z}, \mathcal{F}_{t,t}^{\mathcal{Z}, \mathcal{Z}})$$
messbar ist, und somit gilt wieder mit Lemma \ref{thm:pushforward_expectancy}
\begin{equation}\label{eq:310_3}
    \mathbb{E}_\pi(U(\ip(\mathbb{X})) \vert \ip_{1:t}(\mathbb{X}), \ip_{1:t}(\mathbb{Y})) = \mathbb{E}_{\overline\pi}(U \vert \mathcal{F}_{t,t}^{\mathbb{Z}, \mathcal{Z}})(\ip(\mathbb{X}, \ip(\mathbb{Y})))
\end{equation}
Wir setzen die Gleichungen \ref{eq:310_1}, \ref{eq:310_2} und \ref{eq:310_3} zusammen und erhalten
$$\mathbb{E}_{\overline\pi}(U \vert \mathcal{F}_{t, 0}^{\mathcal{Z}, \mathcal{Z}})(\ip(\mathbb{X}), \ip(\mathbb{Y})) = \mathbb{E}_{\overline\pi}(U\vert \mathcal{F}_{t,t}^{\mathbb{X}, \mathbb{Y}})(\ip(\mathbb{X}, \ip(\mathbb{Y})))$$
$\pi$-fast sicher, und somit
$$\mathbb{E}_{\overline\pi}(U \vert \mathcal{F}_{t, 0}^{\mathcal{Z}, \mathcal{Z}}) = \mathbb{E}_{\overline\pi}(U\vert \mathcal{F}_{t,t}^{\mathbb{X}, \mathbb{Y}})$$
$\overline\pi$-fast sicher. Mit Lemma \ref{thm:causality_characterization} ist $\overline\pi$ kausal.
\item Antikausalität ist genau Kausalität mit vertauschten Rollen. Mit dem zweiten Schritt ist somit, falls $\pi$ bikausal ist, $(\ip(\mathbb{X}), \ip(\mathbb{Y}))_*\pi$ sowohl kausal als auch - durch Vertauschen der Rollen - antikausal. Somit ist es bikausal.
\end{enumerate}
\end{proof}
\begin{theorem}
    Seien $\mathbb{X}, \mathbb{Y} \in \mathcal{FP}_p$ und $\overline{\mathbb{X}}, \overline{\mathbb{Y}} \in \CFP_p$ die assoziierten kanonischen Prozesse. Dann gilt
    $$\mathcal{AW}_p(\mathbb{X}, \mathbb{Y}) = \mathcal{AW}_p(\overline{\mathbb{X}}, \overline{\mathbb{Y}}) = \mathcal{W}_p(\ip_1(\mathbb{X}), \ip_1(\mathbb{Y}))$$
\end{theorem}
\begin{proof}
\begin{enumerate}
\item Betrachte zunächst die erste Gleichheit. Beachte, dass für eine bikausale Kopplung $\pi \in \cplbc(\mathbb{X}, \mathbb{Y})$ nach dem letzten Lemma 
$$\overline{\pi}:=(\ip(\mathbb{X}), \ip(\mathbb{Y}))_*\pi \in \cplbc(\overline{\mathbb{X}}, \overline{\mathbb{Y}})$$
Weiterhin gilt für diese Kopplungen
\begin{align*}
    \mathbb{E}_{\overline{\pi}}(d^p(\overline{X}, \overline{Y}))&=\mathbb{E}_{\overline{\pi}}(d^p(Z^-(z), Z^-(\widetilde{z}))) \\
    &= \mathbb{E}_\pi(d^p(Z^-(\ip(\mathbb{X})), Z^-(\ip(\mathbb{Y})))) \\
    &= \mathbb{E}_\pi(d^p(X, Y))
\end{align*}
Da wir also jede bikausale Kopplung von $\mathbb{X}$ und $\mathbb{Y}$ zu einer von $\overline{\mathbb{X}}$ und $\overline{\mathbb{Y}}$ übertragen können, die die gleichen Kosten erzeugt, gilt auf jeden Fall $\mathcal{AW}_p(\mathbb{X}, \mathbb{Y}) \geq \mathcal{AW}_p(\overline{\mathbb{X}}, \overline{\mathbb{Y}})$. Für die andere Ungleichung benötigen wir eine Approximation von $\overline{\pi} \in \cplbc(\overline{\mathbb{X}}, \overline{\mathbb{Y}})$ durch Verteilungen der Form $(\ip(\mathbb{X}), \ip(\mathbb{Y}))_*\pi, \pi \in \cplbc(\mathbb{X}, \mathbb{Y})$. In dem dieser Arbeit zu Grunde liegenden Paper zeigen die Autoren, das Verteilungen dieser Form tatsächlich dicht in $\cplbc(\overline{\mathbb{X}}, \overline{\mathbb{Y}})$ liegen. Da die Kostenfunktion stetig bezüglich der Wassersteinmetrik ist ($d^p$ hat $p$-Wachstum und die Evaluationen $Z^-$ sind kontraktiv), wäre damit die erste Gleichheit gezeigt. Der Beweis ist aber hoch technisch, wir betrachten hier daher nur den einfachereren Fall, dass $\Omega^\mathbb{X}$ und $\Omega^\mathbb{Y}$ polnisch sind und somit Disintegration erlauben. 

Sei $\overline{\pi} \in \cplbc(\overline{\mathbb{X}}, \overline{\mathbb{Y}})$. Wir schreiben
$$\gamma := (\id, \ip(\mathbb{X}))_*\mathbb{P}^\mathbb{X} \text{ und } \hat{\gamma}:=(\id, \ip(\mathbb{Y}))_*\mathbb{P}^\mathbb{Y}$$
Da die zugrundeliegenden Räume polnisch sind, können wir die Verteilungen disintegrieren. Wir schreiben $(\gamma_z)_{z \in \mathcal{Z}}:=\mathcal{L}(\id_{\Omega^\mathbb{X}} \vert \ip(\mathbb{X}))$ und $(\hat{\gamma}_{\hat{z}})_{\hat{z} \in \mathcal{Z}}:=\mathcal{L}(\id_{\Omega^\mathbb{Y}} \vert \ip(\mathbb{Y}))$.
Betrachte die Verteilung 
$$\pi(A\times B) := \int \gamma_z(A) \gamma_{\hat{z}}(B)\overline{\pi}(dz, d\hat{z})$$
Mit Korollar \ref{thm:pushforward_law} ist $\ip(\mathbb{X})_*\mathcal{L}(\id \vert \ip(\mathbb{X}))= \mathcal{L}(\ip(\mathbb{X}) \vert \ip(\mathbb{X}))= \delta_{z}$ und genauso $\ip(\mathbb{Y})_*\mathcal{L}(\id \vert \ip(\mathbb{Y})) = \delta_{\hat{z}}$. Aus diesem Grund ist
\begin{align*}
(\ip(\mathbb{X}), \ip(\mathbb{Y}))_*\pi(A \times B) &= \int \gamma_z(\ip(\mathbb{X})^{-1}(A)) \gamma_{\hat{z}}(\ip(\mathbb{Y})^{-1}(B))\overline{\pi}(dz, d\hat{z}) \\
&= \int \ip(\mathbb{X})_*\gamma_z(A) \ip(\mathbb{Y})_*\gamma_{\hat{z}}(B) \overline{\pi}(dz, d\hat{z}) \\
&= \int \delta_z(A) \delta_{\hat{z}}(B)\overline{\pi}(dz, d\hat{z}) \\
&= \overline{\pi}(A \times B)
\end{align*}
Wenn wir nun also beweisen, dass $\pi \in \cplbc(\mathbb{X}, \mathbb{Y})$, so haben wir die erste Gleichheit gezeigt. Zunächst ist $\pi$ überhaupt eine Kopplung,  da 
$$\pi(A\times \Omega^{\mathbb{Y}}) = \int \gamma_z(A) \overline{\pi(dz, d\hat{z})} = \int \gamma_z(A) \mathbb{P}^{\ip(\mathbb{X})}(dz) = \mathbb{P}^\mathbb{X}(A)$$
und das gleiche für die zweite Komponente.

Aufgrund der Symmetrie zeigen wir nur die Kausalität von $\pi$ über Lemma \ref{thm:causality_characterization}. Sei dazu $V$ beschränkt und $\mathcal{F}_{t}^{\mathbb{Y}}$ messbar. Wir müssen zeigen, dass 
$$\mathbb{E}_\pi(V \vert \mathcal{F}_{N, 0}^{\mathbb{X}, \mathbb{Y}}) = \mathbb{E}_\pi(V \vert \mathcal{F}_{t, 0}^{\mathbb{X}, \mathbb{Y}})$$
Wir bemerken zunächst, dass wir $(\hat{\gamma})_{z\in\mathcal{Z}}$ auch hätten schreiben können als $\mathcal{L}(\id_{\Omega^\mathbb{Y}} \vert \id_{\mathcal{Z}})$, da Disintegration nur eine Frage der Verteilungen im Wertebereich ist, und bezüglich $\gamma$ die Variablen $(\id_\Omega^\mathbb{Y}, \id_{\mathcal{Z}})$ genauso verteilt sind wie $(\id_{\Omega^\mathbb{Y}}, \ip(\mathbb{Y}))$ bezüglich $\mathbb{P}^\mathbb{Y}$. Somit ist nach Lemma \ref{thm:law_expectancy_connection}
$$\mathbb{E}_{\hat{\gamma}}(V \vert \mathcal{F}_{0,N}^{\mathbb{Y}, \overline{\mathbb{Y}}}) = \int V(\hat{\omega})\hat{\gamma}_{\hat{z}}(d\hat{\omega}) \quad \text{fast sicher}$$
Da $\hat{\gamma}$ bikausal ist, ist mit Lemma \ref{thm:causality_characterization} die linke Seite $\mathcal{F}_{0,t}^{\mathbb{Y}, \overline{\mathbb{Y}}}$ messbar, somit kann die rechte Seite auf einer $\mathbb{P}^\mathbb{Y}$-Nullmenge abgeändert werden um $\mathcal{F}_t^{\overline{\mathbb{Y}}}$ messbar zu werden. Mit der gleichen Argumentation ist, weil $\overline{\pi}$ bikausal ist, auch
$$\mathbb{E}_{\overline{\pi}}\left(\int V(\hat{\omega})\hat{\gamma}_{\hat{z}}(d\hat{\omega}) \left\vert \mathcal{F}_{N,0}^{\overline{\mathbb{X}}, \overline{\mathbb{Y}}}\right.\right)=\int\left(\int V(\hat{\omega})\hat{\gamma}_{\hat{z}}(d\hat{\omega})\right)\overline{\pi}_z(d\hat{z})$$
$\mathcal{F}_t^{\overline{\mathbb{X}}}$-messbar, da der Integrand wie vorher festgestellt $\mathcal{F}_t^{\overline{\mathbb{Y}}}$-messbar ist (mit $\overline{\pi}_z$ ist hier die Disintegration von $\overline{\pi}$ bezüglich der ersten Marginalie gemeint). Und wieder mit der gleichen Argumentation ist nun für $\gamma_\omega(dz):=\mathcal{L}(\ip(\mathbb{X}) \vert \mathcal{F}_N^{\mathbb{X}})(\omega)$ auch
$$\mathbb{E}_{\gamma}\left( \int \int V(\hat{\omega}) \hat{\gamma}_{\hat{z}}(d\hat{\omega}) \overline{\pi}_z(d\hat{z}) \left\vert \mathcal{F}_{N,0}^{\mathbb{X}, \overline{\mathbb{X}}} \right. \right) = \int\int\int V(\hat{\omega}) \hat{\gamma}_{\hat{z}}(d\hat{\omega}) \overline{\pi}_z(d\hat{z})\gamma_\omega(dz)$$
$\mathcal{F}_t^{\mathbb{X}}$-messbar. Der Term auf der rechten Seite ist aber gerade $\mathbb{E}_\pi(V \vert \mathcal{F}_{N,0}^{\mathbb{X}, \mathbb{Y}})$, weil für jedes $\mathcal{F}_{N}^{\mathbb{X}}$-messbare $U$ gilt
\begin{align*}
    \mathbb{E}_{\pi}\biggl(U(\omega) \int\int\int &V(\hat{\omega}) \hat{\gamma}_{\hat{z}}(d\hat{\omega}) \overline{\pi}_z(d\hat{z})\gamma_\omega(dz) \biggr) \\
    &= \mathbb{E}_{\mathbb{P}^{\mathbb{X}}}\left(U(\omega) \int\int\int V(\hat{\omega}) \hat{\gamma}_{\hat{z}}(d\hat{\omega}) \overline{\pi}_z(d\hat{z})\gamma_\omega(dz) \right) \\
    &= \int\int\int\int V(\hat{\omega}) \hat{\gamma}_{\hat{z}}(d\hat{\omega}) \overline{\pi}_z(d\hat{z})U(\omega)\gamma_\omega(dz) \mathbb{P}^{\mathbb{X}}(d\omega) \\
    &= \int\int\int V(\hat{\omega}) \hat{\gamma}_{\hat{z}}(d\hat{\omega}) \overline{\pi}_z(d\hat{z})U(\omega)\gamma(d\omega, dz) \\
    &= \int\int\int\int V(\hat{\omega}) \hat{\gamma}_{\hat{z}}(d\hat{\omega}) \overline{\pi}_z(d\hat{z})U(\omega)\gamma_z(d\omega)\mathbb{P}^{\ip(\mathbb{X})}(dz)\\
    &= \int\int\int\int V(\hat{\omega}) \hat{\gamma}_{\hat{z}}(d\hat{\omega}) \overline{\pi}_z(d\hat{z})U(\omega)\gamma_z(d\omega)\mathbb{P}^{\ip(\mathbb{X})}(dz) \\
    &= \int\int\left(\int U(\omega)\gamma_z(d\omega) \int V(\hat{\omega}) \hat{\gamma}_{\hat{z}}(d\hat{\omega}) \right)\overline{\pi}_z(d\hat{z})\mathbb{P}^{\ip(\mathbb{X})}(dz) \\
    &= \int\left(\int U(\omega)\gamma_z(d\omega) \int V(\hat{\omega}) \hat{\gamma}_{\hat{z}}(d\hat{\omega}) \right)\overline{\pi}(dz, d\hat{z}) \\
    &= \int U(\omega)V(\hat{\omega})\pi(d\omega, d\hat{\omega})
\end{align*}
Insgesamt ist also $\mathbb{E}_\pi(V\vert \mathcal{F}_{N,0}^{\mathbb{X}, \mathbb{Y}})$ schon $\mathcal{F}_{t,0}^{\mathbb{X}, \mathbb{Y}}$ messbar, also
$$\mathbb{E}_\pi(V \vert \mathcal{F}_{N,0}^{\mathbb{X}, \mathbb{Y}}) = \mathbb{E}_\pi(V \vert \mathcal{F}_{t,0}^{\mathbb{X}, \mathbb{Y}})$$
und somit $\pi\in \cplc(\mathbb{X}, \mathbb{Y})$. Wegen der Symmetrie gilt $\pi \in \cplbc(\mathbb{X}, \mathbb{Y})$ und wie oben besprochen folgt $\mathcal{AW}_p(\mathbb{X}, \mathbb{Y}) = \mathcal{AW}_p(\overline{\mathbb{X}}, \overline{\mathbb{Y}})$.
\item
Wir beweisen nun die zweite Gleichheit. Schreibe $\mu := \mathcal{L}(\ip_1(\mathbb{X}))$ und $\nu := \mathcal{L}(\ip_1(\mathbb{Y}))$. Mit Lemma \ref{thm:causality_kernel_characterization} gilt
$$\mathcal{AW}_p^p(\overline{\mathbb{X}}, \overline{\mathbb{Y}}) = \inf_{\pi \in \cpl(\mu, \nu)} \inf_{(k_{t=1}^{N-1})} \int \sum_{t=1}^{N}d^p(z_t^-, \hat{z}_t^-) (\pi_1 \otimes k_1 \otimes ... \otimes k_{N-1})(dz, d\hat{z})$$
wobei das zweite Infimum über Kerne der Form
\begin{equation}\label{eq:311_0}
k_t: \mathcal{Z}_{1:t}\times \mathcal{Z}_{1:t} \rightarrow \mathcal{P}_p(\mathcal{Z}_{t+1}\times \mathcal{Z}_{t+1}) \text{ mit } k_t^{z_{1:t}, \hat{z}_{1:t}} \in \cpl(z_t^+, \hat{z}_t^+)
\end{equation}
läuft. Nach Lemma \ref{thm:optimal_coupling} können wir für jedes $1 \leq t \leq N-1$ $k_t^*$ als einen solchen messbaren Kern wählen, dass $k_t^{*, z_{1:t}, \hat{z}_{1:t}}$ eine optimale Kopplung von $z_t^+$ und $\hat{z}_t^-$ ist. Für jedes $1\leq t\leq N-1$, $z_{1:t}, \hat{z}_{1:t} \in \mathcal{Z}_{1:t}$ und Kerne wie in \ref{eq:311_0}:
\begin{align}\label{eq:311_1}
    \begin{split}
    d^p(z_t, \hat{z}_t) &= d^p(z_t^-, \hat{z}_t^-) + \mathcal{W}_p^p(z_t^+, \hat{z}_t^+) \\
    &\leq d^p(z_t^-, \hat{z}_t^-) + \int d^p(z_{t+1}, \hat{z}_{t+1}) k_t^{z_{1:t}, \hat{z}_{1:t}}(dz_{t+1}, d\hat{z}_{t+1})
    \end{split}
\end{align}
und es gilt Gleichheit für die Kerne $k_t^*$. Induktiv gilt nun für $1 \leq M \leq N$, dass
$$\int d^p(z_1, \hat{z}_1)\pi_1(dz, d\hat{z}) \leq \int \sum_{t=1}^{M-1} d^p(z_t^-, \hat{z}_t^-) +d^p(z_M, \hat{z}_M)(\pi_1 \otimes k_1 \otimes ... \otimes k_{N-1})(dz, d\hat{z})$$
mit Gleichheit für $k_t^*$: Für $M=1$ ist sind beide Seiten der Gleichung identisch. Nehme nun an, die Aussage gelte für ein fixes $M$. Dann folgt die Gültigkeit für $M+1$ durch
\begin{alignat*}{2}
\int &d^p(z_1, \hat{z}_1)\pi_1(dz_1, d\hat{z}_1) \\
&\leq \int \sum_{t=1}^{M-1} d^p(z_t^-, \hat{z}_t^-) +d^p(z_M, \hat{z}_M)(\pi_1 \otimes k_1 \otimes ... \otimes k_{M-1})(dz_{1:M}, d\hat{z}_{1:M}) \\
&\leq\int \sum_{t=1}^{M}d^p(z_t^-, \hat{z}_t^-) + \\
    &\quad\int d^p(z_{M+1}, \hat{z}_{M+1}) k_M^{z_{1:M}, \hat{z}_{1:M}}(dz_{M+1}, \hat{z}_{M+1}) (\pi_1 \otimes k_1 \otimes ... \otimes k_{M-1})(dz_{1:M}, d\hat{z}_{1:M})\\
&= \int \sum_{t=1}^{M}d^p(z_t^-, \hat{z}_t^-) + d^p(z_{M+1}, \hat{z}_{M+1})(\pi_1 \otimes ...\otimes k_M)(dz_{1:M+1}, d\hat{z}_{1:M+1})
\end{alignat*}
Die erste Ungleichung gilt nach Induktionsvoraussetzung und auch mit Gleichheit für $k_t^*$. Die zweite Ungleichung (und Gleichheit für $k_t^*$) ist genau Gleichung \ref{eq:311_1}.
Für $M=N-1$ erhalten Wir
$$\int d^p(z_1, \hat{z}_1)\pi_1(dz_1, d\hat{z}_1) = \inf_{(k_t^{N-1})} \sum_{t=1}^Nd^p(z_t^-, \hat{z}_t^-)(\pi_1 \otimes ... \otimes k_{N-1})(dz, d\hat{z})$$
Indem wir das Infimum über $\pi_1 \in \cpl(\mu, \nu)$ auf beiden Seiten nehmen, erhalten wir
$$\mathcal{AW}_p(\overline{\mathbb{X}}, \overline{\mathbb{Y}}) = \mathcal{W}_p(\mu, \nu) = \mathcal{W}_p(\ip_1(\mathbb{X}), \ip_1(\mathbb{Y}))$$
\end{enumerate}
\end{proof}

\section{Appendix}
Die folgende Charakterisierung von bikausalen Kopplungen stammt aus \cite[Lemma A.1]{main_paper}
\begin{lemma} \label{thm:causality_kernel_characterization}
Seien $\mathbb{X}, \mathbb{Y} \in \CFP_p$, $\pi \in \mathcal{P}_p(\mathcal{Z}\times\mathcal{Z})$ und setze $\pi_1 := \operatorname{pj}_{\mathcal{Z}_1\times \mathcal{Z}_1}\pi$. Dann ist $\pi \in \cplbc(\mathbb{X}, \mathbb{Y})$ genau dann, wenn
$$\pi_1 \in \cpl(\mathcal{L}(\ip_1(\mathbb{X})), \mathcal{L}(\ip_1(\mathbb{Y})))$$
und für $1\leq t \leq N-1$ Kerne
$$k_t:\mathcal{Z}_{1:t} \times \mathcal{Z}_{1:t} \rightarrow \mathcal{P}_p(\mathcal{Z}_{t+1} \times \mathcal{Z}_{t+1}) \text{ mit } k_t^{z_{1:t}, \hat{z}_{1:t}} \in \cpl(z_t^+, \hat{z}_t^+)$$
existieren, sodass
$$\pi = \pi_1 \otimes k_1 \otimes ... \otimes k_{N-1}$$
\end{lemma}
\begin{proof}
Da $\mathbb{X}, \mathbb{Y} \in \CFP_p$ können wir die Verteilungen schreiben als $\mathbb{P}^{\mathbb{X}} = \uf_1(\mu)$, $\mathbb{P}^{\mathbb{Y}}=\uf_1(\nu)$ für $\mu, \nu \in \mathcal{P}_p(\mathcal{Z}_1)$.
\begin{enumerate}
\item
Nehmen wir zunächst an, dass $\pi \in \cplbc(\mathbb{X}, \mathbb{Y})$. Dann können wir die Kerne wählen als 
$$k_t = \mathcal{L}_\pi(z_{t+1}, \hat{z}_{t+1} \vert \mathcal{F}_{t,t}^{\mathcal{Z}, \mathcal{Z}})$$
Für diese Wahl gilt $\pi = \pi_1 \otimes k_1 \otimes ... \otimes k_{N-1}$ und mit Lemma \ref{thm:pushforward_law} sind die Marginalien gegeben durch 
$${\operatorname{pj}_1}_*\mathcal{L}_\pi(z_{t+1}, \hat{z}_{t+1} \vert \mathcal{F}_{t,t}^{\mathcal{Z}, \mathcal{Z}}) = \mathcal{L}_\pi(\operatorname{pj}_1(z_{t+1}, \hat{z}_{t+1}) \vert \mathcal{F}_{t,t}^{\mathcal{Z}, \mathcal{Z}}) = \mathcal{L}_\pi(z_{t+1},  \vert \mathcal{F}_{t,t}^{\mathcal{Z}, \mathcal{Z}})$$
und $\mathcal{L}_\pi(\hat{z}_{t+1} \vert \mathcal{F}_{t,t}^{\mathcal{Z},\mathcal{Z}})$. Da $\pi$ bikausal ist, gilt nach Lemma \ref{thm:causality_characterization} 
$$\mathcal{L}_\pi(z_{t+1} \vert \mathcal{F}_{t,t}^{\mathcal{Z,Z}}) = \mathcal{L}_\pi(z_{t+1} \vert \mathcal{F}_{t,0}^{\mathcal{Z,Z}})=\mathcal{L}_{\mathbb{P}^{\mathbb{X}}}(z_{t+1} \vert \mathcal{F}_t^\mathcal{Z})$$
wobei wir für die letzte Gleichheit Lemma \ref{thm:pushforward_expectancy} auf die Projektion auf die erste Koordinate angewendet haben. Aufgrund der Form $\mathbb{P}^{\mathbb{X}} = \uf_1(\mu)$ gilt aber $\mathcal{L}_{\mathbb{P}^{\mathbb{X}}}(z_{t+1} \vert \mathcal{F}_t^{\mathcal{Z}}) = z_t^+$: Dieser Kern ist $\mathcal{F}_t^{\mathcal{Z}}$-messbar, und die induzierte Verteilung ist 
$$\mathcal{L}_{\mathbb{P}^{\mathbb{X}}}(z_{1:t}) \otimes z_t^+ = \mu \otimes z_1^+ \otimes ... \otimes z_{t-1}^+  \otimes z_t^+ = \operatorname{pj}_{1:t+1}(\uf_1(\mu)) = \mathcal{L}_{\mathbb{P}^{\mathbb{X}}}(z_1,...,z_{t+1})$$
Es gilt also $\mathcal{L}_\pi(z_{t+1} \vert \mathcal{F}_{t,t}^{\mathcal{Z,Z}}) = z_t^+$ und analog $\mathcal{L}_\pi(\hat{z}_{t+1} \vert \mathcal{F}_{t,t}^{\mathcal{Z,Z}}) = \hat{z}_t^+$ und somit $k_t^{z_{1:t}, \hat{z}_{1:t}} \in \cpl(z_t^+, \hat{z}_t^+)$.

Nach Bemerkung \ref{thm:ip_of_canonical_process} ist $\mathcal{L}(\ip_1(\mathbb{X})) = \mu = \pj_1(\mathbb{P}^\mathbb{X}) = \pj_1(\pi_1)$. Es gilt also auch $\pi_1 \in \cpl(\mathcal{L}(\ip_1(\mathbb{X})), \mathcal{L}(\ip_1(\mathbb{Y})))$.
\item
Nehme nun umgekehrt an, dass es solche Kerne $k_t$ gibt. Dann ist 
$$\operatorname{pj}_1(\pi) = \operatorname{pj}_1(\pi_1 \otimes k_1 \otimes ... \otimes k_{N-1}) = \mu \otimes z_1^+ \otimes ... \otimes z_{N-1}^+ = \uf_1(\mu) = \mathbb{P}^\mathbb{X}$$
und $\operatorname{pj}_2(\pi) = \mathbb{P}^\mathbb{Y}$ und somit $\pi \in \cpl(\mathbb{X}, \mathbb{Y})$. 

Weiterhin gilt $k_t = \mathcal{L}_\pi(z_{t+1}, \hat{z}_{t+1} \vert \mathcal{F}_{t,t}^{\mathcal{Z,Z}})$, da $\mathcal{L}(z_{1:t}, \hat{z}_{1:t}) \otimes k_t = \mathcal{L}(z_{1:t+1}, \hat{z}_{1:t+1})$. Für ein $\mathcal{F}_{t+1}^\mathcal{Z}$-messbares $U$ gilt dann
\begin{align*}
    \mathbb{E}_\pi(U(z_{1:t+1}) \vert \mathcal{F}_{t,t}^{\mathcal{Z,Z}}) &= \int U(z_{1:t}, z_{t+1}) k_t^{z_{1:t}, \hat{z}_{1:t}}(dz_{t+1}, d\hat{z}_{t+1}) \\
    &= \int U(z_{1:t}, z_{t+1}) z_t^+(dz_{t+1})
\end{align*}
und der letzte Term ist $\mathcal{F}_{t,0}^{\mathcal{Z,Z}}$-messbar. Mit der Turmeigenschaft folgt also für ein $\mathcal{F}_N^\mathcal{Z}$-messbares $U$, dass $\mathbb{E}_\pi\left(U \vert \mathcal{F}_{t,t}^{\mathcal{Z,Z}}\right)$ $\mathcal{F}_{t,0}^{\mathcal{Z,Z}}$-messbar ist und damit Kausalität von $\pi$. Aus Symmetriegründen ist $\pi$ bikausal, also $\pi \in \cplbc(\mathbb{X}, \mathbb{Y})$.
\end{enumerate}
\end{proof}
Der folgende Satz stammt aus \cite[Satz 4.5]{markov_processes_ethier} und charakterisiert separierende Teilmengen von $C_b$ für Verteilungen:
\begin{theorem} \label{thm:separating_measures}
    Sei $(A, d)$ ein polnischer Raum und $M \subset C_b(A)$ eine punktetrennende Algebra, das heißt für $x,y\in A$ existiert ein $f\in M$ mit $f(x)\neq f(y)$. Dann ist $M$ \emph{separierend} für $\mathcal{P}(A)$, das heißt für $\mu, \nu \in \mathcal{P}(A)$ mit
    \begin{equation}\label{eq:A20}
        \mu(f)=\nu(f) \quad \forall f \in M 
    \end{equation}
    gilt $\mu=\nu$.
\end{theorem}
\begin{proof}
    Seien $\mu, \nu \in \mathcal{P}(A)$ mit $\mu(f)=\nu(f)\forall f\in M$. Da $M$ eine Algebra ist, ist auch $H:=\{f+a \vert f \in M, a\in \mathbb{R}\}$ eine Algebra, und Gleichung \ref{eq:A20} gilt auch für $h \in H$. Sei $\varepsilon>0$. Da $(A, d)$ polnisch ist, sind $\mu$ und $\nu$ straff, wir können also kompakte $K_1,K_2 \subset A$ wählen, sodass $\mu(K_1)\geq 1-\varepsilon$ und $\nu(K_2)\geq 1-\varepsilon$. Für das kompakte $K:=K_1\cup K_2$ gilt dann $\mu(K)\geq 1-\varepsilon$ und $\nu(K)\geq 1-\varepsilon$. Sei $g \in C_b(A)$ beliebig. Die Menge $\{h_{\vert K} \vert h \in H\}$ ist eine punktetrennende Algebra und enthält die Funktion $h\equiv 1$. Nach dem Satz von Stone-Weierstraß liegt sie dicht in $C_b(K)$ bezüglich $\|\cdot \|_\infty$. Wir können also $(g_n) \subset H$ wählen mit $\sup\limits_{x\in K} |g_n(x)-g(x)| \rightarrow 0$ für $n\rightarrow \infty$. Für jedes $g_n$ gilt

    \begin{equation}\label{eq:A21}
    \int_A g_n \exp(-\varepsilon g_n^2) d\mu = \int_A g_n \exp(-\varepsilon g_n^2) d\nu
    \end{equation}
    nach dominierter Konvergenz, da 
    $$ g_n \sum_{k=0}^{N} \frac{(-\varepsilon g_n^2)^k}{k!} \rightarrow g_n \exp(-\varepsilon g_n^2)$$
    dominiert durch $\exp\left(\|g_n\|_{\infty}^2\right)$, und die Funktionen auf der linken Seite liegen in der Algebra $H$. Es gilt
    \begin{align*}
        \left| \int ge^{-\varepsilon g^2} \right.&\left.d\mu - \int ge^{-\varepsilon g^2}d\nu \right|  
        \leq \left| \int_S ge^{-\varepsilon g^2}d\mu - \int_K ge^{-\varepsilon g^2}d\mu \right| \\
        &+ \left| \int_K ge^{-\varepsilon g^2}d\mu - \int_K g_ne^{-\varepsilon g_n^2}d\mu \right| 
        + \left| \int_K g_ne^{-\varepsilon g_n^2}d\mu - \int_S g_ne^{-\varepsilon g_n^2}d\mu \right| \\
        &+ \left| \int_S g_ne^{-\varepsilon g_n^2}d\mu - \int_S g_ne^{-\varepsilon g_n^2}d\nu \right| 
        + \left| \int_S g_ne^{-\varepsilon g_n^2}d\nu - \int_K g_ne^{-\varepsilon g_n^2}d\nu \right| \\
        &+ \left| \int_K g_ne^{-\varepsilon g_n^2}d\nu - \int_K ge^{-\varepsilon g^2}d\nu \right| 
        + \left| \int_K ge^{-\varepsilon g^2}d\nu - \int_S ge^{-\varepsilon g^2}d\nu \right| \\
    \end{align*}
    Der 4. Summand auf der rechten Seite ist 0 nach Gleichung \ref{eq:A21}. Die Summanden 1, 3, 5 und 7 sind jeweils beschränkt durch $\gamma \sqrt{\varepsilon}, \gamma:=\sup_{t\geq 0}te^{-t^2}$, da 
    $$te^{-\varepsilon t^2}=\frac{1}{\sqrt{\varepsilon}}(\sqrt{\varepsilon} t)e^{-(\sqrt{\varepsilon}t)^2} \leq \frac{1}{\sqrt{\varepsilon}} \gamma \, \text{ und } \, \mu(S\setminus K), \nu(S\setminus K)<\varepsilon$$
    Die Summanden 2 und 6 konvergieren gegen $0$ für $n\rightarrow \infty$, da $g_n \rightarrow g$ gleichmäßig auf $K$ und $t \mapsto te^{-\varepsilon t^2}$ lipschitzstetig ist, also auch $g_n e^{-\varepsilon g_n^2} \rightarrow g e^{-\varepsilon g^2}$ gleichmäßig auf $K$. Wir erhalten für $n\rightarrow \infty$
    $$\left|\int ge^{-g^2\varepsilon}d\mu - \int ge^{-\varepsilon g^2} d\nu \right|\leq 4\gamma\sqrt{\varepsilon}$$
    Für $\varepsilon \rightarrow 0$ folgt mit dominierter Konvergenz ($g$ und somit $ge^{-\varepsilon g^2}$ sind unabhängig von $\varepsilon$ beschränkt und $ge^{-\varepsilon g^2} \rightarrow g$ punktweise für $\varepsilon \rightarrow 0$), dass
    $$\int gd\mu = \int gd\nu$$

    Sei nun $E \subset A$ abgeschlossen. Für $n\in \mathbb{N}$ sind die Funktionen $g_n: x \mapsto 1 - n \cdot d(x, E)$ stetig und beschränkt, also $\mu(g_n) = \nu(g_n)$ für alle $n \in \mathbb{N}$. Es gilt $g_n \rightarrow \mathds{1}_E$ dominiert durch $1$, mit dominierter Konvergenz gilt also $\mu(E)=\nu(E)$. Abgeschlossene Mengen sind ein $\cap$-stabiler Erzeuger von $\mathcal{B}(A)$, insgesamt folgt also $\mu=\nu$.
\end{proof}

\end{document}