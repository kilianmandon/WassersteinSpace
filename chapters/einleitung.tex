Diese Arbeit bespricht die \emph{adaptierte Wasserstein-Metrik} wie sie von Daniel Bartl, Mathias Beiglböck und Gudmund Pammer in \cite{main_paper} vorgestellt wurde. Die adaptierte Wasserstein-Metrik $\mathcal{AW}_p$ ist eine Metrik auf dem Raum adaptierter stochastischer Prozesse $\FP_p$, die nicht nur die Verteilung, sondern auch die Filtration der Prozesse berücksichtigt. Für viele wichtige Operationen der Wahrscheinlichkeitstheorie wie die Doob-Zerlegung, Optimal-Stopping-Probleme oder Martingale ist die Berücksichtigung der gegebenen Filtration essentiell. Die adaptierte Wasserstein-Metrik kann diesen Anspruch in vielen Punkten erfüllen, so zeigen wir etwa in Abschnitt \ref{ch:topological_properties} die Stetigkeit der Doob-Zerlegung und die Stetigkeit des Ertrags eines Optimal-Stopping-Problems. 

Die Arbeit beginnt mit einer ausführlichen Vorstellung der Grundlagen für die adaptierte Wasserstein-Metrik. Kapitel \ref{ch:basics} bespricht die klassische Wasserstein-Metrik, bedingte Erwartungen und Verteilungen sowie die Eigenschaft der Bikausalität von Kopplungen. Es werden viele kleinere Aussagen gemacht, die für die übrige Arbeit wichtig sind.

Die Kapitel \ref{ch:wasserstein_space}, \ref{ch:adapted_functions} und \ref{ch:topological_properties} erarbeiten die wichtigsten Aussagen des zugrundeliegenden Papers \cite{main_paper} und stellen die Beweise in ausführlicherer Form dar. In Kapitel \ref{ch:wasserstein_space} wird ein polnischer Raum $(\mathcal{Z}_1, d)$ konstruiert, sodass $(\FP_p, \mathcal{AW}_p)$ isomorph zu $\mathcal{P}_p(\mathcal{Z}_1)$ mit der klassischen Wasserstein-Metrik ist. In Kapitel \ref{ch:adapted_functions} werden die \emph{adaptierten Funktionen} vorgestellt, die für adaptierte stochastische Prozesse separierend wirken: Zwei Prozesse $\mathbb{X,Y}$ sind genau dann gleich bezüglich der adaptierten Wasserstein-Metrik, wenn die Erwartungswerte $\mathbb{E}(f(\mathbb{X}))$ und $\mathbb{E}(f(\mathbb{Y}))$ für alle adaptierten Funktionen $f$ übereinstimmen. In Kapitel \ref{ch:topological_properties} werden wichtige geometrische und topologische Eigenschaften der adaptierten Wasserstein-Metrik hergeleitet. Es wird gezeigt, dass eine Familie $(\mathbb{X}_i)_{i\in I}$ genau dann relativ kompakt ist, wenn die Verteilungen $(\mathcal{L}(X_i))_{i\in I}$ relativ kompakt bezüglich der Wasserstein-Metrik sind, das heißt wenn $(\mathcal{L}(X_i))_{i\in I}$ straff und uniform $p$-integrierbar sind. Außerdem wird die Stetigkeit von wichtigen Operationen wie der Doob-Zerlegung und Optimal-Stopping gezeigt, sowie die Eigenschaft, dass $\FP_p$ ein geodätischer Raum ist.

Zusätzlich zur Vorstellung dieser Ergebnisse des Papers werden als eigene Resultate in Proposition \ref{thm:awp_convergence_char} die Konvergenz bezüglich $\mathcal{AW}_p$ charakterisiert und in Kapitel \ref{ch:implementation} der Fall von endlichen Wahrscheinlichkeitsräumen betrachtet, in welchem die adaptierte Wasserstein-Metrik mit Methoden der linearen Programmierung berechnet werden kann.