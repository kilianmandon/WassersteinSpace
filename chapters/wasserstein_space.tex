\begin{definition}[Kanonischer Raum]
    Sei $p \in [1, \infty)$. Wir definieren iterativ eine Folge von geschachtelten Räumen wie folgt: Schreibe $(\mathcal{Z}_N, d_{\mathcal{Z}_N}) := (\mathcal{X}_N, d_{\mathcal{X}_N})$ und für $t=N-1, ..., 1$ setze
    \begin{equation}
        \mathcal{Z}_t := \mathcal{Z}_t^- \times \mathcal{Z}_t^+ := \mathcal{X}_t \times \mathcal{P}_p(\mathcal{Z}_{t+1})
    \end{equation}
    mit der Metrik $d^p_{\mathcal{Z}_t} := d^p_{\mathcal{X}_t} + \mathcal{W}^p_{p, \mathcal{Z}_{t+1}}$. Elemente von $\mathcal{Z}_t$ schreiben wir als $z_t=(z_t^-, z_t^+) \in \mathcal{Z}_t^- \times \mathcal{Z}_t^+$. Der \emph{kanonische filtrierte Raum} ist nun
    \begin{equation}
        (\mathcal{Z}, \mathcal{F}^\mathcal{Z}, (\mathcal{F}_t^\mathcal{Z})_{t=1}^N)
    \end{equation}
    wobei $\mathcal{Z} = \mathcal{Z}_{1:N}$, $\mathcal{F}^\mathcal{Z}$ die Borelsche $\sigma$-Algebra auf $\mathcal{Z}$ und $\mathcal{F}^\mathcal{Z}_t=\sigma(z \mapsto z_{1:t})$ die kleinste $\sigma$-Algebra ist, die die Projektionen bis zur $t$-ten Koordinate messbar macht.
\end{definition}
Die auf diese Weise konstruierten Räume sind polnisch, da $\mathcal{P}_p$ über einem polnischen Raum nach Lemma \ref{thm:pp_is_polish} polnisch ist und das Produkt von polnischen Räumen mit Produktmetrik auch wieder polnisch ist. 
\begin{definition}[Der Informationsprozess]
Für $\mathbb{X}\in \mathcal{FP}_p$ definieren wir iterativ den assoziierten \emph{Informationsprozess} wie folgt: Für $t=N$ sei $\ip_N(\mathbb{X}):=X_N$ und für $t=N-1,...,1$ sei
\begin{align*}
   \ip_t(\mathbb{X})&:=(\ip_t^-(\mathbb{X}), \ip_t^+(\mathbb{X})) \\
   &:= (X_t, \mathcal{L}(\ip_{t+1}(\mathbb{X}) \vert \mathcal{F}_t^{\mathbb{X}})) \in \mathcal{Z}_t
\end{align*}
$\ip_t(\mathbb{X})$ ist messbar bezüglich $\mathcal{F}_t^\mathbb{X}$, der Informationsprozess ist also adaptiert an die Filtration $\left(\mathcal{F}_t^\mathbb{X}\right)_{t=1}^N$.
\end{definition}
\begin{remark}
Mit unserer Vorarbeit im letzten Kapitel ist diese Konstruktion wohldefiniert: Mit der zweiten Aussage aus Korollar \ref{thm:pmoments} hat induktiv jedes Paar $(X_t, \mathcal{L}(\ip_{t+1}(\mathbb{X})\vert \mathcal{F}_t^\mathbb{X}))$ wieder $p$-te Momente, während die erste Aussage liefert, dass wir die bedingte Verteilung als messbare Abbildung mit Werten in $\mathcal{P}_p$ wählen können.
\end{remark}
Der eben konstruierte kanonische filtrierte Raum und der Informationsprozess entstehen durch eine Schachtelung von Räumen von Verteilungen. Für die Analyse dieser Objekte ist es nützlich einen Operator einzuführen, der diese Schachtelung wieder ``entfaltet'':
\begin{definition}[Unfold-Operator]
Für $1\leq t\leq N$ ist der Unfold-Operator eine Abbildung $\uf_t:\mathcal{P}_p(\mathcal{Z}_t) \rightarrow \mathcal{P}_p(\mathcal{Z}_{t:N})$. Wir definieren ihn durch $\uf_N=\operatorname{id}$ und für $1\leq t\leq N-1$ induktiv durch
$$\uf_{t}(\mu)(dz_{t:N})=\mu(dz_t)\uf_{t+1}(z_t^+)(dz_{t+1:N})$$
Für eine messbare Abbildung $f:\mathcal{Z}_{t:N}\rightarrow \mathbb{R}$ und ein Maß $\mu\in\mathcal{P}_p(\mathcal{Z}_t)$ bedeutet das
$$\int f(z_{t:N}) \uf_t(\mu)(dz_{t:N}) = \int\int ...\int f(z_t,...,z_N)z^+_{N-1}(dz_N)...z_{t}^+(dz_{t+1})\mu(dz_t)$$
\end{definition}
\begin{lemma}\label{thm:bounded_unfold}
    Es gibt Punkte $a_t \in \mathcal{Z}_t$ und Zahlen $n_t \in \mathbb{N}$, $1\leq t\leq N$, sodass für alle $1\leq t\leq N-1$ und $z_t \in \mathcal{Z}_t$ gilt
    $$\int d^p(a_{t+1:n}, z_{t+1:N}) \uf_{t+1}(z_t^+)(dz_{t+1:N})\leq n_td^p(a_t, z_t)$$
    Weiterhin gibt es für eine Verteilung $\mu \in \mathcal{P}_p(\mathcal{Z}_t)$ eine Zahl $n\in \mathbb{N}$ mit
    $$\int d^p(z_{t:N}, a_{t:N})\uf_t(\mu)(dz_{t:N})\leq n\int d^p(z_t, a_t)\mu(dz_t)< \infty$$ 
    Es hat also $\uf_t(\mu)$ $p$-te Momente und der Unfold-Operator ist wohldefiniert.
\end{lemma}
\begin{proof}
Sei $a_N \in \mathcal{X}_N$ und für $1\leq t \leq N-1$ wähle $a_t^-\in\mathcal{X}_t$ und $a_t^+ = \delta_{a_{t+1}}$. 
Wir zeigen die erste Behauptung induktiv. Für $t=N-1$ gilt sie, da
$$\int d^p(a_N, z_N)z_{N-1}^+(dz_N)=\mathcal{W}_p^p(z_{N-1}^+, \delta_{a_N}) \leq d^p(z_{N-1}, a_{N-1})$$
Beachte hierbei wieder, dass es in der Definition der Wasserstein-Metrik bezüglich einem Dirac-Maß nur eine mögliche Kopplung gibt. Nehmen wir nun an, die Aussage gilt für $t+1$. Dann folgt die Aussage für $t$:
\begin{align*}
\int &d^p(a_{t+1:N}, z_{t+1:N}) \uf_{t+1}(z_t^+)(dz_{t+1:N}) \\
&= \int\int d^p(a_{t+2:N}, z_{t+2:N}) + d^p(a_{t+1}, z_{t+1}) \uf_{t+2}(z_{t+1}^+)(dz_{t+2:N})z_t^+(dz_{t+1}) \\
&\leq \int n_{t+1}d^p(a_{t+1}, z_{t+1}) + d^p(a_{t+1}, z_{t+1}) z_t^+(dz_{t+1}) \\
&= \int (n_{t+1}+1)d^p(a_{t+1}, z_{t+1}) z_t^+(dz_{t+1}) \\
&= (n_{t+1}+1)\mathcal{W}_p^p(\delta_{a_{t+1}}, z_t^+) \\
&\leq (n_{t+1}+1)d_p^p(a_t, z_t)
\end{align*}
Mit einem ähnlichen Argument gilt nun auch die zweite Behauptung:
\begin{align*}
\int &d^p(z_{t:N}, a_{t:N})\uf_{t}(\mu)(dz_{t:N}) \\
&= \int\int d^p(z_{t+1:N}, a_{t+1:N})+d^p(z_t, a_t)\uf_{t+1}(z_t^+)(dz_{t+1:N})\mu(dz_t) \\
&\leq \int (n_{t}+1)d^p(z_t, a_t)\mu(dz_t) < \infty
\end{align*}
wobei letzte Geichung gilt, da $\mu \in\mathcal{P}_p(\mathcal{Z}_t)$.
\end{proof}
Das folgende Lemma wird für den Beweis der Lipschitzstetigkeit von $\uf_t$ genutzt:
\begin{lemma}\label{thm:lipschitz_continuous_kernel}
    Seien $\mathcal{A,B}$ polnische Räume und $k: \mathcal{A} \rightarrow \mathcal{P}_p(\mathcal{B})$ lipschitzstetig mit Konstante $L$. Dann ist die Abbildung 
    $$\mathcal{P}_p(\mathcal{A}) \rightarrow \mathcal{P}_p(\mathcal{A} \times \mathcal{B}), \mu \mapsto \mu \otimes k$$
    lipschitzstetig mit Konstante $(1+L^p)^\frac{1}{p}$ (bezüglich der $p$-Produktmetrik auf $\mathcal{A}\times \mathcal{B}$).
\end{lemma}
\begin{proof}
    Seien $\mu, \nu \in \mathcal{P}_p(\mathcal{A})$ und sei $\pi$ eine optimale Kopplung der beiden. Wir können nach Lemma \ref{thm:optimal_coupling} eine messbare Abbildung $\gamma: \mathcal{P}_p(\mathcal{B})\times \mathcal{P}_p(\mathcal{B}) \rightarrow \mathcal{P}_p(\mathcal{B}\times \mathcal{B})$ wählen, sodass $\gamma^{b, \hat{b}}$ eine optimale Kopplung für $b,\hat{b} \in \mathcal{P}_p(\mathcal{B})$ ist. Da $k$ eine messbare Abbildung ist, ist die Abbildung $a,\hat{a} \mapsto \gamma^{k_a, k_{\hat{a}}}=:\gamma^{a,\hat{a}}$ auch messbar und somit ein Kern. Wir betrachten nun 
    $$\Pi(da, db, d\hat{a}, d\hat{b}) = \pi(da,d\hat{a}) \gamma^{a,\hat{a}}(db, d\hat{b})$$
    Für messbare $A\subset\mathcal{A}, B\subset \mathcal{B}$ ist 
    \begin{align*}
        \Pi(A, B, \mathcal{A}, \mathcal{B}) &= \int \mathds{1}_{a \in A} \gamma^{a,\hat{a}}(B, \mathcal{B}) \pi(da, d\hat{a}) \\
        &= \int \mathds{1}_{a \in A} k^a(B) \mu(da) \\
        &= (\mu \otimes k)(A\times B)
    \end{align*}
    und analog für die zweite Marginalie, somit ist $\Pi$ tatsächlich eine Kopplung von $\mu \otimes k$ und $\nu \otimes k$. Nun ist
    \begin{align*}
        \mathcal{W}_p^p(\mu \otimes k, \nu \otimes k) &\leq \int \int d^p(a, \hat{a}) + d^p(b, \hat{b}) \gamma^{a, \hat{a}}(db, d\hat{b}) \pi(da, d\hat{a}) \\
        &= \int d^p(a, \hat{a}) + \mathcal{W}_p^p(k^a, k^{\hat{a}})\pi(da, d\hat{a}) \\
        &\leq \int d^p(a, \hat{a}) + L^p d^p(a, \hat{a}) \pi(da, d\hat{a}) \\
        &= (L^p+1) \int d^p(a, \hat{a}) \pi(da, d\hat{a}) \\
        &= (L^p+1) \mathcal{W}_p^p(\mu, \nu)
    \end{align*}
    und damit $\mathcal{W}_p(\mu \otimes k, \nu \otimes k) \leq (L^p+1)^\frac{1}{p} \mathcal{W}_p(\mu, \nu)$.
\end{proof}
\begin{lemma}\label{thm:properties_unfold}
Für $1\leq t\leq N-1$ gelten die folgenden Eigenschaften:
\begin{enumerate}[(i)]
\item $\uf_t$ ist lipschitzstetig (auch für $t=N$).
\item Für $\mathbb{X}\in\mathcal{FP}_p$ gilt
\begin{align}
    \mathcal{L}(\ip(\mathbb{X})\vert \mathcal{F}_t^\mathbb{X})=\delta_{\ip_{1:t}(\mathbb{X})}\otimes \uf_{t+1}(\ip_t^+(\mathbb{X}))
\end{align}
und 
$$\mathcal{L}(\ip(\mathbb{X}))=\uf_1(\mathcal{L}(\ip_1(\mathbb{X})))$$
Für $f:\mathcal{Z}\rightarrow \mathbb{R}$ beschränkt und Borel-messbar gilt
\begin{align}\label{eq:uf_expected_value}
    \mathbb{E}(f(\ip(\mathbb{X}))\vert \mathcal{F}_t^\mathbb{X}) = \int f(\ip_{1:t}(\mathbb{X}), z_{t+1:N}) \uf_{t+1}(\ip_{t}^+(\mathbb{X}))(dz_{t+1:N})
\end{align}
\end{enumerate}
\end{lemma}
\begin{proof}
\begin{enumerate}[(i)]
    \item Wir zeigen die Behauptung induktiv. $\uf_N$ ist die Identität und lipschitzstetig. Sei nun $\uf_{t+1}$ lipschitzstetig mit Konstante $L$. \\
     Dann ist $\uf_t(\mu)(dz_{t:N}) = \mu(dz_t)\uf_{t+1}(z_t^+)(dz_{t+1:N}) = \mu \otimes k$ für den Kern $k = \uf_{t+1} \circ \pj_+$. Die Projektion auf $z_t^+$ ist kontraktiv, also lipschitzstetig, somit ist $k$ lipschitzstetig als Komposition lipschitzstetiger Funktionen. Nach Lemma \ref{thm:lipschitz_continuous_kernel} ist dann auch $\uf_t$ lipschitzstetig.

    \item $\ip_{1:t}$ ist messbar bezüglich $\mathcal{F}_t^\mathbb{X}$ und $\ip_t^+(\mathbb{X})=\mathcal{L}(\ip_{t+1}(\mathbb{X})\vert \mathcal{F}_t^\mathbb{X})$ genauso. Da $\uf_t$ mit der ersten Aussage des Beweises stetig und somit auch messbar ist, ist $\uf_{t+1}(\ip_t^+(\mathbb{X}))$ ein $\mathcal{F}_t^\mathbb{X}$ messbarer Kern. Mit Lemma \ref{thm:kernel_prod} ist dann auch $\delta_{\ip_{1:t}(\mathbb{X})}\otimes \uf_{t+1}(\ip_t^+(\mathbb{X}))$ ein $\mathcal{F}_t^\mathbb{X}$ messbarer Kern. Mit Lemma \ref{thm:law_expectancy_connection} reicht es also für jedes $t$ Gleichung \ref{eq:uf_expected_value} zu prüfen. Wir zeigen die Behauptung nun durch Induktion über $t$. Für $t=N-1$ ist mit Lemma \ref{thm:determined_kernel} 
    $$\mathcal{L}(\ip(\mathbb{X})\vert \mathcal{F}_{N-1}^\mathbb{X})=\delta_{\ip_{1:N-1}(\mathbb{X})} \otimes \mathcal{L}(\ip_N(\mathbb{X})\vert \mathcal{F}_{N-1}^\mathbb{X})$$
    was genau $\delta_{\ip_{1:N-1}(\mathbb{X})}\otimes \uf_{N}(\ip_{N-1}^+(\mathbb{X}))$ entspricht. Nehmen wir nun an, die Gleichung gilt für ein $2\leq t\leq N-1$. Für $f: \mathcal{Z}\rightarrow\mathbb{R}$ beschränkt und messbar, betrachte
    $$g(z_{1:t})=\int f(z_{1:t}, z_{t+1:N})\uf_{t+1}(z_t^+)(dz_{t+1:N})$$
    Nach Induktionsvoraussetzung gilt $\mathbb{E}(f(\ip(\mathbb{X}))\vert\mathcal{F}_t^\mathbb{X})=g(\ip_{1:t}(\mathbb{X}))$.
    Dann ist 
    \begin{align*}
    \mathbb{E}(f(\ip(\mathbb{X}))\vert\mathcal{F}_{t-1}^\mathbb{X}) &= \mathbb{E}\left( \mathbb{E}(f(\ip(\mathbb{X}))\vert \mathcal{F}_{t}^\mathbb{X}) \vert \mathcal{F}_{t-1}^\mathbb{X}\right) \\
    &= \mathbb{E}(g(\ip_{1:t}(\mathbb{X}))\vert \mathcal{F}_{t-1}^\mathbb{X})
    \end{align*}
    Da wieder mit Lemma \ref{thm:determined_kernel} 
    $$\mathcal{L}(\ip_{1:t}\vert \mathcal{F}_{t-1}^\mathbb{X})=\delta_{\ip_{1:t-1}(\mathbb{X})}\otimes \mathcal{L}(\ip_t\vert \mathcal{F}_{t-1}^\mathbb{X})=\delta_{\ip_{1:t-1}(\mathbb{X})}\otimes \ip_{t-1}^+(\mathbb{X})$$
    folgt also mit Lemma \ref{thm:law_expectancy_connection}
    \begin{align*}
    \mathbb{E}(f(\ip(\mathbb{X}))\vert \mathcal{F}_{t-1}^\mathbb{X}) &= \int g(\ip_{1:t-1}(\mathbb{X}), z_t) \ip_{t-1}^+(\mathbb{X})(dz_t) \\
    &= \int \int f(\ip_{1:t-1}(\mathbb{X}), z_{t:N}) \uf_{t+1}(z_t^+)(dz_{t+1:N})\ip_{t-1}^+(\mathbb{X})(dz_t) \\
    &= \int f(\ip_{1:t-1}(\mathbb{X}), z_{t:N}) \uf_{t}(\ip_{t-1}^+(\mathbb{X}))(dz_{t:N})
    \end{align*}
    und damit die erste Behauptung.
    Der letzte Teil der Behauptung
    $$\mathcal{L}(\ip(\mathbb{X}))=\uf_1(\mathcal{L}(\ip_1(\mathbb{X})))$$
    folgt, da für die Kontruktion von $f$ und $g$ wie oben für $t=1$ folgt
    \begin{align*}
        \mathbb{E}(f(\ip(\mathbb{X}))) &= \mathbb{E}(g(\ip_1(\mathbb{X}))) = \int g(z_1) \mathcal{L}(\ip_1(\mathbb{X}))(dz_1) \\
        &= \int \int f(z_{1:N}) \uf_{2}(z_1^+)(dz_{2:N}) \mathcal{L}(\ip_1(\mathbb{X}))(dz_1) \\
        &= \int f(z_{1:N}) \uf_1(\mathcal{L}(\ip_1(\mathbb{X})))(dz_{1:N})
    \end{align*}
\end{enumerate}
\end{proof}
Mit dieser Darstellung der bedingten Verteilung vom Informationsprozess können wir auch bedingte Verteilungen und Erwartungen von Funktionen in $\ip(\mathbb{X})$ konstruieren:
\begin{lemma}[Der Informationsprozess ist ''self-aware'']\label{thm:self_awareness}
Für jede beschränkte, Borel - messbare (stetige) Funktion $f:\mathcal{Z}\rightarrow \mathbb{R}$ und $1\leq t\leq N$ gibt es eine beschränkte, Borel-messbare (stetige) Funktion $g:\mathcal{Z}_{1:t}\rightarrow \mathbb{R}$ mit 
$$\mathbb{E}(f(\ip(\mathbb{X}))\vert \mathcal{F}_t^\mathbb{X}) = g(\ip_{1:t}(\mathbb{X}))) \quad \text{ für alle }\mathbb{X}\in\mathcal{FP}_p$$
Allgemeiner: Für einen polnischen Raum $\mathcal{A}$ und eine Borel-messbare (stetige) Funktion mit $1$-Wachstum $f:\mathcal{Z}\rightarrow \mathcal{A}$ und $1\leq t\leq N$ gibt es eine messbare (stetige) Funktion $g:\mathcal{Z}_{1:t}\rightarrow \mathcal{P}_p(\mathcal{A})$ mit $1$-Wachstum, sodass
$$\mathcal{L}(f(\ip(\mathbb{X}))\vert \mathcal{F}_t^\mathbb{X})=g(\ip_{1:t}(\mathbb{X}))\quad\text{ für alle } \mathbb{X}\in\mathcal{FP}_p$$
\end{lemma}
\begin{proof}
Sei $f:\mathcal{Z}\rightarrow \mathcal{A}$ messbar (bzw. stetig) mit $1$-Wachstum. Die Abbildung 
$$F:\mathcal{Z}_{1:t}\rightarrow\mathcal{P}_p(\mathcal{Z}), \quad z_{1:t}\mapsto \delta_{z_{1:t}} \otimes \uf_{t+1}(z_t^+)$$ 
ist lipschitzstetig: $z_{1:t} \mapsto \delta_{z_{1:t}}$ ist lipschitz (sogar eine Isometrie) und $z_{1:t} \mapsto \uf_{t+1}(z_t^+)$ ist auch lipschitz nach Lemma \ref{thm:properties_unfold}. Allgemein ist für polnische Räume $\mathcal{A},\mathcal{B}$ die Abbildung 
$$\mathcal{P}_p(\mathcal{A})\times \mathcal{P}_p(\mathcal{B}) \rightarrow \mathcal{P}_p(\mathcal{A}\times \mathcal{B}), (\mu,\nu) \mapsto \mu \otimes \nu$$
lipschitzstetig: Für $(\mu_1, \nu_1), (\mu_2, \nu_2)$ wähle optimale Kopplungen $\pi_1$ von $\mu_1, \mu_2$ und $\pi_2$ von $\nu_1,\nu_2$. Dann ist $\pi_1 \otimes \pi_2$ eine Kopplung von $(\mu_1\otimes \nu_1, \mu_2 \otimes \nu_2)$ und es gilt 
\begin{align*}
    \mathcal{W}_p^p(\mu_1\otimes\nu_1, \mu_2\otimes\nu_2) &\leq \int d^p(a_1, a_2) + d^p(b_1, b_2) (\pi_1 \otimes \pi_2)(da_1, da_2, db_1, db_2) \\
    &= \mathcal{W}_p^p(\mu_1, \mu_2) + \mathcal{W}_p^p(\nu_1, \nu_2) = d^p((\mu_1, \nu_1), (\mu_2, \nu_2))
\end{align*}

$F$ ist also in der Tat lipschitz als Komposition lipschitzstetiger Funktionen. Betrachte nun die Funktion $G(z_{1:t}) = f_*F(z_{1:t})$. Mit Lemma \ref{thm:pushforward_measurable} ist $G$ messbar (und stetig, falls $f$ stetig ist). Da $f_*$ $1$-stetig und $F$ lipschitz ist, hat $G$ $1$-Wachstum. Mit Lemma \ref{thm:pushforward_law} und dem zweiten Teil von Lemma \ref{thm:properties_unfold} gilt die Gleichungskette
$$\mathcal{L}(f(\ip(\mathbb{X}))\vert \mathcal{F}_t^\mathbb{X}) = f_*\mathcal{L}(\ip(\mathbb{X}) \vert \mathcal{F}_t^\mathbb{X}) = f_*F(\ip_{1:t}(\mathbb{X}))=G(\ip_{1:t}(\mathbb{X}))$$

Für die erste Gleichung betrachten wir nun den Spezialfall $\mathcal{A}=\mathbb{R}$. Konstruiere $G$ genau wie zuvor. Dann gilt mit Lemma \ref{thm:law_expectancy_connection}
$$\mathbb{E}(f(\ip(\mathbb{X}))\vert \mathcal{F}_t^\mathbb{X}) = \int a \mathcal{L}(f(\ip(\mathbb{X}))\vert \mathcal{F}_t^\mathbb{X})(da) = \int a G(\ip_{1:t}(\mathbb{X})(da)$$
Die Abbildung $\mathcal{P}_p(\mathbb{R}) \rightarrow \mathbb{R}, \mu \mapsto \int a \mu(da)$ ist stetig, da $\operatorname{id}$ stetig mit $p$-Wachstum ist (beachte $p\geq 1$). Somit ist $g(z_{1:t}) = \int a G(z_{1:t})(da)$ messbar und, falls $f$ (und somit auch $G$) stetig ist, auch stetig. Insgesamt folgt die Behauptung.
\end{proof}

\begin{definition}[Kanonischer filtrierter Prozess]
    Wir nennen $\mathbb{X} \in \mathcal{FP}_p$ einen \emph{kanonischen filtrierten Prozess}, kurz $\mathbb{X} \in \CFP_p$, falls $\mathbb{X}$ der Form
    \begin{equation}\label{eq:def_canonical_filtered_process}
    \mathbb{X} = \left( \mathcal{Z}, \mathcal{F}^\mathcal{Z}, \uf_1(\bar{\mu}), \left(\mathcal{F}_t^\mathcal{Z}\right)_{t=1}^N, Z^-\right)
    \end{equation}
    ist, wobei $\bar{\mu} \in \mathcal{P}_p(\mathcal{Z}_1)$.
\end{definition}
\begin{remark}\label{thm:ip_of_canonical_process}
    Für einen kanonisch filtrierten Prozess $\mathbb{X}$ gilt fast sicher $\ip(\mathbb{X}) = Z_{1:N}$. In der Tat: Es gilt $\ip_N(\mathbb{X}) = Z_N^- = Z_N$, und falls die Aussage für ein $2\leq t+1 \leq N$ gilt, so folgt die Gültigkeit für $t$, da 
    $$\ip_t(\mathbb{X}) = (Z_t^-, \mathcal{L}(Z_{t+1} \vert \mathcal{F}_t^\mathcal{Z}))$$
    und $\mathcal{L}(Z_{t+1} \vert \mathcal{F}_t^\mathcal{Z}) = Z_t^+$: $Z_t^+$ ist messbar bezüglich $\mathcal{F}_t^\mathcal{Z}$ und es gilt 
    $$\uf_1(\mu) \vert_{\mathcal{F}_t^\mathcal{Z}} = \mu(dz_1)dz_1^+(dz_2)...z_{t-1}^+(dz_t)$$
    und 
    $$\uf_1(\mu) \vert_{\mathcal{F}_t^\mathcal{Z}} \otimes z_t^+ = \mu(dz_1)....z_{t-1}^+(dz_t)z_t^+(dz_{t+1}) = \uf_1(\mu)\vert_{\mathcal{F}_{t+1}^\mathcal{Z}} = \mathcal{L}(Z_1,...,Z_{t+1})$$
    Insgesamt ist also 
    $$\ip_t(\mathbb{X}) = (Z_t^-, Z_t^+) = Z_t$$
    und induktiv $\ip(\mathbb{X}) = Z_{1:N}$.
    Insbesondere ist $\mathcal{L}(\ip_1(\mathbb{X})) = \mathcal{L}(Z_1) = \overline{\mu}$.
\end{remark}
\begin{definition}[Assoziierter kanonischer filtrierter Prozess]
    Für $\mathbb{X} \in \mathcal{FP}_p$ sei $\overline{\mathbb{X}} \in \CFP_p$ gegeben durch Gleichung \ref{eq:def_canonical_filtered_process} bezüglich $\bar\mu := \mathcal{L}(\ip_1(\mathbb{X}))$, mit Lemma \ref{thm:properties_unfold} also
    \begin{equation}\label{eq:associated_process}
        \overline{\mathbb{X}} = \left( \mathcal{Z}, \mathcal{F}^\mathcal{Z}, \mathcal{L}(\ip(\mathbb{X})), \left(\mathcal{F}_t^\mathcal{Z}\right)_{t=1}^N, Z^-\right)
    \end{equation}
    Wir nennen $\overline{\mathbb{X}}$ den \emph{zu $\mathbb{X}$ assoziierten kanonischen filtrierten Prozess}. Mit Bemerkung \ref{thm:ip_of_canonical_process} gilt, falls $\mathbb{X}$ bereits ein kanonischer filtrierter Prozess ist, $\overline{\mathbb{X}} = \mathbb{X}$. Wir haben also tatsächlich eine Einbettung von $\mathcal{FP}_p$ in $\CFP_p$.
\end{definition}
% TODO: Aufführung von Faktorlemma und Portmonteau-Theorem in der Einleitung?
\begin{lemma}
Seien $\mathbb{X}, \mathbb{Y} \in \mathcal{FP}_p$ und $\overline{\mathbb{X}}, \overline{\mathbb{Y}}$ die assoziierten kanonischen Prozesse. Dann gelten die folgenden Aussagen:
\begin{enumerate}
\item $(\id, \ip(\mathbb{X}))_*\mathbb{P}^\mathbb{X} \in \cplbc(\mathbb{X}, \overline{\mathbb{X}})$.
\item Für $\pi \in \cplc(\mathbb{X}, \mathbb{Y})$ ist $(\ip(\mathbb{X}), \ip(\mathbb{Y}))_* \pi \in \cplc(\overline{\mathbb{X}}, \overline{\mathbb{Y}})$.
\item Für $\pi \in \cplbc(\mathbb{X}, \mathbb{Y})$ ist $(\ip(\mathbb{X}), \ip(\mathbb{Y}))_* \pi \in \cplbc(\overline{\mathbb{X}}, \overline{\mathbb{Y}})$.
\end{enumerate}
\end{lemma}
\begin{proof}
\begin{enumerate}[(i)]
\item Schreibe $\gamma:=(\id, \ip(\mathbb{X}))_* \mathbb{P}^\mathbb{X}$. Wir prüfen zunächst Kausalität über den dritten Punkt der Charakterisierung von Kausalität in Lemma \ref{thm:causality_characterization}. Sei dazu $V: \mathcal{Z}\rightarrow\mathbb{R}$ beschränkt und $\mathcal{F}_t^\mathcal{Z}$-messbar. Bezüglich $\gamma$ gilt $z=\ip(\mathbb{X})$ fast sicher auf $\Omega^\mathbb{X} \times \mathcal{Z}$ (denn $\gamma\left(z \neq \ip(\mathbb{X})\right) = \mathbb{P}^\mathbb{X}\left(\ip(\mathbb{X}) \neq \ip(\mathbb{X})\right)$). Wir haben also fast sicher $V(z) = V(\ip(\mathbb{X}))$. Da $V$ $\mathcal{F}_t^\mathcal{Z}$-messbar ist, faktorisiert es nach Lemma \ref{thm:factorization_lemma} über die Abbildung $z \mapsto z_{1:t}$. Es gilt also
$$V(\ip(\mathbb{X})) = \tilde{V}(\ip_{1:t}(\mathbb{X}))$$
$\ip(\mathbb{X})$ ist adaptiert an $\mathcal{F}_t^\mathbb{X}$, der Term auf der rechten Seite (und somit auch $V(\ip(\mathbb{X}))$) ist also $\mathcal{F}_t^\mathbb{X}$-messbar. Insgesamt erhalten wir
\begin{align*}
    \mathbb{E}_\gamma(V \vert \mathcal{F}_{N, 0}^{\mathbb{X}, \mathcal{Z}}) &= \mathbb{E}_\gamma(V(\ip(\mathbb{X}))\vert \mathcal{F}_{N, 0}^{\mathbb{X}, \mathcal{Z}}) = V(\ip(\mathbb{X})) \\
    &= \mathbb{E}_\gamma(V(\ip(\mathbb{X})) \vert \mathcal{F}_{t, 0}^{\mathbb{X}, \mathcal{Z}}) = \mathbb{E}_\gamma(V \vert \mathcal{F}_{t, 0}^{\mathbb{X}, \mathcal{Z}})
\end{align*}
Für Kausalität von $\overline{\mathbb{X}}$ zu $\mathbb{X}$ überprüfen wir den zweiten Punkt von Lemma \ref{thm:causality_characterization}: Sei $U: \mathcal{Z}\rightarrow \mathbb{R}$ beschränkt und $\mathcal{F}_N^\mathcal{Z}$-messbar. Wieder gilt $U(z) = U(\ip(\mathbb{X}))$ $\gamma$-fast sicher. Des weiteren gilt
\begin{equation}\label{eq:39i_1}
\mathbb{E}_\gamma(U \vert \mathcal{F}_{t,t}^{\mathbb{X}, \mathcal{Z}}) = \mathbb{E}_\gamma(U \vert \mathcal{F}_{t, 0}^{\mathbb{X}, \mathcal{Z}})
\end{equation}
denn jede $\mathcal{F}_{t,t}^{\mathbb{X}, \mathcal{Z}}$-messbare Funktion $G$ ist fast sicher $\mathcal{F}_{t, 0}^{\mathbb{X}, \mathcal{Z}}$-messbar: $G$ ist messbar bezüglich der kleinsten $\sigma$-Algebra, bezüglich der 
$$(\id, \pj_{1:t}): (\Omega^\mathbb{X}, \mathcal{F}^\mathbb{X})\otimes (\mathcal{Z}, \mathcal{F}^{\mathcal{Z}}) \rightarrow (\Omega^\mathbb{X}, \mathcal{F}_t^\mathbb{X}) \otimes (\mathcal{Z}_{1:t}, \mathcal{B}(\mathcal{Z}_{1:t}))$$
messbar ist. Somit faktorisiert $G$ darüber: $G(\omega, z) = \tilde{G}(\omega, z_{1:t}) = \tilde{G}(\omega, \ip_{1:t}(\mathbb{X}))$ fast sicher. Die letzte Abbildung ist aber $\mathcal{F}_{t,0}^{\mathbb{X}, \mathcal{Z}}$-messbar, da sie der Form $\widetilde{G}\circ H$ für die messbare Abbildung 
$$H: (\Omega^\mathbb{X}, \mathcal{F}_t^\mathbb{X}) \rightarrow (\Omega^\mathbb{X}, \mathcal{F}_t^\mathbb{X}) \otimes (\mathcal{Z}_{1:t}, \mathcal{B}(\mathcal{Z}_{1:t})), \omega \mapsto (\omega, \ip_{1:t}(\omega))$$
ist. 

Weiterhin gilt
\begin{equation}\label{eq:39i_2}
\mathbb{E}_\gamma(U \vert \mathcal{F}_{0, t}^{\mathbb{X}, \mathcal{Z}}) = \mathbb{E}(U \vert \ip_{1:t}(\mathbb{X}))
\end{equation}
denn die linke Seite ist $\mathcal{F}_{0, t}^{\mathbb{X}, \mathcal{Z}}$-messbar und faktorisiert also als $\mathbb{E}_\gamma(U \vert \mathcal{F}_{0,t}^{\mathbb{X}, \mathcal{Z}}) = G(z_{1:t})=G(\ip_{1:t}(\mathbb{X}))$ fast sicher. Somit ist sie messbar bezüglich $\sigma(\ip_{1:t}(\mathbb{X}))$ und umgekehrt faktorisiert die rechte Seite als $F(\ip_{1:t}(\mathbb{X}))=F(z_{1:t})$ und ist messbar bezüglich $\mathcal{F}_{0,t}^{\mathbb{X}, \mathcal{Z}}$.
Zuletzt gilt 
\begin{equation}\label{eq:39i_3}
\mathbb{E}_\gamma(U(\ip(\mathbb{X}))\vert \mathcal{F}_{t, 0}^{\mathbb{X}, \mathcal{Z}}) = \mathbb{E}_\gamma(U(\ip(\mathbb{X})) \vert \ip_{1:t}(\mathbb{X}))
\end{equation}
denn nach Lemma \ref{thm:self_awareness} faktorisiert die linke Seite über $\ip_{1:t}(\mathbb{X})$ und ist damit messbar diesbezüglich. Wir setzen nun die Gleichungen \ref{eq:39i_1}, \ref{eq:39i_2}, \ref{eq:39i_3} zusammen:
\begin{align*}
    \mathbb{E}_\gamma(U \vert \mathcal{F}_{t,t}^{\mathbb{X}, \mathcal{Z}}) &= \mathbb{E}_\gamma(U(\ip(\mathbb{X})) \vert \mathcal{F}_{t, 0}^{\mathbb{X}, \mathcal{Z}}) 
    = \mathbb{E}_\gamma(U(\ip(\mathbb{X}))\vert \ip_{1:t}(\mathbb{X})) \\
    &= \mathbb{E}_\gamma(U(\ip(\mathbb{X}))\vert \mathcal{F}_{0, t}^{\mathbb{X}, \mathcal{Z}}) = \mathbb{E}_\gamma(U \vert \mathcal{F}_{0, t}^{\mathbb{X}, \mathcal{Z}})
\end{align*}
\item Schreibe $\overline\pi:=(\ip(\mathbb{X}), \ip(\mathbb{Y}))_* \pi$ und sei $U:\mathcal{Z}\rightarrow \mathbb{R}$ $\mathcal{F}_{N}^{\mathcal{Z}}$-messbar und beschränkt. $U(\ip(\mathbb{X}))$ ist beschränkt und $\mathcal{F}_N^\mathbb{X}$-messbar. Mit Lemma \ref{thm:causality_characterization} gilt also
\begin{equation}\label{eq:310_0}
    \mathbb{E}_\pi(U(\ip(\mathbb{X}))\vert \mathcal{F}_{t,t}^{\mathbb{X}, \mathbb{Y}}) = \mathbb{E}_\pi(U(\ip(\mathbb{X}))\vert \mathcal{F}_{t, 0}^{\mathbb{X}, \mathbb{Y}}) = \mathbb{E}_\pi(U(\ip(\mathbb{X})) \vert \ip_{1:t}(\mathbb{X}))
\end{equation}
wobei wir in der letzten Gleichheit wieder benutzt haben, dass auf $\mathcal{F}_t^\mathbb{X}$ bedingte Erwartungen nach Lemma \ref{thm:self_awareness} über $\ip_{1:t}(\mathbb{X})$ faktorisieren. Die kleinste $\sigma$-Algebra auf $\Omega^\mathbb{X}$, bezüglich der $\ip(\mathbb{X}): \Omega^{\mathbb{X}} \rightarrow (\mathcal{Z}, \mathcal{F}_t^\mathcal{Z})$ messbar ist, ist genau die kleinste $\sigma$-Algebra bezüglich der $\ip_{1:t}(\mathbb{X})$ messbar ist. Mit Lemma \ref{thm:pushforward_expectancy} gilt also
\begin{equation}\label{eq:310_1}
    \mathbb{E}_{\overline\pi}(U \vert \mathcal{F}_{t, 0}^{\mathcal{Z}, \mathcal{Z}})(\ip(\mathbb{X})) = \mathbb{E}_\pi(U(\ip(\mathbb{X})) \vert \ip_{1:t}(\mathbb{X})) = \mathbb{E}_\pi(U(\ip(\mathbb{X})) \vert \mathcal{F}_{t,t}^{\mathbb{X}, \mathbb{Y}})
\end{equation}
wobei die letzte Gleichheit nach Gleichung \ref{eq:310_0} gilt.
An dem zweiten Term sieht man, dass der dritte Term bereits $\sigma(\ip_{1:t}(\mathbb{X}))$, also auch  $\sigma(\ip_{1:t}(\mathbb{X}), \ip_{1:t}(\mathbb{Y}))$-messbar ist. Gleichzeitig ist $\sigma(\ip_{1:t}(\mathbb{X}), \ip_{1:t}(\mathbb{Y})) \subset \mathcal{F}_{t,t}^{\mathbb{X}, \mathbb{Y}}$, da der Informationsprozess adaptiert ist. Mit der Turmeigenschaft erhalten wir
\begin{align}\label{eq:310_2}
    \begin{split}
\mathbb{E}_\pi(U(\ip(\mathbb{X})) \vert \mathcal{F}_{t,t}^{\mathbb{X}, \mathbb{Y}}) &= \mathbb{E}_\pi\left[\mathbb{E}_\pi(U(\ip(\mathbb{X}))\vert \mathcal{F}_{t,t}^{\mathbb{X}, \mathbb{Y}}) \vert \ip_{1:t}(\mathbb{X}), \ip_{1:t}(\mathbb{Y})\right] \\ 
&= \mathbb{E}_\pi\left(U(\ip(\mathbb{X})) \vert \ip_{1:t}(\mathbb{X}), \ip_{1:t}(\mathbb{Y})\right)
    \end{split}
\end{align}
Nun ist $\sigma(\ip_{1:t}(\mathbb{X}), \ip_{1:t}(\mathbb{Y}))$ die kleinste $\sigma$-Algebra, bezüglich derer 
$$\ip(\mathbb{X}), \ip(\mathbb{Y}): \Omega^\mathbb{X} \times \Omega^\mathbb{Y} \rightarrow (\mathcal{Z}\times \mathcal{Z}, \mathcal{F}_{t,t}^{\mathcal{Z}, \mathcal{Z}})$$
messbar ist, und somit gilt wieder mit Lemma \ref{thm:pushforward_expectancy}
\begin{equation}\label{eq:310_3}
    \mathbb{E}_\pi(U(\ip(\mathbb{X})) \vert \ip_{1:t}(\mathbb{X}), \ip_{1:t}(\mathbb{Y})) = \mathbb{E}_{\overline\pi}(U \vert \mathcal{F}_{t,t}^{\mathcal{Z}, \mathcal{Z}})(\ip(\mathbb{X}, \ip(\mathbb{Y})))
\end{equation}
Wir setzen die Gleichungen \ref{eq:310_1}, \ref{eq:310_2} und \ref{eq:310_3} zusammen und erhalten
$$\mathbb{E}_{\overline\pi}(U \vert \mathcal{F}_{t, 0}^{\mathcal{Z}, \mathcal{Z}})(\ip(\mathbb{X}), \ip(\mathbb{Y})) = \mathbb{E}_{\overline\pi}(U\vert \mathcal{F}_{t,t}^\mathcal{Z,Z})(\ip(\mathbb{X}, \ip(\mathbb{Y})))$$
$\pi$-fast sicher, und somit
$$\mathbb{E}_{\overline\pi}(U \vert \mathcal{F}_{t, 0}^{\mathcal{Z}, \mathcal{Z}}) = \mathbb{E}_{\overline\pi}(U\vert \mathcal{F}_{t,t}^{\mathbb{X}, \mathbb{Y}})$$
$\overline\pi$-fast sicher. Mit Lemma \ref{thm:causality_characterization} ist $\overline\pi$ kausal.
\item Antikausalität ist genau Kausalität mit vertauschten Rollen. Mit dem zweiten Schritt ist somit, falls $\pi$ bikausal ist, $(\ip(\mathbb{X}), \ip(\mathbb{Y}))_*\pi$ sowohl kausal als auch - durch Vertauschen der Rollen - antikausal. Damit ist es bikausal.
\end{enumerate}
\end{proof}
Der folgende Satz ist eine der zentralen Aussagen in dieser Arbeit: 
\begin{theorem}\label{thm:adapted_wasserstein_equalities}     
    Seien $\mathbb{X}, \mathbb{Y} \in \mathcal{FP}_p$ und $\overline{\mathbb{X}}, \overline{\mathbb{Y}} \in \CFP_p$ die assoziierten kanonischen Prozesse. Dann gilt
    $$\mathcal{AW}_p(\mathbb{X}, \mathbb{Y}) = \mathcal{AW}_p(\overline{\mathbb{X}}, \overline{\mathbb{Y}}) = \mathcal{W}_p(\ip_1(\mathbb{X}), \ip_1(\mathbb{Y}))$$
\end{theorem}
Wir beweisen den Satz nur für den Fall, dass $\Omega^\mathbb{X}$ und $\Omega^\mathbb{Y}$ polnisch sind. An welcher Stelle dies wichtig ist und warum die Aussage auch allgemein gilt wird kurz im Beweis besprochen.
\begin{proof}
\begin{enumerate}
\item Betrachte zunächst die erste Gleichheit. Für eine bikausale Kopplung $\pi \in \cplbc(\mathbb{X}, \mathbb{Y})$ gilt nach dem letzten Lemma 
$$\overline{\pi}:=(\ip(\mathbb{X}), \ip(\mathbb{Y}))_*\pi \in \cplbc(\overline{\mathbb{X}}, \overline{\mathbb{Y}})$$
Weiterhin gilt für diese Kopplungen
\begin{align*}
    \mathbb{E}_{\overline{\pi}}(d^p(\overline{X}, \overline{Y}))&=\mathbb{E}_{\overline{\pi}}(d^p(Z^-(z), Z^-(\widetilde{z}))) \\
    &= \mathbb{E}_\pi(d^p(Z^-(\ip(\mathbb{X})), Z^-(\ip(\mathbb{Y})))) \\
    &= \mathbb{E}_\pi(d^p(X, Y))
\end{align*}
Da wir also jede bikausale Kopplung von $\mathbb{X}$ und $\mathbb{Y}$ zu einer von $\overline{\mathbb{X}}$ und $\overline{\mathbb{Y}}$ übertragen können, die die gleichen Kosten erzeugt, gilt auf jeden Fall \\
$\mathcal{AW}_p(\mathbb{X}, \mathbb{Y}) \geq \mathcal{AW}_p(\overline{\mathbb{X}}, \overline{\mathbb{Y}})$. Für die andere Ungleichung benötigen wir eine Approximation von $\overline{\pi} \in \cplbc(\overline{\mathbb{X}}, \overline{\mathbb{Y}})$ durch Verteilungen der Form $(\ip(\mathbb{X}), \ip(\mathbb{Y}))_*\pi$, $\pi \in \cplbc(\mathbb{X}, \mathbb{Y})$. In dem dieser Arbeit zu Grunde liegenden Paper zeigen die Autoren, dass Verteilungen dieser Form tatsächlich dicht in $\cplbc(\overline{\mathbb{X}}, \overline{\mathbb{Y}})$ liegen. Da die Kostenfunktion stetig bezüglich der Wasserstein-Metrik ist ($d^p$ hat $p$-Wachstum und die Evaluationen $Z^-$ sind kontraktiv), wäre damit die erste Gleichheit gezeigt. Der Beweis ist aber sehr technisch, wir betrachten hier daher nur den einfacheren Fall, dass $\Omega^\mathbb{X}$ und $\Omega^\mathbb{Y}$ polnisch sind und somit Disintegration erlauben. 

Sei $\overline{\pi} \in \cplbc(\overline{\mathbb{X}}, \overline{\mathbb{Y}})$. Wir schreiben
$$\gamma := (\id, \ip(\mathbb{X}))_*\mathbb{P}^\mathbb{X} \text{ und } \hat{\gamma}:=(\id, \ip(\mathbb{Y}))_*\mathbb{P}^\mathbb{Y}$$
Da die zugrundeliegenden Räume polnisch sind, können wir die Verteilungen disintegrieren. Wir schreiben $(\gamma_z)_{z \in \mathcal{Z}}:=\mathcal{L}(\id_{\Omega^\mathbb{X}} \vert \ip(\mathbb{X}))$ und $(\hat{\gamma}_{\hat{z}})_{\hat{z} \in \mathcal{Z}}:=\mathcal{L}(\id_{\Omega^\mathbb{Y}} \vert \ip(\mathbb{Y}))$.
Betrachte die Verteilung 
$$\pi(A\times B) := \int \gamma_z(A) \hat{\gamma}_{\hat{z}}(B)\overline{\pi}(dz, d\hat{z})$$
Mit Lemma \ref{thm:pushforward_law} ist $\ip(\mathbb{X})_*\mathcal{L}(\id \vert \ip(\mathbb{X}))= \mathcal{L}(\ip(\mathbb{X}) \vert \ip(\mathbb{X}))= \delta_{z}$ und genauso $\ip(\mathbb{Y})_*\mathcal{L}(\id \vert \ip(\mathbb{Y})) = \delta_{\hat{z}}$. Aus diesem Grund ist
\begin{align*}
(\ip(\mathbb{X}), \ip(\mathbb{Y}))_*\pi(A \times B) &= \int \gamma_z(\ip(\mathbb{X})^{-1}(A)) \hat{\gamma}_{\hat{z}}(\ip(\mathbb{Y})^{-1}(B))\overline{\pi}(dz, d\hat{z}) \\
&= \int \ip(\mathbb{X})_*\gamma_z(A) \ip(\mathbb{Y})_*\hat{\gamma}_{\hat{z}}(B) \overline{\pi}(dz, d\hat{z}) \\
&= \int \delta_z(A) \delta_{\hat{z}}(B)\overline{\pi}(dz, d\hat{z}) \\
&= \overline{\pi}(A \times B)
\end{align*}
Wenn wir nun also beweisen, dass $\pi \in \cplbc(\mathbb{X}, \mathbb{Y})$, so haben wir die erste Gleichheit gezeigt. Zunächst ist $\pi$ überhaupt eine Kopplung,  da 
$$\pi(A\times \Omega^{\mathbb{Y}}) = \int \gamma_z(A) \overline{\pi}(dz, d\hat{z}) = \int \gamma_z(A) \mathcal{L}(\ip(\mathbb{X})(dz) = \mathbb{P}^\mathbb{X}(A)$$
und analog für die zweite Komponente.

Aufgrund der Symmetrie zeigen wir nur die Kausalität von $\pi$ über Lemma \ref{thm:causality_characterization}. Sei dazu $V$ beschränkt und $\mathcal{F}_{t}^{\mathbb{Y}}$-messbar. Wir müssen zeigen, dass 
$$\mathbb{E}_\pi(V \vert \mathcal{F}_{N, 0}^{\mathbb{X}, \mathbb{Y}}) = \mathbb{E}_\pi(V \vert \mathcal{F}_{t, 0}^{\mathbb{X}, \mathbb{Y}})$$
Wir bemerken zunächst, dass wir $(\hat{\gamma}_{\hat{z}})_{\hat{z}\in\mathcal{Z}}$ auch hätten schreiben können als $\mathcal{L}_{\hat{\gamma}}(\id_{\Omega^\mathbb{Y}} \vert \id_{\mathcal{Z}})$, da Disintegration nur eine Frage der Verteilungen im Wertebereich ist, und bezüglich $\hat{\gamma}$ die Variablen $(\id_\Omega^\mathbb{Y}, \id_{\mathcal{Z}})$ genauso verteilt sind wie $(\id_{\Omega^\mathbb{Y}}, \ip(\mathbb{Y}))$ bezüglich $\mathbb{P}^\mathbb{Y}$. Somit ist nach Lemma \ref{thm:law_expectancy_connection}
$$\mathbb{E}_{\hat{\gamma}}(V \vert \mathcal{F}_{0,N}^{\mathbb{Y}, \overline{\mathbb{Y}}}) = \int V(\hat{\omega})\hat{\gamma}_{\hat{z}}(d\hat{\omega}) \quad \text{fast sicher}$$
Da $\hat{\gamma}$ bikausal ist, ist mit Lemma \ref{thm:causality_characterization} die linke Seite $\mathcal{F}_{0,t}^{\mathbb{Y}, \overline{\mathbb{Y}}}$-messbar, somit kann die rechte Seite auf einer $\mathcal{L}(\ip(\mathbb{Y}))$-Nullmenge abgeändert werden um $\mathcal{F}_t^{\overline{\mathbb{Y}}}$-messbar zu werden. Mit der gleichen Argumentation ist, weil $\overline{\pi}$ bikausal ist, auch
$$\mathbb{E}_{\overline{\pi}}\left(\int V(\hat{\omega})\hat{\gamma}_{\hat{z}}(d\hat{\omega}) \left\vert \mathcal{F}_{N,0}^{\overline{\mathbb{X}}, \overline{\mathbb{Y}}}\right.\right)=\int\left(\int V(\hat{\omega})\hat{\gamma}_{\hat{z}}(d\hat{\omega})\right)\overline{\pi}_z(d\hat{z})$$
fast sicher $\mathcal{F}_t^{\overline{\mathbb{X}}}$-messbar, da der Integrand wie vorher festgestellt $\mathcal{F}_t^{\overline{\mathbb{Y}}}$-messbar ist (mit $\overline{\pi}_z$ ist hier die Disintegration von $\overline{\pi}$ bezüglich der ersten Marginalie gemeint). Dass wir den Integranden für die Messbarkeit auf einer $\mathcal{L}(\ip(\mathbb{Y}))$-Nullmenge abändern mussten macht keine Probleme: Für zwei Funktionen $G(\hat{z})$ und $\tilde{G}(\hat{z})$ mit $G=\tilde{G}$ $\mathcal{L}(\ip(\mathbb{Y})$-fast sicher gilt auch $\overline{\pi}_z(G) = \overline{\pi}_z(\tilde{G})$ $\mathcal{L}(\ip(\mathbb{X}))$-fast sicher, da
$$\int \left|\overline{\pi}_z(G) - \overline{\pi}_z(\tilde{G})\right| \mathcal{L}(\ip(\mathbb{X}))(dz) \leq \int \int \left|G(\hat{z}) - \tilde{G}(\hat{z})\right| \overline{\pi}(dz, d\hat{z}) = 0$$

Auch in den weiteren Schritten ist das ''fast sicher'' unproblematisch.

Wieder mit der gleichen Argumentation ist nun für $\gamma_\omega(dz):=\mathcal{L}_\gamma(\id_{\mathcal{Z}} \vert \mathcal{F}_N^{\mathbb{X}})(\omega)$ auch
$$\mathbb{E}_{\gamma}\left( \int \int V(\hat{\omega}) \hat{\gamma}_{\hat{z}}(d\hat{\omega}) \overline{\pi}_z(d\hat{z}) \left\vert \mathcal{F}_{N,0}^{\mathbb{X}, \overline{\mathbb{X}}} \right. \right) = \int\int\int V(\hat{\omega}) \hat{\gamma}_{\hat{z}}(d\hat{\omega}) \overline{\pi}_z(d\hat{z})\gamma_\omega(dz)$$
$\mathcal{F}_t^{\mathbb{X}}$-messbar. Der Term auf der rechten Seite ist aber gerade $\mathbb{E}_\pi(V \vert \mathcal{F}_{N,0}^{\mathbb{X}, \mathbb{Y}})$, weil für jedes $\mathcal{F}_{N}^{\mathbb{X}}$-messbare $U$ gilt
\begin{align*}
    \mathbb{E}_{\pi}\biggl(U(\omega) \int\int\int &V(\hat{\omega}) \hat{\gamma}_{\hat{z}}(d\hat{\omega}) \overline{\pi}_z(d\hat{z})\gamma_\omega(dz) \biggr) \\
    &= \int\int\int\int V(\hat{\omega}) \hat{\gamma}_{\hat{z}}(d\hat{\omega}) \overline{\pi}_z(d\hat{z})U(\omega)\gamma_\omega(dz) \mathbb{P}^{\mathbb{X}}(d\omega) \\
    &= \int\int\int V(\hat{\omega}) \hat{\gamma}_{\hat{z}}(d\hat{\omega}) \overline{\pi}_z(d\hat{z})U(\omega)\gamma(d\omega, dz) \\
    &= \int\int\int\int V(\hat{\omega}) \hat{\gamma}_{\hat{z}}(d\hat{\omega}) \overline{\pi}_z(d\hat{z})U(\omega)\gamma_z(d\omega)\mathcal{L}(\ip(\mathbb{X}))(dz)\\
    &= \int\int\left(\int U(\omega)\gamma_z(d\omega) \int V(\hat{\omega}) \hat{\gamma}_{\hat{z}}(d\hat{\omega}) \right)\overline{\pi}_z(d\hat{z})\mathcal{L}(\ip(\mathbb{X}))(dz) \\
    &= \int\left(\int U(\omega)\gamma_z(d\omega) \int V(\hat{\omega}) \hat{\gamma}_{\hat{z}}(d\hat{\omega}) \right)\overline{\pi}(dz, d\hat{z}) \\
    &= \int U(\omega)V(\hat{\omega})\pi(d\omega, d\hat{\omega})
\end{align*}
Insgesamt ist also $\mathbb{E}_\pi(V\vert \mathcal{F}_{N,0}^{\mathbb{X}, \mathbb{Y}})$ schon $\mathcal{F}_{t,0}^{\mathbb{X}, \mathbb{Y}}$-messbar, also
$$\mathbb{E}_\pi(V \vert \mathcal{F}_{N,0}^{\mathbb{X}, \mathbb{Y}}) = \mathbb{E}_\pi(V \vert \mathcal{F}_{t,0}^{\mathbb{X}, \mathbb{Y}})$$
und somit $\pi\in \cplc(\mathbb{X}, \mathbb{Y})$. Wegen der Symmetrie gilt $\pi \in \cplbc(\mathbb{X}, \mathbb{Y})$ und wie im Beweisanfang besprochen folgt $\mathcal{AW}_p(\mathbb{X}, \mathbb{Y}) = \mathcal{AW}_p(\overline{\mathbb{X}}, \overline{\mathbb{Y}})$.
\item
Wir beweisen nun die zweite Gleichheit. Schreibe $\mu := \mathcal{L}(\ip_1(\mathbb{X}))$ und $\nu := \mathcal{L}(\ip_1(\mathbb{Y}))$. Mit Lemma \ref{thm:causality_kernel_characterization} gilt
$$\mathcal{AW}_p^p(\overline{\mathbb{X}}, \overline{\mathbb{Y}}) = \inf_{\pi \in \cpl(\mu, \nu)} \inf_{(k_{t=1}^{N-1})} \int \sum_{t=1}^{N}d^p(z_t^-, \hat{z}_t^-) (\pi_1 \otimes k_1 \otimes ... \otimes k_{N-1})(dz, d\hat{z})$$
wobei das zweite Infimum über Kerne der Form
\begin{equation}\label{eq:311_0}
k_t: \mathcal{Z}_{1:t}\times \mathcal{Z}_{1:t} \rightarrow \mathcal{P}_p(\mathcal{Z}_{t+1}\times \mathcal{Z}_{t+1}) \text{ mit } k_t^{z_{1:t}, \hat{z}_{1:t}} \in \cpl(z_t^+, \hat{z}_t^+)
\end{equation}
läuft. Nach Lemma \ref{thm:optimal_coupling} können wir für jedes $1 \leq t \leq N-1$ $k_t^*$ als einen solchen messbaren Kern wählen, dass $k_t^{*, z_{1:t}, \hat{z}_{1:t}}$ eine optimale Kopplung von $z_t^+$ und $\hat{z}_t^+$ ist. Für jedes $1\leq t\leq N-1$, $z_{1:t}, \hat{z}_{1:t} \in \mathcal{Z}_{1:t}$ und Kerne wie in \ref{eq:311_0} gilt:
\begin{align}\label{eq:311_1}
    \begin{split}
    d^p(z_t, \hat{z}_t) &= d^p(z_t^-, \hat{z}_t^-) + \mathcal{W}_p^p(z_t^+, \hat{z}_t^+) \\
    &\leq d^p(z_t^-, \hat{z}_t^-) + \int d^p(z_{t+1}, \hat{z}_{t+1}) k_t^{z_{1:t}, \hat{z}_{1:t}}(dz_{t+1}, d\hat{z}_{t+1})
    \end{split}
\end{align}
und es gilt Gleichheit für die Kerne $k_t^*$. Induktiv gilt nun für $1 \leq M \leq N$, dass
$$\int d^p(z_1, \hat{z}_1)\pi_1(dz, d\hat{z}) \leq \int \left(\sum_{t=1}^{M-1} d^p(z_t^-, \hat{z}_t^-)\right) +d^p(z_M, \hat{z}_M)(\pi_1 \otimes k_1 \otimes ... \otimes k_{N-1})(dz, d\hat{z})$$
mit Gleichheit für $k_t^*$: Für $M=1$ sind beide Seiten der Gleichung identisch. Nehme nun an, die Aussage gelte für ein fixes $M$. Dann folgt die Gültigkeit für $M+1$ durch
\begin{alignat*}{2}
\int &d^p(z_1, \hat{z}_1)\pi_1(dz_1, d\hat{z}_1) \\
&\leq \int \sum_{t=1}^{M-1} d^p(z_t^-, \hat{z}_t^-) +d^p(z_M, \hat{z}_M)(\pi_1 \otimes k_1 \otimes ... \otimes k_{M-1})(dz_{1:M}, d\hat{z}_{1:M}) \\
&\leq\int \sum_{t=1}^{M}d^p(z_t^-, \hat{z}_t^-) + \\
    &\quad\int d^p(z_{M+1}, \hat{z}_{M+1}) k_M^{z_{1:M}, \hat{z}_{1:M}}(dz_{M+1}, \hat{z}_{M+1}) (\pi_1 \otimes k_1 \otimes ... \otimes k_{M-1})(dz_{1:M}, d\hat{z}_{1:M})\\
&= \int \sum_{t=1}^{M}d^p(z_t^-, \hat{z}_t^-) + d^p(z_{M+1}, \hat{z}_{M+1})(\pi_1 \otimes ...\otimes k_M)(dz_{1:M+1}, d\hat{z}_{1:M+1})
\end{alignat*}
Die erste Ungleichung gilt nach Induktionsvoraussetzung und auch mit Gleichheit für $k_t^*$. Die zweite Ungleichung (und Gleichheit für $k_t^*$) ist genau Gleichung \ref{eq:311_1}.
Für $M=N$ erhalten wir
$$\int d^p(z_1, \hat{z}_1)\pi_1(dz_1, d\hat{z}_1) = \inf_{(k_t^{N-1})} \sum_{t=1}^Nd^p(z_t^-, \hat{z}_t^-)(\pi_1 \otimes ... \otimes k_{N-1})(dz, d\hat{z})$$
Indem wir das Infimum über $\pi_1 \in \cpl(\mu, \nu)$ auf beiden Seiten nehmen, erhalten wir
$$\mathcal{W}_p(\mu, \nu) = \mathcal{W}_p(\ip_1(\mathbb{X}), \ip_1(\mathbb{Y})) = \mathcal{AW}_p(\overline{\mathbb{X}}, \overline{\mathbb{Y}})$$
\end{enumerate}
\end{proof}

\begin{definition}
    Den Raum
    $$\operatorname{FP}_p := \faktor{\mathcal{FP}_p}{\mathcal{AW}_p}$$

    mit der Metrik $\mathcal{AW}_p$ bezeichnen wir als den \emph{Wassersteinraum stochastischer Prozesse}.
\end{definition}
\begin{remark}
    \begin{enumerate}[(i)]
    \item Auf dem Faktorraum $\operatorname{FP}_p$, indem wir $\mathbb{X} \sim \mathbb{Y}$ setzen falls \\
     $\mathcal{AW}_p(\mathbb{X}, \mathbb{Y})=0$, ist $\mathcal{AW}_p$ tatsächlich eine Metrik: Durch Satz \ref{thm:adapted_wasserstein_equalities} erbt $\mathcal{AW}_p$ die Dreiecksungleichung von der Metrik $\mathcal{W}_p$, da für $\mathbb{X,Y,W} \in \mathcal{FP}_p$ gilt 
    \begin{align*}
        \mathcal{AW}_p(\mathbb{X,W}) &= \mathcal{W}_p(\ip_1(\mathbb{X}), \ip_1(\mathbb{W})) \\
        &\leq \mathcal{W}_p(\ip_1(\mathbb{X}), \ip_1(\mathbb{Y})) + \mathcal{W}_p(\ip_1(\mathbb{Y}), \ip_1(\mathbb{W})) \\
        &= \mathcal{AW}_p(\mathbb{X,Y}) + \mathcal{AW}_p(\mathbb{Y, W})
    \end{align*}
    Hieraus folgt auch die Wohldefiniertheit von $\mathcal{AW}_p$ auf $\operatorname{FP}_p$, da für $\hat{\mathbb{X}}, \mathbb{X,Y} \in \mathcal{FP}_p$ mit $\mathbb{X}\sim\hat{\mathbb{X}}$ gilt
    \begin{align*}
        \mathcal{AW}_p(\mathbb{X,Y}) &= \mathcal{AW}_p(\mathbb{X}, \mathbb{Y}) - \mathcal{AW}_p(\mathbb{X}, \hat{\mathbb{X}}) \\
        &\leq \mathcal{AW}_p(\hat{\mathbb{X}}, \mathbb{Y}) \\
        &\leq \mathcal{AW}_p(\hat{\mathbb{X}}, \mathbb{X}) + \mathcal{AW}_p(\mathbb{X,Y}) \\
        &= \mathcal{AW}_p(\mathbb{X,Y})
    \end{align*}
    Auch Symmetrie und Nichtnegativität werden von $\mathcal{W}_p$ geerbt, und die positive Definitheit folgt direkt aus der Konstruktion der Äquivalenzklassen.

    \item Nach Bemerkung \ref{thm:ip_of_canonical_process} gilt für einen Prozess $\mathbb{X} \in \mathcal{FP}_p$ und den assoziierten Prozess $\overline{\mathbb{X}}$, dass $\mathcal{L}(\ip_1(\overline{\mathbb{X}})) = \mathcal{L}(\ip_1(\mathbb{X}))$, insbesondere folgt also auch aus Satz \ref{thm:adapted_wasserstein_equalities}, dass $\mathbb{X} \sim \overline{\mathbb{X}}$.

    \item Seien $\mathbb{X,Y}\in \CFP_p$ kanonische filtrierte Prozesse, erzeugt durch Verteilungen $\bar{\mu},\bar{\nu} \in \mathcal{P}_p(\mathcal{Z}_1)$. Mit Satz \ref{thm:adapted_wasserstein_equalities} gilt $\mathbb{X} \sim \mathbb{Y}$ genau dann, wenn $\mathcal{L}(\ip_1(\mathbb{X})) = \mathcal{L}(\ip_1(\mathbb{Y}))$. Nach Bemerkung \ref{thm:ip_of_canonical_process} ist $\mathcal{L}(\ip_1(\mathbb{X}))=\bar{\mu}$. Kanonisch filtrierte Prozesse sind damit genau dann gleich in $\FP_p$, wenn sie vollständig identisch sind. 
    \item Mit den Punkten (ii) und (iii) liegt in jeder Äquivalenzklasse genau ein Element aus $\CFP_p$. $\CFP_p$ ist also ein \emph{Repräsentantensystem} für $\FP_p$.
\end{enumerate}
\end{remark}
\begin{lemma}\label{thm:isometric_fp_pz}
    Die Abbildung 
    $$\FP_p \rightarrow \mathcal{P}_p(\mathcal{Z}_1), \quad \mathbb{X} \mapsto \mathcal{L}(\ip_1(\mathbb{X}))$$
     ist ein isometrischer Isomorphismus.
\end{lemma}
\begin{proof}
    Für zwei Prozesse $\mathbb{X,Y} \in \FP_p$ mit $\mathbb{X}\sim \mathbb{Y}$ gilt 
    $$0=\mathcal{AW}_p(\mathbb{X}, \mathbb{Y}) = \mathcal{W}_p(\ip_1(\mathbb{X}), \ip_1(\mathbb{Y}))$$
    also $\mathcal{L}(\ip_1(\mathbb{X})) = \mathcal{L}(\ip_1(\mathbb{Y}))$. Die Abbildung ist somit wohldefiniert. Weiterhin gilt mit Satz \ref{thm:adapted_wasserstein_equalities} für $\mathbb{X,Y} \in \FP_p$
    $$\mathcal{AW}_p(\mathbb{X,Y}) = \mathcal{W}_p(\ip_1(\mathbb{X}), \ip_1(\mathbb{Y}))$$
    Die Abbildung ist also eine Isometrie. 

    Für $\mu \in \mathcal{P}_p(\mathcal{Z}_1)$ betrachten wir den induzierten kanonischen filtrierten Prozess
    $$\mathbb{X} = \left( \mathcal{Z}, \mathcal{F}_t^\mathcal{Z}, \uf_1(\mu), (\mathcal{F}_t^\mathcal{Z})_{t=1}^N, Z^-\right)$$
    Nach Bemerkung \ref{thm:ip_of_canonical_process} gilt $\mathcal{L}(\ip_1(\mathbb{X})) = \mu$. Somit ist die Abbildung auch surjektiv und insgesamt ein isometrischer Isomorphismus.
\end{proof}
Wir betrachten nun auf $\operatorname{FP}_p$ zusätzlich die \emph{schwache adaptierte Topologie}. Damit ist die Topologie gemeint, die wir durch $\mathcal{AW}_p$ erhalten, wenn wir auf den polnischen Räumen $(X_t, d_t)$ die Metrik $d_t$ durch $\widehat{d}_t := d_t \wedge 1$ ersetzen. Das hat keine Auswirkungen auf die Äquivalenzklassen von $\mathcal{FP}_p$, da die Paare $\mathbb{X,Y}$ mit $\mathcal{AW}_p(\mathbb{X,Y})=0$ bezüglich beiden Metriken identisch sind.
\begin{lemma}\label{thm:weak_adapted_semicontinuity}
    Die Abbildung $(\mathbb{X,Y}) \mapsto \mathcal{AW}_p(\mathbb{X,Y})$ ist halbstetig von unten bezüglich der schwachen adaptierten Topologie auf $\operatorname{FP}_p\times \operatorname{FP}_p$.
\end{lemma}
\begin{proof}
    Seien $(\mathbb{X}_n, \mathbb{Y}_n) \rightarrow (\mathbb{X,Y})$ in der schwachen adaptierten Topologie. Dann gilt nach Übergang zu den Räumen $(X_t, \widehat{d}_t)$ $\mathbb{X}_n \rightarrow \mathbb{X}$ bezüglich $\mathcal{AW}_p$ und mit Satz \ref{thm:adapted_wasserstein_equalities} folgt auf diesen Räumen $\ip_1(\mathbb{X}_n) \rightarrow \ip_1(\mathbb{X})$ bezüglich $\mathcal{W}_p$. Mit Lemma \ref{thm:weak_topology_metric} konvergiert $\ip_1(\mathbb{X}_n) \rightarrow \ip_1(\mathbb{X})$ schwach (und alle Überlegungen genauso für $\mathbb{Y}$). Nach Lemma \ref{thm:semicontinuous_wasserstein} ist $\mathcal{W}_p$ halbstetig von unten bezüglich schwacher Konvergenz, somit gilt
    \begin{align*}
        \liminf_{n\rightarrow\infty} \mathcal{AW}_p(\mathbb{X}_n,\mathbb{Y}_n) &= \liminf_{n\rightarrow\infty} \mathcal{W}_p(\ip_1(\mathbb{X}_n), \ip_1(\mathbb{Y}_n)) \\
        &\geq \mathcal{W}_p(\ip_1(\mathbb{X}), \ip_1(\mathbb{Y})) \\
        &= \mathcal{AW}_p(\mathbb{X}, \mathbb{Y})
    \end{align*}
    und damit die Behauptung.
\end{proof}