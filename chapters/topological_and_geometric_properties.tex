\subsection{Kompaktheit in $\FP_p$}

\begin{theorem}[Satz von Prokhorov für $\mathcal{P}_p$]\label{thm:prokhorov_in_pp}
    Sei $(A, d)$ ein polnischer Raum, $\Pi\subseteq\mathcal{P}_p(A)$. Dann sind die Folgenden äquivalent:
    \begin{enumerate}
        \item[(i)] $\Pi$ ist relativ kompakt, das heißt der Abschluss von $\Pi$ ist kompakt in $\mathcal{P}_p(A)$.
        \item[(ii)] $\Pi$ ist straff und für ein fixiertes $a_0 \in A$ und $\varepsilon>0$ existiert eine kompakte Menge $K\subset A$ sodass
        $$\sup_{\pi \in \Pi} \int d^p(a_0, a)\mathds{1}_{a \notin K} \pi(da) \leq \varepsilon$$
        \item[(iii)] $\Pi$ ist straff und für ein fixiertes $a_0 \in A$ und $\varepsilon>0$ existiert ein Ball $B_R(x_0)$ sodass
        $$\sup_{\pi \in \Pi} \int d^p(a_0, a)\mathds{1}_{a \notin B_R(x_0)} \pi(da) \leq \varepsilon$$
    \end{enumerate}
\end{theorem}
\begin{proof}
    Es gilt (ii) $\Rightarrow$ (iii) da jede kompakte Menge beschränkt ist, also in einem Ball liegt. Weiterhin gilt (i) $\Rightarrow$ (ii): Falls $\Pi$ relativ kompakt ist, so ist es auch relativ kompakt aufgefasst in $\mathcal{P}(A)$. Mit dem Satz von Prokhorov in $\mathcal{P}(A)$ (Satz \ref{thm:prokhorov}) ist $\Pi$ straff. Fixiere $a_0 \in A$. Wegen der Straffheit von $\Pi$ existiert für jedes $n \in \mathbb{N}$ eine kompakte Menge $K_n$ mit 
    $$\mu(K_n^c) < \frac{1}{n} \forall \mu \in \Pi$$
    Wir ersetzen $K_n$ durch $\cup_{i=1}^n K_i$ und erhalten, dass die $K_i$ monoton steigen. Angenommen (ii) gilt nicht, dann existiert ein $\varepsilon > 0$ und eine Folge $(\mu_n) \subseteq \Pi$ mit $\int d^p(a_0, a) \mathds{1}_{a \in K_n^c} \mu_n(da) \geq \varepsilon$. Nach Übergang zu einer Teilfolge erhalten wir $\mu_n \rightarrow \mu$ für ein $\mu \in \mathcal{FP}_p$, da $\Pi$ relativ kompakt ist. Es gilt $\mathbb{E}_\mu(d^p(a_0, a)) < \infty$ und somit $\mathbb{E}_\mu(d^p(a_0, a) \mathds{1}_{a \in K_n^c}) \rightarrow 0$ mit dominierter Konvergenz, da $\mu(K_n^c) = \lim_{m\rightarrow \infty} \mu_m(K_n^c) \leq \frac{1}{n}\rightarrow 0$ für $n\rightarrow \infty$. Aber gleichzeitig gilt für $N \in \mathbb{N}$
    $$\mathbb{E}_\mu(d^p(a_0, a) \mathds{1}_{a \in K_N^c}) = \lim_{n\rightarrow\infty} \mathbb{E}_{\mu_n}(d^p(a_0, a) \mathds{1}_{a \in K_N^c}) \geq \limsup_{n\rightarrow \infty} \mathbb{E}_{\mu_n}(d^p(a_0, a) \mathds{1}_{a \in K_n^C}) \geq \varepsilon$$
    weil $\mathds{1}_{K_n^c} \leq \mathds{1}_{K_N^c}$ für fast alle $n$ aufgrund der Monotonie der $K_n$. Insgesamt haben wir einen Widerspruch und erhalten (i) $\Rightarrow$ (ii).

    Als letztes bleibt die Implikation (iii) $\Rightarrow$ (i) zu zeigen. Erfülle also $\Pi$ die Eigenschaft (iii). Sei $(\mu_n)\subset \Pi$. Da $\Pi$ aufgefasst in $\mathcal{P}(A)$ straff, also relativ kompakt ist, erhalten wir nach Übergang zu einer Teilfolge $\mu_n \rightarrow \mu$ schwach für ein $\mu \in \mathcal{P}(A)$. Für $\varepsilon > 0$ wähle einen Ball $B_R(a_0)$, sodass 
    $$\sup_{\pi \in \Pi} \int d^p(a_0, a) \mathds{1}_{a \notin B_R(a_0)} \pi(da) < \varepsilon$$
    Es gilt auch $\mu_n \rightarrow \mu$ auf den Teilmengen $B_R(x_0)$ und $B_R(x_0)^c$ bezüglich der endlichen (nicht unbedingt Wahrscheinlichkeits-) Maße $\mu(\cdot \cap B_R(x_0))$ und $\mu(\cdot \cap B_R(x_0)^c)$ ($B_R(x_0)$ ist offen und $B_R(x_0)^c$ ist abgeschlossen, wir können also das Portmonteau-Theorem für offene bzw. abgeschlossene Mengen auf die Unterräume anwenden). 

    Auf $B_R(a_0)$ ist $d^p(a_0, \cdot)$ stetig und beschränkt und auf $B_R(a_0)^c$ können wir $d^p(a_0, \cdot)$ zu diesem Zweck abschneiden. Insgesamt erhalten wir
    \begin{align*}
        \mathbb{E}_\mu(d^p(a_0, a)) &= \mathbb{E}_\mu(d^p(a_0, a)\mathds{1}_{B_R(a_0)}) + \mathbb{E}_\mu(d^p(a_0, a) \mathds{1}_{B_R(a_0)^c}) \\
        &= \lim_{n\rightarrow \infty} \mathbb{E}_{\mu_n}(d^p(a_0, a) \mathds{1}_{B_R(a_0)}) + \lim_{m\rightarrow\infty}\mathbb{E}_{\mu}(m\wedge d^p(a_0, a) \mathds{1}_{B_R(a_0)^c}) \\
        &= \lim_{n\rightarrow\infty} \mathbb{E}_{\mu_n}(d^p(a_0, a) \mathds{1}_{B_R(a_0)}) + \lim_{m\rightarrow \infty}\lim_{n\rightarrow\infty} \mathbb{E}_{\mu_n}(m\wedge d^p(a_0, a) \mathds{1}_{B_R(a_0)^c}) \\
        &\leq \lim_{n\rightarrow\infty}\mathbb{E}_{\mu_n}(d^p(a_0, a)\mathds{1}_{B_R(a_0)}) + \varepsilon \\
        &\leq \liminf_{n\rightarrow\infty} \mathbb{E}_{\mu_n}(d^p(a_0, a)) + \varepsilon
    \end{align*}
    Aus der vorletzten Ungleichung erhalten wir insbesondere $\mathbb{E}_{\mu}(d^p(a_0, a)) < \infty$, also $\mu \in \mathcal{P}_p(A)$. Umgekehrt ist 
    \begin{align*}
        \limsup_{n\rightarrow\infty}\mathbb{E}_{\mu_n}(d^p(a_0, a)) &\leq \lim_{n\rightarrow\infty} \mathbb{E}_{\mu_n}(d^p(a_0, a) \mathds{1}_{B_R(a_0)}) + \varepsilon \\
        &= \mathbb{E}_{\mu}(d^p(a_0, a)\mathds{1}_{B_R(a_0)}) + \varepsilon \\
        &\leq \mathbb{E}_{\mu}(d^p(a_0, a)) + \varepsilon
    \end{align*}
    Da $\varepsilon$ beliebig gewählt war, folgt $\mu_n \rightarrow \mu$ in $\mathcal{P}_p(A)$.
\end{proof}
\begin{lemma}\label{thm:intensity_compactness}
    Seien $\mathcal{A,B}$ polnische Räume. Wir betrachten die \emph{Intensitätsabbildung} 
    $$\hat{I}: \mathcal{P}_p(\mathcal{A} \times \mathcal{P}_p(B)), \pi \mapsto \int \int \cdot \,\nu(db) \pi(da, d\nu)$$
    Es sind äquivalent:
    \begin{enumerate}
        \item[(i)] $\Pi \subseteq \mathcal{P}_p(\mathcal{A}\times \mathcal{P}_p(\mathcal{B}))$ ist relativ kompakt.
        \item[(ii)] $\hat{I}(\Pi) \subseteq \mathcal{P}_p(\mathcal{A} \times \mathcal{B})$ ist relativ kompakt.
    \end{enumerate}
\end{lemma}
\begin{proof}
    Für die Richtung (i) $\Rightarrow$ (i) reicht es zu zeigen, dass $\hat{I}$ stetig ist. Für eine Folge $\pi_n \rightarrow \pi$ in $\mathcal{P}_p(\mathcal{A} \times \mathcal{P}_p(\mathcal{B}))$ müssen wir also zeigen, dass $\hat{I}(\pi_n) \rightarrow \hat{I}(\pi)$ schwach und ausgewertet an $d^p((a_0, \nu_0), \cdot)$. Schwache Konvergenz können wir nach dem Portmonteau-Theorem an gleichmäßig stetigen $f \in C_b(\mathcal{A} \times \mathcal{B})$ überprüfen. Für solche Funktionen ist das innere Integral
    $$(a, \nu) \mapsto \int f(a, b) \nu(db)$$
    beschränkt und stetig, da für $(a_n, \nu_n) \rightarrow (a, \nu)$ gilt
    \begin{align*}
        \left| \int f(a,b) \nu(db) - \int f(a_n, b) \nu_n(db) \right| \leq  
        &\left|\int  f(a,b) \nu(db) - \int f(a,b) \nu_n(db) \right| \\
        &+\int \left|f(a, b) - f(a_n, b) \right|\nu_n(db)  
    \end{align*}
    Der erste Term auf der rechten Seite konvergiert gegen 0 da $\nu_n \rightarrow \nu$ und der zweite da $f$ gleichmäßig stetig ist. Für einen fixierten Punkt $(a_0, b_0) \in \mathcal{A} \times \mathcal{B}$ ist das innere Integral
    $$ (a, \nu) \mapsto \int d^p(a_0, a) +d^p(b_0, b) \nu(db)$$
    wieder mit der gleichen Argumentation stetig, da $d^p(b_0, b)$ stetig mit $p$-Wachstum ist (für den ersten Term) und gleichmäßig stetig (für den zweiten Term). Weiterhin ist $\int d^p(b_0, b)\nu(db) = \mathcal{W}_p^p(\nu, \delta_{b_0})$, die Funktion hat also $p$-Wachstum.

    In beiden Fällen ist das innere Integral stetig und beschränkt bzw. mit $p$ Wachstum. Da $\pi_n \rightarrow \pi$ in $\mathcal{W}_p$ konvergiert also auch das gesamte Integral und $\hat{I}$ ist stetig.

    Für die Rückrichtung sei $\hat{I}(\Pi) \subset \mathcal{P}_p(\mathcal{A}\times \mathcal{B})$ relativ kompakt. An der zweiten Charakterisierung vom Satz von Prokhorov für $\mathcal{P}_p$, Satz \ref{thm:prokhorov_in_pp}, sieht man dass relative Kompaktheit auf einem Produktraum äquivalent zu relativer Kompaktheit der Marginalien ist. Die Marginalie in $\mathcal{A}$ stimmt bei beiden überein, hier ist also nichts zu zeigen. Wir betrachten also nun die zweite Marginalie von $\Pi$ und wollen das dritte Kriterium aus Satz \ref{thm:prokhorov_in_pp} überprüfen. Zunächst zeigen wir den zweiten Teil:
    \begin{equation}\label{eq:52_0}
        \forall \varepsilon > 0 \exists R>0: \sup_{\pi \in \Pi} \int \mathcal{W}_p(\nu, \delta_{b_0}) \mathds{1}_{\nu \notin B_R(\delta_{b_0})} < \varepsilon
    \end{equation}
    für ein fixes $b_0 \in \mathcal{B}$. Sei $\varepsilon > 0$. $\hat{I}(\Pi)$ ist relativ kompakt, insbesondere ist 
    $$K:=\sup_{\pi \in \Pi} \int d^p(b_0, b) \hat{I}(\pi)(db) = \sup_{\pi \in \Pi} \int \mathcal{W}_p^p(\delta_{b_0}, \nu) \pi(d\nu) < \infty$$
    da $d^p(b_0, b)$ in einem Ball beschränkt ist und das Integral außerhalb von Bällen beliebig klein werden kann. Weiterhin können wir $r>0$ wählen, sodass
    \begin{equation}\label{eq:52_1}
        \sup_{\pi \in \Pi} \int_{B_r(b_0)^c} d^p(b_0, b) \hat{I}(\pi)(db) = \sup_{\pi \in \Pi} \int \int_{B_r(b_0)^c}d^p(b_0, b) \nu(db) \pi(d\nu) < \frac{\varepsilon}{2}
    \end{equation}
    Für die Wahl $R:=\frac{2r^pK}{\varepsilon}$ erhalten wir 
    $$\sup_{\pi \in \Pi} \pi(\nu \notin B_R(\delta_{b_0})) \leq \sup_{\pi \in \Pi} \frac{1}{R} \int_{B_R(\delta_{b_0})^c} \mathcal{W}_p^p(\nu, \delta_{b_0})\pi(d\nu) \leq \frac{K}{R}$$
    und 
    \begin{equation}\label{eq:52_2}
        \sup_{\pi \in \Pi} \int_{B_R(\delta_{b_0})^c} \int_{B_r(b_0)} d^p(b, b_0)\nu(db)\pi(d\nu) \leq \sup_{\pi\in\Pi} \pi(B_R(\delta_{b_0})^c) r^p \leq \frac{\varepsilon}{2}
    \end{equation}
    Indem wir die Gleichungen \ref{eq:52_1} und \ref{eq:52_2} addieren erhalten wir Gleichung \ref{eq:52_0}. Zu zeigen bleibt, dass $\Pi$ straff ist.

    Wir gehen dafür in zwei Schritten vor: Zuerst finden wir eine straffe Teilmenge von $\mathcal{P}_p(\mathcal{B})$ die beliebig groß wird, und anschließend finden wir daraus eine Teilmenge die auch gleichmäßig die $p$-ten Momente integriert, also kompakt ist.

    Zunächst zum ersten Schritt: $\hat{I}(\Pi)$ ist straff, schreibe $K_\varepsilon$ für eine kompakte Menge $K_\varepsilon \subseteq \mathcal{B}$ mit $\sup_{\pi \in \Pi} \hat{I}(\pi)(K_\varepsilon^c) < \varepsilon$. Für $\varepsilon, \eta>0$ und $\pi \in \Pi$ gilt
    \begin{align*}
        \pi\left( \nu \in \mathcal{P}_p(\mathcal{B})\vert \nu(K_{\varepsilon\eta}^c) \geq \eta\right) &\leq \frac{1}{\eta} \int \nu(K_{\varepsilon\eta}^c) \pi(d\nu) \\
        &= \frac{1}{\eta} \hat{I}(\pi)(K_{\varepsilon\eta}^c) \\
        &\leq \varepsilon
    \end{align*}
    Somit gilt für die Menge 
    $$\tilde{K}_\varepsilon := \bigcap_{k\geq 1} \left\{ \nu \in \mathcal{P}_p(\mathcal{B}): \nu(K_{\varepsilon 2^{-k} \frac{1}{k}}^c) < \frac{1}{k}\right\}$$
    für alle $\pi \in \Pi$:
    \begin{align*}
        \pi(\tilde{K}_\varepsilon^c) &= \pi\left(\bigcup_{k\geq 1}\left\{\nu \in \mathcal{P}_p(\mathcal{B}) \vert \nu(K_{\varepsilon\frac{2^-k}{k}}^c) \geq \frac{1}{k}\right\} \right) \\
        &\leq \sum_{k\geq 1} \frac{\varepsilon}{2^k} \\
        &= \varepsilon
    \end{align*}
    und die Menge $\tilde{K}_\varepsilon$ ist straff. 

    Nun zum zweiten Schritt. Für $\varepsilon>0$ können wir also $K_\varepsilon \subset \mathcal{P}_p(\mathcal{B})$ straff wählen mit $\pi(K_\varepsilon^c)<\varepsilon$ für alle $\pi \in \Pi$.
    Es gilt für alle $\pi \in \Pi$ für $n\in\mathbb{N}$ und $R_n>0$
    $$\pi\left(\left\{ \nu: \int_{\{b: d^p(b,b_0)>R_n\}} d^p(b_0, b) \nu(db) \geq \frac{1}{n}\right\}\right) \leq n \int_{\{b:d^p(b,b_0)>R_n\}} d^p(b, b_0) \hat{I}(\pi)(db)$$
    Die rechte Seite wird beliebig klein für große $R_n$, wir können also für $n\in \mathbb{N}$ $R_n$ so groß wählen dass
    $$\sup_{\pi \in \Pi} \pi\left(\left\{ \nu: \int_{b:\{d^p(b,b_0)>R_n\}} d^p(b_0, b) \nu(db) \geq \frac{1}{n}\right\}\right) \leq \frac{\varepsilon}{2^n}$$
    Die Menge
    $$\tilde{K}_\varepsilon := \left\{ \nu \in K_\varepsilon: \int_{\left\{b: d^p(b, b_0)>R_n\right\}}d^p(b, b_0) \nu(db) < \frac{1}{n} \text{ für alle }n \in \mathbb{N}\right\}$$
    erfüllt die zweite Bedingung von Punkt (iii) aus Satz \ref{thm:prokhorov_in_pp} und ist straff als Teilmenge von $K_\varepsilon$, somit ist ihr Abschluss kompakt. Weiterhin gilt für alle $\pi \in \Pi$
    $$\pi(\tilde{K}_\varepsilon) \geq \pi(K_\varepsilon) - \sum_{n\in\mathbb{N}} \frac{\varepsilon}{2^n} \geq 1 - 2\varepsilon$$
    Insgesamt folgt die Behauptung.
\end{proof}
\begin{theorem}
    Eine Teilmenge $\Pi \subseteq \FP_p$ ist relativ kompakt genau dann, wenn $\{\mathcal{L}(X): \mathbb{X} \in \Pi\}$ relativ kompakt in $\mathcal{P}_p(\mathcal{X})$ ist.
\end{theorem}
\begin{proof}
    Nach Lemma \ref{thm:isometric_fp_pz} ist die Abbildung $\mathbb{X} \mapsto \mathcal{L}(\ip_1(\mathbb{X}))$ ein isometrischer Isomorphismus zwischen $\FP_p$ und $\mathcal{P}_p(\mathcal{Z}_1)$. $\Pi \subseteq \FP_p$ ist also genau dann relativ kompakt, wenn $\left\{\mathcal{L}(\ip_1(\mathbb{X})) \vert \mathbb{X} \in \Pi\right\} \subseteq \mathcal{P}_p(\mathcal{Z}_1)$ relativ kompakt ist. Nach der Definition vom Informationsprozess gilt für $1\leq t < N$
    $$\hat{I}\left(\mathcal{L}(X_{1:t-1}, \ip_t(\mathbb{X}))\right) = \hat{I}\left(\mathcal{L}(X_{1:t}, \ip_t^+(\mathbb{X}))\right) = \mathcal{L}(X_{1:t}, \ip_{t+1}(\mathbb{X}))$$
    denn für $A \in \mathcal{B}(\mathcal{X}_{1:t})$ und $B \in \mathcal{B}(\mathcal{Z}_{t+1})$ gilt 
    \begin{align*}
        \int &\mathds{1}_{x_{1:t} \in A} \mathds{1}_{z_{t+1} \in B} \hat{I}(\mathcal{L}(X_{1:t}, \ip_t^+(\mathbb{X})))(dx_{1:t}, dz_{t+1}) \\
        &= \int \int \mathds{1}_{x_{1:t} \in A} \mathds{1}_{z_{t+1} \in B} z_t^+(dz_{t+1}) \mathcal{L}(X_{1:t}, \ip_t^+(\mathbb{X}))(dx_{1:t}, dz_t^+) \\
        &= \int \int \mathds{1}_{X_{1:t}(\omega)\in A} \mathds{1}_{z_{t+1} \in B} \ip_t^+(\mathbb{X})(\omega)(dz_{t+1}) \mathbb{P}^\mathbb{X}(d\omega) \\
        &= \left(\mathbb{P}^{\mathbb{X}} \otimes \mathcal{L}(\ip_{t+1}(\mathbb{X}) \vert \mathcal{F}_t^\mathbb{X})\right)(X_{1:t} \in A, z_{t+1} \in B) \\ 
        &= \mathbb{P}^\mathbb{X}(X_{1:t} \in A, \ip_{t+1}(\mathbb{X}) \in B)
    \end{align*}
    wobei wir in der letzten Umformung benutzt haben, dass $\{X_{1:t} \in A\} \in \mathcal{F}_t^\mathbb{X}$ da der Prozess adaptiert ist. Mit Lemma \ref{thm:intensity_compactness} sind also für $1 \leq t < N$ äquivalent:
    \begin{itemize}
        \item $\left\{ \mathcal{L}(X_{1:t-1}, \ip_t(\mathbb{X})) \vert \mathbb{X} \in \Pi\right\}$ ist relativ kompakt. 
        \item $\left\{ \mathcal{L}(X_{1:t}, \ip_{t+1}(\mathbb{X})) \vert \mathbb{X} \in \Pi\right\}$ ist relativ kompakt. 
    \end{itemize}
    Indem wir diese Aussage iterativ anwenden sind äquivalent:
    \begin{itemize}
        \item $\left\{\mathcal{L}(\ip_1(\mathbb{X})) \vert \mathbb{X} \in \Pi\right\}$ ist relativ kompakt.
        \item $\left\{ \mathcal{L}(X_{1:N-1}, \ip_N(\mathbb{X})) \vert \mathbb{X} \in \Pi \right\} = \left\{ \mathcal{L}(X_{1:N}) \vert \mathbb{X} \in \Pi \right\}$ ist relativ kompakt.
    \end{itemize}
    Da $\Pi$ relativ kompakt ist genau dann wenn die erste dieser Mengen relativ kompakt ist folgt die Behauptung.
\end{proof}

\subsection{Optimal Stopping}
Es sei $c: \mathcal{X} \times \{1,...,N\} \rightarrow \mathbb{R}$ eine Borel-messbare Abbildung, sodass $c_t$ nur von $x_{1:t}$ abhängt. Wir setzen
\begin{equation}
    v_c(\mathbb{X}) := \inf\limits_{\tau \in \ST(\mathbb{X})} \mathbb{E}(c_\tau(X))
\end{equation}
für $\mathbb{X} \in \mathcal{FP}_p$, wobei wir mit $\ST(\mathbb{X})$ die Menge aller $\left(\mathcal{F}_t^\mathbb{X}\right)_{t=1}^N$-Stoppzeiten mit Werten in $\{1,...,N\}$ bezeichnen. In Beispiel \ref{thm:adapted_examples} haben wir bereits gesehen, dass der Wert von $v_c(\mathbb{X})$ nicht von der Wahl eines Repräsentanten in $\FP_p$ abhängt.

\begin{corollary}
    Seien $\mathbb{X,Y} \in \FP_p$ sodass $c_{1:N}(X)$ und $c_{1:N}(Y)$ integrierbar sind. Dann gilt
    $$v_c(\mathbb{X}) - v_c(\mathbb{Y}) \leq \inf_{\pi \in \cplc(\mathbb{X,Y})} \mathbb{E}_\pi \left[\max_{1\leq t\leq N} |c_t(X) - c_t(Y)\right] $$
    Insbesondere gilt: Falls $c_t$ für jedes $1\leq t \leq N$ $L$-Lipschitzstetig ist, so gilt 
    $$|v_c(\mathbb{X}) - v_c(\mathbb{Y})| \leq L \cdot \mathcal{AW}_1(\mathbb{X,Y}) \quad \text{ für alle } \mathbb{X,Y} \in \FP_1$$
\end{corollary}
\begin{proof}
Wir besprechen zunächst die zweite Folgerung. Falls alle $c_t$ $L$-Lipschitzstetig sind, so sind insbesondere $c_t(X)$ integrierbar für $\mathbb{X} \in \FP_1$: Fixiere $a_{1:N} \in \mathcal{X}$, dann ist 
$$\mathbb{E}(|c_t(X)| - |c_t(a_{1:N}|)) \leq \mathbb{E}(|c_t(X) - c_t(a_{1:N})|) \leq L \mathbb{E}(d(X, a_{1:N})) < \infty$$
da $X \in \FP_1$. Antikausalität ist Kausalität mit vertauschten Rollen. Damit folgt aus der ersten Gleichung 
\begin{align*}
    |v_c(\mathbb{X}) - v_c(\mathbb{Y}) | &\leq \inf_{\pi \in \cplbc(\mathbb{X,Y})} \mathbb{E}_\pi(\max_{1\leq t\leq N}|c_t(X) - c_t(Y)|) \\
    &\leq \inf_{\pi \in \cplbc(\mathbb{X,Y})} \mathbb{E}_\pi(\max_{1\leq t\leq N}L\cdot d(X,Y)) \\
    &= L \cdot \inf_{\pi \in \cplbc(\mathbb{X,Y})} \mathbb{E}_\pi(d(X,Y)) \\
    &= L\cdot \mathcal{AW}_1(\mathbb{X,Y})
\end{align*}
 
Nun zur ersten Aussage. Sei dazu $\varepsilon > 0$ und $\tau^*\in \ST(\mathbb{Y})$ eine Stoppzeit mit $\mathbb{E}(c_{\tau^*}(Y)) \leq v_c(\mathbb{Y}) + \varepsilon$. Sei $\pi \in \cplc(\mathbb{X,Y})$ und für $u \in [0,1]$ setze 
$$\sigma_u := \min \left\{ t \in \{1,...,N\}: \pi\left(\tau^*\leq t \vert \mathcal{F}_{N,0}^\mathbb{X,Y}\right) \geq u\right\} $$
$\{\tau^* \leq t\}$ ist $\mathcal{F}_t^\mathbb{Y}$-messbar, mit Lemma \ref{thm:causality_characterization} ist also $\pi\left( \tau^* \leq t \vert \mathcal{F}_{N,0}^\mathbb{X,Y}\right) = \pi\left( \tau^* \leq t \vert \mathcal{F}_{t,0}^\mathbb{X,Y} \right)$ bereits $\mathcal{F}_t^\mathbb{X}$-messbar. Damit ist $\sigma_u \in \ST(\mathbb{X})$ (da insbesondere auch $\sigma_u \leq N$ fast sicher). Daraus folgt 
\begin{align*}
    v_c(\mathbb{X}) &\leq \inf_{u \in [0,1]} \mathbb{E}_\pi(c_{\sigma_u}(X)) \\
    &\leq \int_0^1 \mathbb{E}_\pi(c_{\sigma_u}(X)) du \\
    &= \sum_{t=0}^N \int_0^1 \mathbb{E}_\pi(c_t(X) \mathds{1}_{\sigma_u = t}) du\\
    &= \sum_{t=0}^N \int_0^1 \mathbb{E}_\pi(c_t(X) \mathds{1}_{\pi(\tau^*\leq t-1 \mathcal{F}_{N,0}^\mathbb{X,Y}) < u \leq \pi(\tau^*\leq t \vert \mathcal{F}_{N,0}^\mathbb{X,Y})}) du \\
    &= \sum_{t=0}^N \mathbb{E}_\pi\left[c_t(X) \int_0^1 \mathds{1}_{\pi(\tau^*\leq t-1 \mathcal{F}_{N,0}^\mathbb{X,Y}) < u \leq \pi(\tau^*\leq t \vert \mathcal{F}_{N,0}^\mathbb{X,Y})} du  \right] \\ 
    &= \sum_{t=0}^N \mathbb{E}_\pi\left[c_t(X) \pi\left(t-1 < \tau^* \leq t \vert \mathcal{F}_{N,0}^\mathbb{X,Y}\right) \right]  \\
    &= \sum_{t=0}^N \mathbb{E}_\pi\left[ c_{t}(X) \pi(\tau^*=t \vert \mathcal{F}_{N,0}^\mathbb{X,Y})\right] \\
    &= \sum_{t=0}^N \mathbb{E}_\pi\left[ c_t(X) \pi(\tau^*=t)\right] \\ 
    &= \mathbb{E}_\pi(c_{\tau^*}(X))
\end{align*}
Für die vorletzte Gleichung haben wir benutzt, dass $c_t(X)$ messbar bezüglich $\mathcal{F}_{N,0}^\mathbb{X,Y}$ ist. Aus dieser Gleichung, zusammen mit der Wahl von $\tau^* \in \ST(\mathbb{Y})$ mit $\mathbb{E}_\pi(c_{\tau^*}(Y)) \leq v_c(\mathbb{Y}) + \varepsilon$, erhalten wir
$$v_c(\mathbb{X}) - v_c(\mathbb{Y}) \leq \mathbb{E}_\pi(c_{\tau^*}(X)) - \mathbb{E}_\pi(c_{\tau^*}(Y)) + \varepsilon \leq \mathbb{E}_\pi\left[ \max_{1\leq t\leq N} |c_t(X) - c_t(Y)| \right] + \varepsilon$$
Da $\varepsilon>0$ und $\pi \in \cplc(\mathbb{X,Y})$ beliebig gewählt wurden folgt die Behauptung.
\end{proof}

\subsection{Martingale}
In diesem Abschnitt sei $\mathcal{X}_t := \mathbb{R}^d, 1\leq t \leq N$ und metrisiert durch eine Norm.
\begin{proposition}
    Die Menge 
    $$\mathcal{M}_p := \{\mathbb{X} \in \FP_p: \mathbb{X} \text{ ist ein Martingal} \}$$
    ist abgeschlossen bezüglich $\mathcal{AW}_p$.
\end{proposition}
Wenn $\mathcal{X}_t$ stattdessen eine beschränkte Teilmenge von $\mathbb{R}^d$ wäre, folgt diese Proposition direkt aus Beispiel \ref{thm:adapted_examples} und Proposition \ref{thm:adapted_continuity}. Den allgemeinen Fall zeigen wir über direkte Kopplungen.
% TODO: Huesmann fragen ob das Argument auch allgemein funktioniert
\begin{proof}
    Sei $(\mathbb{X}^n)_{n\in\mathbb{N}} \subset \FP_p$ eine Folge von Martingalen und konvergent gegen $\mathbb{X} \in \FP_p$. Sei $\pi \in \cplbc(\mathbb{X}^n, \mathbb{X})$ und $1\leq t\leq s \leq N$. 
    Nach Lemma \ref{thm:causality_characterization} gilt (da $X_s$ $\mathcal{F}_N^\mathbb{X}$-messbar ist)
    $$\mathbb{E}_\pi(X_s \vert \mathcal{F}_t^\mathbb{X}) = \mathbb{E}_\pi(X_s \vert \mathcal{F}_{t,t}^{\mathbb{X}, \mathbb{X}^n}) \text{ und } \mathbb{E}_\pi(X_s^n \vert \mathcal{F}_t^{\mathbb{X}, \mathbb{X}^n}) = \mathbb{E}_\pi(X_s^n \vert \mathcal{F}_t^{\mathbb{X}^n}) = X_t^n $$
    Die letzte Gleichung gilt, da $\mathbb{X}^n$ ein Martingal ist. Setze $\Delta^n := (X_t^n - X_t) + (X_s - X_s^n)$, dann gilt (da $X_t$ und $X_t^n$ messbar bezüglich $\mathcal{F}_{t,t}^{\mathbb{X}, \mathbb{X}_n}$ sind)
    \begin{align*}
        \mathbb{E}\left[ \left|X_t - \mathbb{E}[X_s \vert \mathcal{F}_t^\mathbb{X}] \right|\right] &= \mathbb{E}_\pi\left[\left|X_t^n - \mathbb{E}_\pi\left[ X_s^n+ \Delta^n \vert \mathcal{F}_{t,t}^{\mathbb{X}, \mathbb{X}^n}\right] \right| \right] \\
        &\leq \mathbb{E}_\pi\left[\left|X_t^n - \mathbb{E}_\pi\left[ X_s^n\vert \mathcal{F}_{t,t}^{\mathbb{X}, \mathbb{X}^n}\right] \right| + \left|\mathbb{E}_\pi\left[ \Delta^n \vert \mathcal{F}_{t,t}^{\mathbb{X}, \mathbb{X}^n}\right] \right|\right] \\
        &= \mathbb{E}_\pi\left[\left|\mathbb{E}_\pi\left[ \Delta^n \vert \mathcal{F}_{t,t}^{\mathbb{X}, \mathbb{X}^n}\right] \right| \right] \\
        &\leq \mathbb{E}_\pi \left[ \mathbb{E}_\pi \left[ \left| \Delta^n\right| \vert \mathcal{F}_{t,t}^{\mathbb{X}, \mathbb{X}^n}\right]\right] \\
        &= \mathbb{E}_\pi\left[ \left| \Delta^n\right| \right] \\
        &\leq \mathbb{E}_\pi\left[\left| X_t^n - X_t \right|\right] + \mathbb{E}_\pi\left[\left| X_s^n - X_s \right|\right] \\
        &\leq \mathbb{E}_\pi\left[|X_t^n - X_t|^p \right]^{\frac{1}{p}} + \mathbb{E}_\pi\left[|X_s^n - X_s|^p \right]^{\frac{1}{p}} \\
        &\leq 2\mathbb{E}_\pi \left[ d^p(X^n, X) \right]^\frac{1}{p}
    \end{align*}
    Da $\pi \in \cplbc(\mathbb{X}, \mathbb{X}^n)$ beliebig war, wird $\mathbb{E}\left[X_t - \mathbb{E}[X_s \vert \mathcal{F}_t^\mathbb{X}] \right]$ beschränkt durch \\ $2\mathcal{AW}_p(\mathbb{X}^n, \mathbb{X})$. Da dieser Term gegen $0$ konvergiert ist $\mathbb{X}$ ein Martingal. 
\end{proof}
\begin{corollary}
    Seien $\mu_1,...,\mu_N \in \mathcal{P}_p(\mathbb{R}^d)$. Die Menge von Martingalen mit festgelegten Marginalien $\mu_{1:N}$
    $$\mathcal{M}_p(\mu_{1:N}) := \{ \mathbb{X} \in \mathcal{M}_p: X_t \sim \mu_t \text{ für alle } 1\leq t \leq N\}$$
    ist kompakt bezüglich $\mathcal{AW}_p$.
\end{corollary}
\begin{proof}
    Für eine Folge $(\mathbb{X}^n)_{n\in \mathbb{N}} \subset \mathcal{M}_p(\mu_{1:N})$ konvergent gegen $\mathbb{X} \in \FP_p$ gilt nach der vorherigen Proposition $\mathbb{X} \in \mathcal{M}_p$. Die adaptierte Wassersteindistanz ist äquivalent zur Wassersteindistanz von $\mathcal{L}(X_{1:N})$ bis darauf, dass sie weniger Mengen für das Infimum zulässt. Somit ist die Wassersteindistanz beschränkt durch die adaptierte Wassersteindistanz und wir erhalten $X^n_{1:N} \rightarrow X_{1:N}$ bezüglich $\mathcal{W}_p$, also insbesondere schwach. Nach Lemma \ref{thm:closed_couplings} konvergieren auch die Marginalien der Verteilung und die Grenzwerte sind eindeutig $\mu_{1:N}$. Insgesamt ist also $\mathbb{X} \in \mathcal{M}_p(\mu_{1:N})$ ud die Menge ist abgeschlossen.

    Zu prüfen bleibt noch, dass die Menge zum Beispiel das zweite Kriterium aus Satz \ref{thm:prokhorov_in_pp} für relative Kompaktheit erfüllt. Zunächst zeigen wir Straffheit. Sei $\varepsilon>0$. Da $\mathcal{X}_{1:N}$ polnisch ist, sind die einzelnen Marginalien $\{\mu_t\}$ straff und wir können kompakte Mengen $K_t$ wählen mit $\mu_t(K_t^c) < \frac{\varepsilon}{N}$. Die Menge $K_{1:N}$ ist kompakt und es gilt für $\pi \in \cpl(\mu_{1:N})$ 
    $$\pi(K_{1:N}^c) \leq \pi\left( \bigcup_{i=1}^N X_{1:i-1} \times K_i^c \times X_{i+1:N}\right) \leq \sum_{i=1}^N \mu(K_i^c) \leq \varepsilon$$
    und insgesamt sind Martingale mit festen Marginalien straff (sogar die Menge aller Kopplungen der Marginalien). 

    Weiterhin liegt für $1\leq t \leq N$ $\mu_t \in \mathcal{P}_p(\mathcal{X}_t)$, das heißt für fixierte $a_t \in \mathcal{X}_t$ ist $\int d^p(a_t, x_t) \mu_t(dx_t) < \infty$. Da $\{\mu_t\}$ straff ist existiert eine Folge von kompakten Menge $K_n \subset \mathcal{X}_t$ mit $\mathds{1}{K_n^c} \rightarrow 0$ fast sicher. Sei $\varepsilon>0$. Mit dominierter Konvergenz können wir ein kompaktes $K_t$ auswählen, sodass
    $$\int d^p(a_t, x_t) \mathds{1}_{K_t^c} \mu_t(dx_t) < \varepsilon$$
    Die Menge $K_{1:N}$ ist wieder kompakt und es gilt für beliebige Kopplungen $\pi \in \cpl(\mu_{1:N})$
    $$\int d^p(a_{1:N}, x_{1:N}) \mathds{1}_{K_{1:N}^c} \pi(dx_{1:N}) \leq \sum_{t=1}^N \int d^p(a_t, x_t) \mathds{1}_{K_t^c} \mu_t(dx_t) \leq N\varepsilon$$
    Insgesamt sind also die Kopplungen von $\mu_{1:N}$ relativ kompakt und die Martingale mit diesen Marginalien abgeschlossen. Damit ist $\mathcal{M}_p(\mu_{1:N})$ kompakt.
\end{proof}

\subsection{Die Doob-Zerlegung}
Für einen integrierbaren Prozess $\mathbb{X} \in \mathcal{FP}_p$ betrachten wir die Prozesse 
$$A_1^\mathbb{X}=0, \quad A_t^\mathbb{X} = \sum_{s=2}^{t} \mathbb{E}\left[ X_{s} - X_{s-1} \vert \mathcal{F}_s^\mathbb{X}\right]$$
und $M_t^\mathbb{X} = X_t - A_t^\mathbb{X}$ bezüglich der Filtration von $\mathbb{X}$. Bis auf Beschränktheit ist $A_t^\mathbb{X}$ der Form einer adaptierten Funktion von Grad 1. Mit dem üblichen Truncation-Argument ist diese Zerlegung also unabhängig von der Wahl eines Vertreters in $\FP_p$. $A_t^\mathbb{X}$ ist \emph{previsibel}, das heißt $A_{t+1}$ ist messbar bezüglich $\mathcal{F}_t$. 

Weiterhin ist $M_t^\mathbb{X} = X_t - \sum_{s=2}^{t}\mathbb{E}[X_s - X_{s-1} \vert \mathcal{F}_s^\mathbb{X}]$, das heißt 
$$\mathbb{E}\left[ M_{t+1} - M_t \vert \mathcal{F}_t\right] = \mathbb{E}\left[ X_{t+1} - X_t - \mathbb{E}[X_{t+1} - X_t \vert \mathcal{F}_t] \vert \mathcal{F}_t \right] = 0$$
$M_t^\mathbb{X}$ ist also ein Martingal. Den Prozess 
$$\mathbb{D}^\mathbb{X} := \left( \Omega^\mathbb{X}, \mathcal{F}^\mathbb{X}, \left( \mathcal{F}_t^\mathbb{X}\right)_{t=1}^N, \mathbb{P}^\mathbb{X}, (M_t^\mathbb{X}, A_t^\mathbb{X})_{t=1}^N \right)$$
nennen wir die \emph{Doob-Zerlegung} von $\mathbb{X}$. 

Die drei Eigenschaften
\begin{enumerate}
    \item $(M_t)_{t=1}^N$ ist ein Martingal.
    \item $(A_t)_{t=1}^N$ ist previsibel mit $A_0=0$. 
    \item $X_t = M_t + A_t$
\end{enumerate}
legen die Zerlegung schon fast sicher fest: Aus diesen drei Eigenschaften folgt
$$\mathbb{E}[X_{t+1} - X_t \vert \mathcal{F}_t^\mathbb{X}] = \mathbb{E}[M_{t+1} - M_t \vert \mathcal{F}_t^\mathbb{X}] + \mathbb{E}[A_{t+1} - A_t \vert \mathcal{F}_t^\mathbb{X}] = A_{t+1} - A_t$$
Da $A_0=0$ sind also die $A_t$ schon fast sicher durch die $X_t$ bestimmt. Wegen dem dritten Punkt sind dadurch auch $M_t = X_t - A_t$ fast sicher festgelegt. 
\begin{proposition}
    Seien $\mathbb{X,Y}\in \FP_p$ integrierbar. Dann gilt
    $$2^{\frac{1-p}{p}}\mathcal{AW}_p(\mathbb{X,Y}) \leq \mathcal{AW}_p(\mathbb{D}^\mathbb{X}, \mathbb{D}^\mathbb{Y}) \leq c\cdot \mathcal{AW}_p(\mathbb{X,Y})$$
    für eine konstante $c=c(p,N)$, wobei wir auf dem Werteraum von $\mathbb{D}^\mathbb{X}$ die $p$-Produktmetrik der Werteräume von $M^\mathbb{X}$ und $A^\mathbb{X}$ betrachten.
\end{proposition}
%TODO: Metrik / Betrag aufräumen
\begin{proof}
    Da die Wahrscheinlichkeitsräume und Filtrationen von $\mathbb{D}^\mathbb{X}$ und $\mathbb{D}^\mathbb{Y}$ mit denen von $\mathbb{X}$ und $\mathbb{Y}$ übereinstimmen sind, ist $\cplbc(\mathbb{X,Y}) = \cplbc(\mathbb{D}^\mathbb{X}, \mathbb{D}^\mathbb{Y})$.

    Es gilt 
    \begin{align*}
        d^p(X,Y) &= \sum_{t=1}^N \left|M_t^\mathbb{X} + A_t^\mathbb{X} - M_t^\mathbb{Y} - A_t^\mathbb{Y}\right|^p \leq \sum_{t=1}^N \left(|M_t^\mathbb{X} - M_t^\mathbb{Y}| + |A_t^\mathbb{X} - A_t^\mathbb{Y}| \right)^p \\
        &\leq \sum_{t=1}^N 2^{p-1} \left( |M_t^\mathbb{X} - M_t^\mathbb{Y}|^p + |A_t^\mathbb{X} - A_t^\mathbb{Y}|^p\right) \\
        &= 2^{p-1}(d^p(M^\mathbb{X}, M^\mathbb{Y}) + d^p(A^\mathbb{X}, A^\mathbb{Y})) = 2^{p-1}\left(d^p\left((M^\mathbb{X}, A^\mathbb{X}), (M^\mathbb{Y}, A^{\mathbb{Y}}) \right) \right)
    \end{align*}
    Für die zweite Ungleichung haben wir ein Konvexitätsargument benutzt: $x\mapsto x^p, p\geq 1$ ist konvex, für $a,b\in \mathbb{R}$ gilt also
    \begin{equation}\label{eq:convexity_argument1}
        (a+b)^p= 2^p(\frac{a+b}{2})^p \leq 2^p(\frac{a^p+b^p}{2}) = 2^{p-1}(a^p+b^p)
    \end{equation}
    Wir erhalten hierdurch
    $$2^{1-p}\mathcal{AW}_p^p(\mathbb{X,Y}) \leq \mathcal{AW}_p^p(\mathbb{D}^\mathbb{X}, \mathbb{D}^\mathbb{Y})$$
    und damit auch
    $$2^{\frac{1-p}{p}}\mathcal{AW}_p(\mathbb{X,Y}) \leq \mathcal{AW}_p(\mathbb{D}^\mathbb{X}, \mathbb{D}^\mathbb{Y})$$
    
    Für die zweite Ungleichung bemerken wir zunächst, dass mit dem gleichen Argument 
    $$|M^\mathbb{X} - M^\mathbb{Y}|^p = |(X-Y) + (A^\mathbb{Y} - A^\mathbb{X})|^p \leq 2^{p-1}(|X-Y|^p + |A^\mathbb{X} - A^\mathbb{Y}|^p)$$
    und somit 
    \begin{equation}\label{eq:doob_ineq} 
        |(M^\mathbb{X}-M^\mathbb{Y}, A^\mathbb{X} - A^\mathbb{Y})|^p \leq 2^{p-1}|X-Y|^p + (2^{p-1}+1)|A^\mathbb{X}-A^\mathbb{Y}|^p
    \end{equation}
    Nach der Definition von $A^\mathbb{X}$ und $A^\mathbb{Y}$ gilt für $2\leq t\leq N$ mit Lemma \ref{thm:causality_characterization} für $\pi \in \cplbc(\mathbb{X,Y})$ 
    $$\mathbb{E}_\pi\left[ X_t - X_{t-1} \vert \mathcal{F}_t^\mathbb{X}\right] = \mathbb{E}_\pi\left[X_t-X_{t-1} \vert \mathcal{F}_{t,t}^\mathbb{X,Y}\right]$$
    und somit
    $$A_t^\mathbb{X} - A_t^\mathbb{Y} = \sum_{s=2}^t \mathbb{E}_\pi\left[ X_s - X_{s-1} - (Y_s - Y_{s-1}) \vert \mathcal{F}_{s,s}^\mathbb{X,Y} \right]$$ 

    Das Konvexitätsargument aus Gleichung \ref{eq:convexity_argument1} kann man auch mit $n$ Summanden durchführen: 
    $$\left(\sum_{i=1}^N a_i\right)^p = N^{p} \left(\sum_{i=1}^N \frac{a_i}{N}\right)^p \leq N^p \sum_{i=1}^N \frac{a_i^p}{N} = N^{p-1} \sum_{i=1}^N a_i^p$$
    Damit erhalten wir 
    \begin{align*}
        \mathbb{E}_\pi(|A^\mathbb{X} - A^\mathbb{Y}|^p) &= \sum_{t=2}^N\mathbb{E}_\pi(|A_t^\mathbb{X} - A_t^\mathbb{Y}|^p) \\
        &= \sum_{t=2}^N \mathbb{E}_\pi\left(\left| \sum_{s=2}^t \mathbb{E}_\pi\left[X_s - X_{s-1} - (Y_s - Y_{s-1} \vert \mathcal{F}_{s,s}^\mathbb{X,Y}) \right]\right|^p \right) \\
        &\leq \sum_{t=2}^N t^{p-1} \sum_{s=2}^t \mathbb{E}_\pi\left[\mathbb{E}_\pi\left[ \left|X_s - X_{s-1} - (Y_s - Y_{s-1}) \vert \mathcal{F}_{s,s}^\mathbb{X,Y} \right|\right]^p\right] \\
        &\leq \sum_{t=2}^N t^{p-1}\sum_{s=2}^t \mathbb{E}_\pi\left[ \left|X_{s}-X_{s-1} - (Y_s - Y_{s-1}) \right|^p\right] \\
        &\leq \sum_{t=2}^N t^{p-1}2^{p-1}\sum_{s=2}^t \mathbb{E}_\pi(|X_s - Y_{s}|^p + |X_{s-1}-Y_{s-1}|^p) \\
        &\leq \sum_{t=2}^N t^{p-1}2^{p} \sum_{s=1}^N \mathbb{E}_\pi(|X_s - Y_s|^p) \\
        &\leq N^p2^p \mathbb{E}_\pi(|X-Y|^p)
    \end{align*}
    Zusammen mit Gleichung \ref{eq:doob_ineq} ist also für jede bikausale Kopplung $\pi \in \cplbc(\mathbb{X,Y})$
    $$\mathbb{E}_\pi\left[\left|(M^\mathbb{X}-M^\mathbb{Y}, A^\mathbb{X}-A^\mathbb{Y}) \right|^p \right]  \leq c\cdot \mathbb{E}_\pi \left[ \left| X-Y\right|^p\right]$$
    für ein $c=c(N,p)$ und da die bikausalen Kopplungen übereinstimmen
    $$\mathcal{AW}_p(\mathbb{D}^\mathbb{X}, \mathbb{D}^\mathbb{Y}) \leq c^\frac{1}{p} \mathcal{AW}_p(\mathbb{X,Y})$$
\end{proof}
